\section{From Microscopic to Macroscopic Behavior}

We explore the fundamental differences between microscopic and macroscopic systems, note that bouncing balls come to rest and hot objects cool, and discuss how the behavior of macroscopic systems is related to the behavior of their microscopic constituents.
Computer simulations are introduced to demonstrate the general qualitative behavior of macroscopic systems.


\subsection{Introduction}

Our goal is to understand the properties of macroscopic systems, that is, systems of many electrons, atoms, molecules, photons, or other constituents.
Examples of familiar macroscopic objects include systems such as air in your room, a glass of water, a coin, and a rubber band -- examples of a gas, liquid, solid, and polymer, respectively.
Less familiar macroscopic systems include superconductors, cell membranes, the brain, the stock market, and neutron stars.

We will find that the type of questions we ask about macroscopic systems differ in important ways from the questions we ask about systems that we treat microscopically.
For example, consider the air in your room.
Have you ever wondered about the trajectory of a particular molecule in the air?
Would knowing that trajectory be helpful in understanding the properties of air?
Instead of questions such as these, examples of questions that we do ask about macroscopic systems include the following:
\begin{enumerate}
	\item How does the pressure of a gas depend on the temperature and the volume of its container?
	\item How das a refrigerator work?
	How can we make it more efficient?
	\item How much energy do we need to add to a kettle of water to change it to steam?
	\item Why are the properties of water different from those of steam, even though water and steam consist of the same type of molecules?
	\item How and why does liquid freeze into a particular crystalline structure?
	\item Why does helium have a superfluid phase at very low temperatures?
	Why do some materials exhibit zero resistance to electrical current at sufficiently low temperatures?
	\item In general, how do the properties of a system emerge from its constituents?
	\item How fast does the current in a river have to be before its flow changes from laminar to turbulent?
	\item What will the weather be tomorrow?
\end{enumerate}
These questions can be roughly classified into three groups.
Questions 1 - 3 are concerned with macroscopic properties such as pressure, volume, and temperature and processes related to heating and work.
These questions are relevant to thermodynamics which provides a framework for relating the macroscopic properties of a system to one another.
Thermodynamics is concerned only with macroscopic quantities and ignores the microscopic variables that characterize individual molecules.
For example, we will find that understanding the maximum efficiency of a refrigerator does not require a knowledge of the particular liquid used as the coolant.
Many of the applications of thermodynamics are to engines, for example, the internal combustion engine and the steam turbine.

Questions 4 - 7 relate to understanding the behavior of macroscopic systems starting from the atomic nature of matter.
For example, we know that water consists of molecules of hydrogen and oxygen.
We also know that the laws of classical and quantum mechanics determine the behavior of molecules at the microscopic level.
The goal of statistical mechanics is to begin with the microscopic laws of physics that govern the behavior of the constituents of the system and deduce the properties of the system as a whole.
Statistical mechanics is a bridge between the microscopic and macroscopic worlds.

Question 8 also relates to a macroscopic system, but temperature is not relevant in this case.
Moreover, turbulent flow continually changes in time.
Question 9 concerns macroscopic phenomena that change with time.
Although there has been progress in our understanding of time-dependent phenomena such as turbulent flow and hurricanes, our understanding of such phenomena is much less advanced than our understanding of time-independent systems.
For this reason we will focus our attention on systems whose macroscopic properties are independent of time and consider questions such as those in Questions 1 - 7.


\subsection{Some Qualitative Observations}

We begin our discussion of macroscopic systems by considering a glass of hot water.
We know that if we place a glass of hot water into a large cold room, the hot water cools until its temperature equals that of the room.
This simple observation illustrates two important properties associated with macroscopic systems -- the importance of temperature and the ``arrow'' of time.
Temperature is familiar because it is associated with the physiological sensations of hot and cold and is important in our everyday experience.
