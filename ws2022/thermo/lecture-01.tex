\section{Equilibrium and entropy}


\subsection{Thermodynamic systems}

\begin{definition}[System]
	A \emph{system} is  a finite quantity of matter or a prescribed region of space.
\end{definition}

\begin{definition}[Boundary]
	The actual or hypothetical envelope enclosing the system is the \emph{boundary} of the system.
\end{definition}

\begin{remark}
	\begin{enumerate}
		\item The boundary may be fixed or it may move, as and when a system containing a gas is compressed or expanded.
		\item The boundary may be real or imaginary.
		It is not difficult to envisage a real boundary but an example of imaginary boundary would be one drawn around a system consisting of the fresh mixture about to enter the cylinder of an internal combustion engine together with the remanants of the last cylinder charge after the exhaust process.
	\end{enumerate}
\end{remark}

\begin{definition}[Closed/Open/Isolated/Adiabatic system]
	\begin{enumerate}
		\item If the boundary of the system is impervious to the flow of matter, it is called a \emph{closed system}.
		\item An \emph{open system} is one in which matter flows into or out of the system.
		\item An \emph{isolated system} is that system which exchanges neither energy nor matter with any other system or with environment.
		\item An \emph{adiabatic system} is one which is thermally insulated for its surroundings.
		It can, however, exchange work with its surroundings.
		If it does not it becomes an isolated system.
	\end{enumerate}
\end{definition}

\begin{definition}[Phase]
	A \emph{phase} is a quantity of matter which is homogeneous throughout in chemical composition and physical structure.
\end{definition}

\begin{definition}[Homo-/Heterogeneous system]
	\begin{enumerate}
		\item A system which consists of a single phase is termed as \emph{homogeneous system}.
		\item A system which consists of two or more phases is called a \emph{heterogeneous system}.
	\end{enumerate}
\end{definition}

\subsection{Macroscopic and microscopic points of view}

Thermodynamic studies are undertaken by the following two different approaches.
\begin{enumerate}
	\item Macroscopic approach -- (Macro means big or total)
	\item Microscopic approach -- (Micro means small)
\end{enumerate}
These approaches are discussed (in a comparative way) below:

\emph{TODO}

\subsection{Thermodynamic equilibrium}

\begin{definition}[Thermodynamic equilibrium]
	A system is in \emph{thermodynamic equilibrium} if the temperature and pressure at all points are same -- there should be no velocity gradient -- the chemical equilibrium is also necessary.
\end{definition}

\begin{remark}
	Systems under temperature and pressure equilibrium but not under chemical equilibrium are simetimes said to be in \emph{metastable equilibirium conditions}.
	It is only under thermodynamic equilibrium conditions that the properties of a system can be fixed.
\end{remark}

Thus for attaining a state of thermodynamic equilibrium the following three types of equilibrium states must be achieved:

\begin{definition}[Thermal/mechanical/chemical equilibrium]
	\begin{enumerate}
		\item \emph{Thermal equilibrium}: The temperature of the system does not change with time and has the same value at all points of the system.
		\item \emph{Mechanical equilibrium}: There are no unbalanced forces within the system or between the surroundings.
		The pressure in the system is the same at all points and does not change with respect to time.
		\item \emph{Chemical equilibrium}: No chemical reaction takes place in the system and the chemical composition which is the same throughout the system does not vary with time.
	\end{enumerate}
\end{definition}

\subsection{Porperties of systems}

\begin{definition}[Property]
	A \emph{property} of a system is a characteristic of the system which depends upon its state, but not upon how the state is reached.
	There are two sorts of properties:
	\begin{enumerate}
		\item \emph{Intensive properties} do not depend on the mass of the system.
		\item \emph{Extensive properties} depend on the mass of the system.
	\end{enumerate}
\end{definition}

\begin{example}
	Temeprature and pressure are intensive properties, while volume is an extensive property.
	Extensive properties are often divided by mass associated with them to obtain the intesive properties.
	For example, if the volume if a system of mass \(m\) is \(V\), then the specific volume of matter within the system is \(\frac{V}{m} = v\) which is an intensive property.
\end{example}


