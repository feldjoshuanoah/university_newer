\documentclass[english]{exercise}

\DeclareMathOperator*{\cond}{cond}
\DeclareMathOperator*{\diag}{diag}
\DeclareMathOperator*{\dist}{dist}
\DeclareMathOperator*{\esssup}{ess\,sup}
\DeclareMathOperator*{\vol}{vol}


\title{Exercise 6}
\author{Joshua Feld (406718)}
\professor{Prof. Kobbelt}
\course{Data Analysis and Visualization}

\begin{document}
	\maketitle


	\section{}

	\begin{quote}
		Assume the scene consists of a set of triangles.
		\begin{enumerate}
			\item Why can't we order these triangles such that closer triangles come before triangles further away?
			\item The \(z\)-buffer can be described as a map. What are domain and image of this map?
			\item Can the \(z\)-buffer-method deal with transparent objects? Explain your reasoning.
		\end{enumerate}
	\end{quote}

	\begin{enumerate}
		\item Possible conterexample: three triangles \(t_1, t_2, t_3\), such that \(t_1\) lies partly before \(t_2\) and partly behind \(t_3\).
		Similarly, \(t_2\) lies partly before \(t_3\).
		\item The \(z\)-buffer assigns each pixel a depth value.
		if the screen has width \(W \in \N\) and height \(H \in \N\), the domain is \(\brackets*{1, W} \times \brackets*{1, H}\).
		The depth values are always positive and possible inifinte, so \(\R^+ \cup \braces*{\infty}\) is an appropriate range.
		With further knowledge about the concrete image discretization, the range could be further restricted.
		\item No.
		To properly render transparent objects, we need to know the order of the objects.
		Since the \(z\)-buffer-method iterates in a random order over the triangles, we don't obtain this ordering.
	\end{enumerate}


	\section{}

	\begin{quote}
		Let \(A := \parentheses*{0, 0}\), \(B := \parentheses*{6, 0}\), and \(C := \parentheses*{0, 4}\) be points in the plane.
		We denote the coefficients of the barycentric combination \(\alpha A + \beta B + \gamma C\) as the row \(\parentheses*{\alpha, \beta, \gamma}\).
		\begin{enumerate}
			\item Compute and plot the points associated to these coefficients:
			\begin{align*}
				&\parentheses*{\frac{1}{3}, \frac{1}{3}, \frac{1}{3}}, && \parentheses*{0, \frac{1}{2}, \frac{1}{2}}, && \parentheses*{\frac{2}{3}, \frac{1}{3}, 0}, && \parentheses*{\frac{5}{6}, \frac{1}{6}, 0}, && \parentheses*{\frac{11}{12}, \frac{1}{12}, 0},\\
				&\parentheses*{\frac{1}{2}, \frac{1}{4}, \frac{1}{4}}, && \parentheses*{-1, 1, 1}, && \parentheses*{1, 1, -1}, && \parentheses*{1, -1, 1}, && \parentheses*{3, -1, -1}.
			\end{align*}
			\item Compute the barycentric coefficients for any point \(\parentheses*{x, y}\).
			\item How can we compute barycentric coefficients of \(P\) for arbitrary points \(A\), \(B\), and \(C\)?
			Is this always possible?
			Argu both algebraically and geometrically.
		\end{enumerate}
	\end{quote}

	\begin{enumerate}
		\item Drawing gives:
		\begin{center}
		    \begin{tikzpicture}
		        \filldraw[fill=white!80!black,draw=black] (0,0) rectangle (8,4.5) node[pos=.5] {Platzhalter};
		    \end{tikzpicture}
		\end{center}
		\item In extended coordinates, we obtain the linear system of equations
		\[
			\begin{pmatrix}
				x\\
				y\\
				1
			\end{pmatrix} = \alpha\begin{pmatrix}
				0\\
				0\\
				1
			\end{pmatrix} + \beta\begin{pmatrix}
				6\\
				0\\
				1
			\end{pmatrix} + \gamma\begin{pmatrix}
				0\\
				4\\
				1
			\end{pmatrix},
		\]
		which gives \(\alpha = 1 - \beta - \gamma\), \(\beta 0 \frac{x}{6}\), and \(\gamma = \frac{y}{4}\).
		\item In general, we can choose \(A\) to be the coordinate origin.
		Then, we need to write \(\vec{AP}\) as a linear combination \(\beta\vec{AB} + \gamma\vec{AC}\).
		This is always possible if and only if \(\vec{AB}\) and \(\vec{AC}\) are linear independent, i. e. exactly if \(\parentheses*{A, B, C}\) forms a non-degenerate triangle.
	\end{enumerate}


	\section{}

	\begin{quote}
		The color from a light source can be represented by a map \(\lambda: \R^+ \to \R^+\), assigning each wavelength an intensity.
		\begin{enumerate}
			\item Are there colours that cannot be described by a single wavelength?
			\item How is the map \(\lambda\) modified when the light reflects from an object?
			\item The eye contains three cones \(s, m, l\) responding to wavelengths.
			Each can be described by a map \(\R^+ \to \R^+\).
			What is a plausible way to obtain the aggregated activation of each cone?
			(Hint: The result should lie in \(\R\).)
			\item Which considerations for choosing colors are supported by https://colorbrewer2.org?
			Are there other important considerations?
		\end{enumerate}
	\end{quote}

	\begin{enumerate}
		\item Yes, each combination of wavelengths is perceived as a colour again.
		Examples are grayscales and shades of magenta (mix of red and violet).
		\item The object has a reflectance function (dependent on light and viewer directions) \(\rho: \R^+ \to \brackets*{0, 1}\), assigning each wavelength the proportion of reflected light.
		The reflected light is represented by
		\[
			\lambda\rho: \R^+ \to \R^+, w \mapsto \lambda\parentheses*{w}\rho\parentheses*{w}.
		\]
		\item We need to combine the individual wavelength responses.
		This is done via integration:
		\[
			\int_{\R^+}\lambda\parentheses*{w}s\parentheses*{w}\d w.
		\]
		With the three cones, the signal \(\lambda\) is transformed into a tuple of three real numbers.
		\item The website offers the following choices:
		\begin{itemize}
			\item How many colors?
			\item What type of data should be represented? (homogeneous scale, emphasising midrange and extreme values, purely qualitative)
			\item Only variations of a single color?
			\item Safe for red-green color blindness?
			\item Can it be printed?
			\item Can it be copied?
		\end{itemize}
		Further considerations could be:
		\begin{itemize}
			\item Which feeling do they provoke?
			\item Are they distracting?
			\item Is there a risk of optical illusions?
			\item Should colors be used at all?
		\end{itemize}
	\end{enumerate}
\end{document}