\documentclass[english]{exercise}

\DeclareMathOperator*{\cond}{cond}
\DeclareMathOperator*{\diag}{diag}
\DeclareMathOperator*{\dist}{dist}
\DeclareMathOperator*{\esssup}{ess\,sup}
\DeclareMathOperator*{\vol}{vol}


\title{Exercise 1}
\author{Joshua Feld (406718)}
\professor{Prof. Kobbelt}
\course{Data Analysis and Visualization}

\begin{document}
	\maketitle


	\section{}

	\begin{quote}
		Write the point \(P\) as a sum of the origin and the scaled basis vectors in the coordinate systems \(\parentheses*{A, \vec{a}_1, \vec{a}_2}\), \(\parentheses*{B, \vec{b}_1, \vec{b}_2}\), and \(\parentheses*{C, \vec{c}_1, \vec{c}_2}\).
		\begin{center}
			\begin{tikzpicture}
				\draw[step=.5,white!50!black] (0,0) grid (4.5,2.5);
				\filldraw (.5,.5) circle (.0625) node[below left] {\(C\)};
				\draw[->] (.5,.5) -- (1,0) node[above right] {\(\vec{c}_1\)};
				\draw[->] (.5,.5) -- (1,1) node[below right] {\(\vec{c}_2\)};
				\filldraw (2,1) circle (.0625) node[below right] {\(P\)};
				\filldraw (1,2) circle (.0625) node[above right] {\(A\)};
				\draw[->] (1,2) -- (2.5,2) node[above right] {\(\vec{a}_1\)};
				\draw[->] (1,2) -- (1,1.5) node[above left] {\(\vec{a}_2\)};
				\filldraw (3.5,1.5) circle (.0625) node[above right] {\(B\)};
				\draw[->] (3.5,1.5) -- (2.5,.5) node[below left] {\(\vec{b}_1\)};
				\draw[->] (3.5,1.5) -- (4,1.5) node[above right] {\(\vec{b}_2\)};
			\end{tikzpicture}
		\end{center}
	\end{quote}

	\[
		P = A + \frac{2}{3}\vec{a}_1 + 2\vec{a}_2 = B + \frac{1}{2}\vec{b}_1 - 2\vec{b}_2 = c + \vec{c}_1 + 2\vec{c}_2.
	\]


	\section{}

	\begin{quote}
		Let \(\parentheses*{\vec{b}_1, \vec{b}_2}\) be the basis of a 2-dimensional vector space.
		Define \(\vec{c}_1 := \vec{b}_1 + \vec{b}_2\) and \(\vec{c}_2 := 2\vec{b}_1 + 3\vec{b}_2\).
		We want to write \(u\vec{b}_1 + v\vec{b}_2\) as a linear combination of \(\vec{c}_1\) and \(\vec{c}_2\).
		\begin{enumerate}
			\item Solve the problem by first representing \(\vec{b}_1\) and \(\vec{b}_2\) as a linear combination of \(\vec{c}_1\) and \(\vec{c}_2\).
			\item Solve the problem by using columns and matrices.
			\item How does the algebraic structure of b) relate to existence and uniqueness of a solution?
			\item What are the benefits and drawbacks of the methods used in a) and b)?
		\end{enumerate}
	\end{quote}

	\begin{enumerate}
		\item The two given equations for \(\vec{c}_1\) and \(\vec{c}_2\) form a linear system of equations which we can solve by applying the Gauss algorithm:
		\[
			\parentheses*{\begin{array}{cc|c}
				1 & 1 & \vec{c}_1\\
				2 & 3 & \vec{c}_2
			\end{array}} \to \parentheses*{\begin{array}{cc|c}
				1 & 1 & \vec{c}_1\\
				0 & 1 & \vec{c}_2 - 2\vec{c}_1
			\end{array}} \to \parentheses*{\begin{array}{cc|c}
				1 & 0 & 3\vec{c}_1 - \vec{c}_2\\
				0 & 1 & \vec{c}_2 - 2\vec{c}_1
			\end{array}}
		\]
		Thus we have \(\vec{b}_1 = 3\vec{c}_1 - \vec{c}_2\) and \(\vec{b}_2 = \vec{c}_2 - 2\vec{c}_1\) and can now plug this into the equation that we want to rewrite
		\[
			u\vec{b}_1 + v\vec{b}_2 = u\parentheses*{3\vec{c}_1 - \vec{c}_2} + v\parentheses*{\vec{c}_2 - 2\vec{c}_1} = \parentheses*{3u - 2v}\vec{c}_1 + \parentheses*{v - u}\vec{c}_2.
		\]
		\item To properly use columns, we need to clearly state the used bases
		\[
			\begin{pmatrix}
				\alpha\\
				\beta
			\end{pmatrix}_B := \alpha\vec{b}_1 + \beta\vec{b}_2, \quad \begin{pmatrix}
				\alpha\\
				\beta
			\end{pmatrix}_C := \alpha\vec{c}_1 + \beta\vec{c}_2.
		\]
		Next, we need to relate the different columns by a matrix \(T_{C \to B}\).
		The given equations for \(\vec{c}_1\) and \(\vec{c}_2\) thus translate to
		\[
			T_{C \to B}\begin{pmatrix}
				1\\
				0
			\end{pmatrix}_C = \begin{pmatrix}
				1\\
				1
			\end{pmatrix}_B, \quad T_{C \to B}\begin{pmatrix}
				0\\
				1
			\end{pmatrix}_C = \begin{pmatrix}
				2\\
				3
			\end{pmatrix}_B.
		\]
		Combining them gives the matrix
		\[
			T_{C \to B} = \begin{pmatrix}
				1 & 2\\
				1 & 3
			\end{pmatrix}.
		\]
		The given task is writing \(\parentheses*{u, v}_B^T\) as a column with respect to \(C\).
		Thus, we need to compute the inverse matrix \(T_{C \to B}^{-1}\), which we can directly calculate by
		\[
			\begin{pmatrix}
				1 & 2\\
				1 & 3
			\end{pmatrix}^{-1} = \frac{1}{1 \cdot 3 - 2 \cdot 1}\begin{pmatrix}
				3 & -2\\
				-1 & 1
			\end{pmatrix} = \begin{pmatrix}
				3 & -2\\
				-1 & 1
			\end{pmatrix}.
		\]
		This gives us
		\[
			T_{B \to C}\begin{pmatrix}
				u\\
				v
			\end{pmatrix}_C = \begin{pmatrix}
				3u - 2v\\
				-u + v
			\end{pmatrix}_C.
		\]
		\item There exists a unique solution for all \(u, v \in \R\) if and only if the matrix \(T_{C \to B}\) is invertible.
		If it is not, there are linear combinations in \(\parentheses*{\vec{b}_1, \vec{b}_2}\) that cannot be represented by linear combinations in \(\parentheses*{\vec{c}_1, \vec{c}_2}\).
		\item Both methods have the same generality.
		The method in b) makes it easier to find the correct algebraic calculations.
		In contrast, it is easy to forget the step to geometry in b).
		A column alone is not an answer, since the coordinate system has to be clearly specified.
	\end{enumerate}


	\section{}

	\begin{quote}
		Let \(\parentheses*{E, \vec{e}_1, \vec{e}_2}\) be a 2-dimensional coordinate system.
		Construct the following transformation matrices:
		\begin{enumerate}
			\item Uniform scaling by a factor of \(3\) followed by a \(90^\circ\) counterclockwise rotation.
			\item A shearing transformation that maps \(\vec{e}_1\) onto itself and \(\vec{e}_2\) onto \(2\vec{e}_2 + \vec{e}_1\).
			\item Non-uniform scaling around the origin by a factor of \(3\) in direction of \(\parentheses*{1, 1, 0}^T\) and by a factor of \(\frac{1}{2}\) in direction of \(\parentheses*{1, -1, 0}^T\).
		\end{enumerate}
	\end{quote}

	\begin{enumerate}
		\item A general uniform scaling matrix in a two-dimensional coordinate system is given by
		\[
			\begin{pmatrix}
				k & 0\\
				0 & k
			\end{pmatrix},
		\]
		where \(k\) is the scaling factor.
		Thus for a scaling by a factor of \(3\) we simply set \(k = 3\).
		We also want to add a \(90^\circ\) counterclockwise rotation. A general rotation matrix is given by
		\[
			\begin{pmatrix}
				\cos\phi & -\sin\phi\\
				\sin\phi & \cos\phi
			\end{pmatrix},
		\]
		which rotates conterclockwise by \(\phi\). Thus for \(\phi = 90^\circ = \frac{\pi}{2}\) we obtain the matrix
		\[
			\begin{pmatrix}
				0 & -1\\
				1 & 0
			\end{pmatrix}.
		\]
		Since we first want to scale and then rotate, the final transformation matrix is constructed as follows:
		\[
			T := \begin{pmatrix}
				0 & -1\\
				1 & 0
			\end{pmatrix}\begin{pmatrix}
				3 & 0\\
				0 & 3
			\end{pmatrix} = \begin{pmatrix}
				0 & -3\\
				3 & 0
			\end{pmatrix}.
		\]
		\item The shearing transformation matrix is simply given by
		\[
			T := \begin{pmatrix}
				1 & 0\\
				1 & 2
			\end{pmatrix}.
		\]
		\item To achieve non-uniform scaling in the specified directions, we first change the coordinate axes, then apply the scaling, and finally change back the axes.
		The coordinate transformation from the local coordinate system to the world coordinate system is defined as
		\[
			L := \begin{pmatrix}
				1 & 1\\
				1 & -1
			\end{pmatrix}.
		\]
		The inverse transformation \(L^{-1}\) thus transforms from the world to the local coordinate system.
		The combined transformation can be computed as follows (recall that the right-most matrix acts first):
		\[
			T := L\begin{pmatrix}
				3 & 0\\
				0 & \frac{1}{2}
			\end{pmatrix}L^{-1}.
		\]
	\end{enumerate}


	\section{}

	\begin{quote}
		In the lecture you learned that a scaling transformation does not flip the orientation if \(\alpha\beta\gamma > 0\).
		More generally, this can be determined by considering how the linear map changes the length of vectors, i.e. whether for the determinant of the transformation matrix holds \(\det\parentheses*{T} > 0\).
		Characterize under which conditions the following transformations are orientation-preserving.
		\begin{enumerate}
			\item An arbitrary 2D rotation,
			\item \(\begin{pmatrix}
				1 & a\\
				0 & 0
			\end{pmatrix}\),
			\item \(\begin{pmatrix}
				-t & a\\
				b & t
			\end{pmatrix}\) for \(t, a, b \in \R\).
		\end{enumerate}
	\end{quote}

	\begin{enumerate}
		\item The matrix
		\[
			\begin{pmatrix}
				\cos\phi & -\sin\phi\\
				\sin\phi & \cos\phi
			\end{pmatrix}
		\]
		rotates points in the \(xy\)-plane counterclockwise through an angle \(\phi\) with respect to the positive \(x\)-axis about the origin of a two-dimensional Cartesian coordinate system.
		The determinant of this matrix is given by
		\[
			\begin{vmatrix}
				\cos\phi & -\sin\phi\\
				\sin\phi & \cos\phi
			\end{vmatrix} = \cos^2 \phi + \sin^2 \phi = 1 > 0
		\]
		and thus an arbitrary 2D rotation is always orientation-preserving.
		\item Applying this transformation matrix always flips the orientation since
		\[
			\begin{vmatrix}
				1 & a\\
				0 & 0
			\end{vmatrix} = 0 \not> 0.
		\]
		\item
		\[
			\begin{vmatrix}
				-t & a\\
				b & t
			\end{vmatrix} = -t^2 - ab \stackrel{!}{>} 0 \iff t^2 + ab < 0 \iff ab < -t^2.
		\]
		From this we can conclude that \(a\) and \(b\) must have different signs (\(a > 0, b < 0\) or \(a < 0, b > 0\)) and \(t \in \parentheses*{-\sqrt{-ab}, \sqrt{-ab}}\) for the transformation matrix to be orientation-preserving.
	\end{enumerate}
\end{document}
