\documentclass[english]{exercise}

\DeclareMathOperator{\HI}{HI}
\DeclareMathOperator{\IoU}{IoU}


\title{Exercise 3}
\author{Joshua Feld (406718)}
\professor{Prof. Kobbelt}
\course{Data Analysis and Visualization}

\begin{document}
	\maketitle


	\section{}

	\begin{quote}
		Given are the following (not necessarily optimal) transport plans.
		The horizontal histogram is called \(X\) and the vertical one \(Y\).
		Which constraints for the EMD are violated?
	\end{quote}

	\begin{enumerate}
		\item The capacity of \(Y\) is exceeded.
		\item No constraints are violated.
		\item The entries of the transport plan have to be \(\ge 0\). The capacity of \(Y\) is also exceeded.
	\end{enumerate}


	\section{}

	\begin{quote}
		Give an example each for the following dimensionalities of the domains and ranges of samples:
		\begin{center}
			\begin{tabular}{r|ccccccc}
				\toprule
				Domain \(d\) & \(1\) & \(2\) & \(2\) & \(2\) & \(3\) & \(3\) & \(4\)\\
				\midrule
				Range \(k\) & \(1\) & \(1\) & \(2\) & \(3\) & \(3\) & \(6\) & \(3\)\\
				\bottomrule
			\end{tabular}
		\end{center}
	\end{quote}

	\begin{itemize}
		\item \(d = 1, k = 1\): Sound pressure over time at a fixed location (e.g. picked up by a microphone).
		\item \(d = 2, k = 1\): Atmospheric pressure on a 2D weather forecast map.
		\item \(d = 2, k = 2\): Airflow velocity in a weather forecast.
		\item \(d = 2, k = 3\): RGB image.
		\item \(d = 3, k = 3\): Orientation of an object at a location in 3D space.
		\item \(d = 3, k = 6\): Electromagnetic fields consisting of 3D electric and magnetic field strengths.
		\item \(d = 4, k = 3\): Time-varying 3D flow field.
	\end{itemize}


	\section{}

	\begin{quote}
		Compute a 1D embedding of the following dataset using PCA.
	\end{quote}

	Due tu the symmetry of the dataset, we can see that the ais of largest variance is facing along the \(x\)-direction, while the axis of smallest variance points orthogonally to it along the \(y\)-direction.
	To compute a 1D embedding of the data, each point can be projected on the axis of largest variance, by projecting it down onto the vector above facing in \(x\)-direction.
	This corresponds to simply dropping the \(y\)-coordinate of each point.


	\section{}

	\begin{quote}
		Which of the following datasets can be properly captured by a PCA analysis?
		Explain why this is the case.
	\end{quote}

	Only the first set of samples can be adequately approximated, because the data is roughly distributed like a multi-variate Gaussian.
	A linear subspace projection like PCA is not able to properly capture
	\begin{itemize}
		\item containment relations between subsets of samples in the embedding space (top-right),
		\item datasets that live on a non-linear lower-dimensional manifold (bottom-left),
		\item clusters of datasets that are generated from different distributions (bottom-right).
	\end{itemize}
\end{document}
