\documentclass[english]{exercise}

\DeclareMathOperator*{\cond}{cond}
\DeclareMathOperator*{\diag}{diag}
\DeclareMathOperator*{\dist}{dist}
\DeclareMathOperator*{\esssup}{ess\,sup}
\DeclareMathOperator*{\vol}{vol}


\title{Exercise 2}
\author{Joshua Feld (406718)}
\course{Data Analysis and Visualization}
\professor{Prof. Kobbelt}

\begin{document}
    \maketitle
    

    \section{}
    
    \begin{quote}
        Let \(P\) and \(Q\) be affine points.
        Use column representations to decide whether \(\alpha P + \beta Q\) (for \(\alpha, bet \in \R\)) is a point or a vector.
        If so what is its geometric interpretation?
    \end{quote}

    Only the extended column representation allows this judgement.
    There, the last component is \(0\) for a vector and \(1\) for a point.
    Thus, the last component of \(\alpha P + \beta Q\) is \(\alpha + \beta\).
    We distinguish three cases:
    \begin{itemize}
        \item \(\alpha + \beta \not\in \braces*{0, 1}\): This combination is not geometrically meaningful.
        \item \(\alpha + \beta = 0\), i.e. \(\beta = -\alpha\): This gives a vector.
        For the geometric interpretation
        \[
            \alpha P + \beta Q = \alpha P - \alpha Q = \alpha\parentheses*{P - Q}.
        \]
        The difference \(P - Q\) represents the vector from \(Q\) to \(P\).
        Thus, \(\alpha P + \beta Q\) represents this vector, scaled by \(\alpha\).
        \item \(\alpha + \beta = 1\), i.e. \(\beta = 1 - \alpha\): This gives a point.
        For the geometric interpretation
        \[
            \alpha P + \beta Q = Q + \alpha\parentheses*{P - Q}.
        \]
        We interpret the vector \(\alpha\parentheses*{P - Q}\) as before.
        Thus, the full object is the point we obtain by starting at \(Q\) and moving by a fraction of \(\alpha\) in the direction of \(P\).
    \end{itemize}
    
    
    \section{}
    
    \begin{quote}
        Let \(\parentheses*{E, \vec{e}_1, \vec{e}_2}\) be an affine coordinate system in two dimensions.
        Let \(\parentheses*{Q, \vec{q}_1, \vec{q}_2}\) be another affine coordinate system, with extended coordinates (with respect to \(\parentheses*{E, \vec{e}_1, \vec{e}_2}\))
        \[
            Q = \begin{pmatrix}
                5\\
                -3\\
                1
            \end{pmatrix}, \quad \vec{q}_1 = \begin{pmatrix}
                2\\
                -1\\
                0
            \end{pmatrix}, \quad \vec{q}_2 = \begin{pmatrix}
                -\frac{1}{3}\\
                \frac{1}{2}\\
                0
            \end{pmatrix}.
        \]
        Construct the coordinate transformation to convert \(E + x\vec{e}_1 + y\vec{e}_2\) into the system \(\parentheses*{Q, \vec{q}_1, \vec{q}_2}\).
    \end{quote}

    We proceed in two steps.
    First, we construct the transformation from \(\parentheses*{E, \vec{e}_1, \vec{e}_2}\) to \(\parentheses*{E, \vec{q}_1, \vec{q}_2}\).
    Afterwards, we construct the transformation to \(\parentheses*{Q, \vec{q}_1, \vec{q}_2}\).
    The first transformation is a change of basis from \(\parentheses*{\vec{e}_1, \vec{e}_2}\) to \(\parentheses*{\vec{q}_1, \vec{q}_2}\).
    The transformation from \(\parentheses*{\vec{q}_1, \vec{q}_2}\) to \(\parentheses*{\vec{e}_1, \vec{e}_2}\) is given by
    \[
        \begin{pmatrix}
            2 & -\frac{1}{3}\\
            -1 & \frac{1}{2}
        \end{pmatrix}.
    \]
    The inverse transformation from \(\parentheses*{\vec{e}_1, \vec{e}_2}\) to \(\parentheses*{\vec{q}_1, \vec{q}_2}\) is
    \[
        \begin{pmatrix}
            2 & -\frac{1}{3}\\
            -1 & \frac{1}{2}
        \end{pmatrix}^{-1} = \begin{pmatrix}
            \frac{3}{4} & \frac{1}{2}\\
            \frac{3}{2} & 3
        \end{pmatrix}.
    \]
    Written in extended coordinates, the coordinate transformation from \(\parentheses*{E, \vec{e}_1, \vec{e}_2}\) to \(\parentheses*{E, \vec{q}_1, \vec{q}_2}\) can thus be represented by the matrix
    \[
        \begin{pmatrix}
            \frac{3}{4} & \frac{1}{2} & 0\\
            \frac{3}{2} & 3 & 0\\
            0 & 0 & 1
        \end{pmatrix}
    \]
    To shift from \(\parentheses*{E, \vec{q}_1, \vec{q}_2}\) to \(\parentheses*{Q, \vec{q}_1, \vec{q}_2}\), we need to translate by \(-\vec{EQ}\).
    Since the extended coordinates of \(Q\) are written with respect to \(\parentheses*{E, \vec{e}_1, \vec{e}_2}\), a conversion with \(T\) is necessary.
    The final transformation is thus
    \[
        \begin{pmatrix}
            x\\
            y\\
            1
        \end{pmatrix} \mapsto T\begin{pmatrix}
            x\\
            y\\
            1
        \end{pmatrix} - T\begin{pmatrix}
            5\\
            -3\\
            0
        \end{pmatrix} = \begin{pmatrix}
            \frac{3}{4} & \frac{1}{2} & -\frac{9}{4}\\
            \frac{3}{2} & 3 & \frac{3}{2}\\
            0 & 0 & 1
        \end{pmatrix}\begin{pmatrix}
            x\\
            y\\
            1
        \end{pmatrix}.
    \]

    
    \section{}
    
    \begin{quote}
        Let the feature vectors \(f_1, f_2 \in \R^5\) be given as
        \[
            f_1 = \begin{pmatrix}
                3.3\\
                2.4\\
                1.9\\
                5.0\\
                3.8
            \end{pmatrix}, \quad f_2 = \begin{pmatrix}
                2.8\\
                2.4\\
                1.1\\
                6.2\\
                4.3
            \end{pmatrix}.
        \]
        \begin{enumerate}
            \item Compute a dissimilarity measure \(\delta\parentheses*{f_1, f_2}\) using the Minkowski distance for \(p = 1, 2, \infty\).
            \item For each result, turn \(\delta\parentheses*{f_1, f_2}\) into a similarity measure using the
            \begin{enumerate}
                \item logarithmic,
                \item exponential
            \end{enumerate}
            similarity functions.
            \item Compute the direct similarity measures
            \begin{enumerate}
                \item dot-product measure \(f_1^T f_2\),
                \item cosine measure \(\frac{f_1^T f_2}{\norm*{f_1}\norm*{f_2}}\).
            \end{enumerate}
        \end{enumerate}
        Use your favorite computing environment (e.g. Python) to compute the results.
    \end{quote}
    
    \begin{enumerate}
        \item The minkowski distance of order \(p\) (where \(p\) is an integer) between two points \(X, Y \in \R^n\) is defined as \(L_p\parentheses*{X, Y} = \parentheses*{\sum_{i = 1}^n \absolute*{x_i - y_i}^p}^{\frac{1}{p}}\).
        For \(p = 1\) we get the Manhattan distance
        \[
            L_1\parentheses*{f_1, f_2} = \sum_{i = 1}^5 \absolute*{f_{1, i} - f_{2, i}} = 3,
        \]
        for \(p = 2\) the Euclidean distance
        \[
            L_2\parentheses*{f_1, f_2} = \sqrt{\sum_{i = 1}^5 \absolute*{f_{1, i} - f_{2, i}}^2} = 1.6062,
        \]
        and for \(p = \infty\) the Chebyshev distance
        \[
            L_\infty\parentheses*{f_1, f_2} = \max_{i = 1, \ldots, 5} \absolute*{f_{1, i} - f_{2, i}} = 1.2.
        \]
        \item We can transform any distance function \(\delta\) to a similarity function \(s\) with any monotonically decreasing function \(g: \R \to \R\)
        \[
            s\parentheses*{X, Y} = g\parentheses*{\delta\parentheses*{X, Y}} \quad \forall X, Y \in \R^n.
        \]
        \begin{enumerate}
            \item The logarithmic similarity function is given by
            \[
                s_{\log}\parentheses*{X, Y} = 1 - \log\parentheses*{1 + \delta\parentheses*{X, Y}}.
            \]
            Plugging in \(L_1, L_2, L_\infty\) for \(\delta\) gives us the values
            \begin{align*}
                s_{\log, 1}\parentheses*{f_1, f_2} &= 1 - \log\parentheses*{1 + L_1\parentheses*{f_1, f_2}} = -0.38629,\\
                s_{\log, 2}\parentheses*{f_1, f_2} &= 1 - \log\parentheses*{1 + L_2\parentheses*{f_1, f_2}} = 0.042107,\\
                s_{\log, \infty}\parentheses*{f_1, f_2} &= 1 - \log\parentheses*{1 + L_\infty\parentheses*{f_1, f_2}} = 0.21154.
            \end{align*}
            \item Analoguously, the exponential similarity function is given by
            \[
                s_{\exp}\parentheses*{X, Y} = e^{-\delta\parentheses*{X, Y}}.
            \]
            Thus plugging in \(L_1, L_2, L_\infty\) for \(\delta\) gives us the values
            \begin{align*}
                s_{\exp, 1}\parentheses*{f_1, f_2} &= e^{-L_1\parentheses*{f_1, f_2}} = 0.049787,\\
                s_{\exp, 2}\parentheses*{f_1, f_2} &= e^{-L_2\parentheses*{f_1, f_2}} = 0.20065,\\
                s_{\exp, \infty}\parentheses*{f_1, f_2} &= e^{-L_\infty\parentheses*{f_1, f_2}} = 0.30119.
            \end{align*}
        \end{enumerate}
        \item
        \begin{enumerate}
            \item The dot product measure for the given points is \(s_d\parentheses*{f_1, f_2} = f_1^T f_2 = 64.43\).
            \item The cosine measure for \(f_1\) and \(f_2\) is \(s_c\parentheses*{f_1, f_2} = \frac{f_1^T f_2}{\norm*{f_1}\norm*{f_2}} = 0.98451\).
        \end{enumerate}
    \end{enumerate}
    
    
    \section{}
    
    \begin{quote}
        \begin{figure}[h]
            \centering
            \begin{tikzpicture}
                \draw[<->] (0,6.5) -- (0,0) -- (7,0);
                \draw (0.625,0) rectangle (1.375,2);
                \draw (1.625,0) rectangle (2.375,4);
                \draw (2.625,0) rectangle (3.375,3);
                \draw (4.625,0) rectangle (5.375,6);
                \draw (5.625,0) rectangle (6.375,2);
                \foreach \i in {0,...,6}
                    \draw (.125,\i) -- (-.125,\i) node[left] {\(\i\)};
                \foreach \i in {0.2,0.4,0.6,0.8,1,1.2}
                    \draw (5*\i,.125) -- (5*\i,-.125) node[below] {\(\i\)};
            \end{tikzpicture}
            \quad
            \begin{tikzpicture}
                \draw[<->] (0,6.5) -- (0,0) -- (7,0);
                \draw (0.625,0) rectangle (1.375,3);
                \draw (1.625,0) rectangle (2.375,6);
                \draw (2.625,0) rectangle (3.375,4);
                \draw (3.625,0) rectangle (4.375,1);
                \draw (4.625,0) rectangle (5.375,5);
                \draw (5.625,0) rectangle (6.375,2);
                \foreach \i in {0,...,6}
                    \draw (.125,\i) -- (-.125,\i) node[left] {\(\i\)};
                \foreach \i in {0.2,0.4,0.6,0.8,1,1.2}
                    \draw (5*\i,.125) -- (5*\i,-.125) node[below] {\(\i\)};
            \end{tikzpicture}
            \caption{Histograms \(h_1\) (left) and \(h_2\) (right)}
            \label{fig:4-1}
        \end{figure}
        For the feature histograms \(h_1, h_2\) compute the dissimilarity measures
        \begin{enumerate}
            \item Hamming distance,
            \item Minkowski distance for \(p = 1, 2, \infty\),
            \item histogram intersection distance \(\HI\parentheses*{h_1, h_2}\) and \(\HI\parentheses*{h_2, h_1}\),
            \item histogram intersection over union distance.
        \end{enumerate}
        Use your favorite computing environment (e.g. Python) to compute the results.
    \end{quote}

    \begin{enumerate}
        \item The Hamming distance between two equal-size histograms is the number of positions at which the corresponding values are different:
        \[
            \Delta\parentheses*{h_1, h_2} = \absolute*{\braces*{i \in \braces*{1, \ldots, 6} : h_{1, i} \ne h_{2, i}}} = 5.
        \]
        \item Similar to the previous exercise, we obtain the following values for the Minkowski distances:
        \begin{align*}
            L_1\parentheses*{h_1, h_2} &= \sum_{i = 1}^6 \absolute*{h_{1, i} - h_{2, i}} = 6,\\
            L_2\parentheses*{h_1, h_2} &= \sqrt{\sum_{i = 1}^6 \absolute*{h_{1, i} - h_{2, i}}^2} = 2.8284,\\
            L_\infty\parentheses*{h_1, h_2} &= \max_{i = 1, \ldots, 6} \absolute*{h_{1, i} - h_{2, i}} = 2.
        \end{align*}
        \item The histogram intersections are given by
        \begin{align*}
            \HI\parentheses*{h_1, h_2} &= \frac{\sum_{i = 1}^6 \min\parentheses*{h_{1, i}, h_{2, i}}}{\sum_{i = 1}^6 h_{2, i}} = 0.7619,\\
            \HI\parentheses*{h_2, h_1} &= \frac{\sum_{i = 1}^6 \min\parentheses*{h_{1, i}, h_{2, i}}}{\sum_{i = 1}^6 h_{1, i}} = 0.94118.
        \end{align*}
        \item The histogram intersection over union on the other hand can be calculated as follows:
        \[
            \IoU\parentheses*{h_1, h_2} = \frac{\sum_{i = 1}^6 \min\parentheses*{h_{1, i}, h_{2, i}}}{\sum_{i = 1}^6 \max\parentheses*{h_{1, i}, h_{2, i}}} = 0.72727.
        \]
    \end{enumerate}
\end{document}
