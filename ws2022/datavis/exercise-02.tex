\documentclass[english]{exercise}

\title{Exercise 2}
\author{Joshua Feld (406718)}
\course{Data Analysis and Visualization}
\professor{Prof. Kobbelt}

\begin{document}
    \maketitle
    

    \section{}
    
    \begin{quote}
        Let \(P\) and \(Q\) be affine points.
        Use column representations to decide whether \(\alpha P + \beta Q\) (for \(\alpha, bet \in \R\)) is a point or a vector.
        If so what is its geometric interpretation?
    \end{quote}
    
    
    \section{}
    
    \begin{quote}
        Let \(\parentheses*{E, \vec{e}_1, \vec{e}_2}\) be an affine coordinate system in two dimensions.
        Let \(\parentheses*{Q, \vec{q}_1, \vec{q}_2}\) be another affine coordinate system, with extended coordinates (with respect to \(\parentheses*{E, \vec{e}_1, \vec{e}_2}\))
        \[
            Q = \begin{pmatrix}
                5\\
                -3\\
                1
            \end{pmatrix}, \quad \vec{q}_1 = \begin{pmatrix}
                2\\
                -1\\
                0
            \end{pmatrix}, \quad \vec{q}_2 = \begin{pmatrix}
                -\frac{1}{3}\\
                \frac{1}{2}\\
                0
            \end{pmatrix}.
        \]
        Construct the coordinate transformation to convert \(E + x\vec{e}_1 + y\vec{e}_2\) into the system \(\parentheses*{Q, \vec{q}_1, \vec{q}_2}\).
    \end{quote}
    
    \section{}
    
    \begin{quote}
        Let the feature vectors \(f_1, f_2 \in \R^5\) be given as
        \[
            f_1 = \begin{pmatrix}
                3.3\\
                2.4\\
                1.9\\
                5.0\\
                3.8
            \end{pmatrix}, \quad f_2 = \begin{pmatrix}
                2.8\\
                2.4\\
                1.1\\
                6.2\\
                4.3
            \end{pmatrix}.
        \]
        \begin{enumerate}
            \item Compute a dissimilarity measure \(\delta\parentheses*{f_1, f_2}\) using the Minkowski distance for \(p = 1, 2, \infty\).
            \item For each result, turn \(\delta\parentheses*{f_1, f_2}\) into a similarity measure using the
            \begin{enumerate}
                \item logarithmic,
                \item exponential
            \end{enumerate}
            similarity functions.
            \item Compute the direct similarity measures
            \begin{enumerate}
                \item dot-product measure \(f_1^T f_2\),
                \item cosine measure \(\frac{f_1^T f_2}{\norm*{f_1}\norm*{f_2}}\).
            \end{enumerate}
        \end{enumerate}
        Use your favorite computing environment (e.g. Python) to compute the results.
    \end{quote}
    
    \begin{enumerate}
        \item
        \item
        \begin{enumerate}
            \item
            \item
        \end{enumerate}
        \item
        \begin{enumerate}
            \item
            \item
        \end{enumerate}
    \end{enumerate}
    
    
    \section{}
    
    \begin{quote}
        \begin{figure}[h]
            \centering
            \begin{tikzpicture}
                \draw[<->] (0,6.5) -- (0,0) -- (7,0);
                \draw (0.625,0) rectangle (1.375,2);
                \draw (1.625,0) rectangle (2.375,4);
                \draw (2.625,0) rectangle (3.375,3);
                \draw (4.625,0) rectangle (5.375,6);
                \draw (5.625,0) rectangle (6.375,2);
                \foreach \i in {0,...,6}
                    \draw (.125,\i) -- (-.125,\i) node[left] {\(\i\)};
                \foreach \i in {0.2,0.4,0.6,0.8,1,1.2}
                    \draw (5*\i,.125) -- (5*\i,-.125) node[below] {\(\i\)};
            \end{tikzpicture}
            \quad
            \begin{tikzpicture}
                \draw[<->] (0,6.5) -- (0,0) -- (7,0);
                \draw (0.625,0) rectangle (1.375,3);
                \draw (1.625,0) rectangle (2.375,6);
                \draw (2.625,0) rectangle (3.375,4);
                \draw (3.625,0) rectangle (4.375,1);
                \draw (4.625,0) rectangle (5.375,5);
                \draw (5.625,0) rectangle (6.375,2);
                \foreach \i in {0,...,6}
                    \draw (.125,\i) -- (-.125,\i) node[left] {\(\i\)};
                \foreach \i in {0.2,0.4,0.6,0.8,1,1.2}
                    \draw (5*\i,.125) -- (5*\i,-.125) node[below] {\(\i\)};
            \end{tikzpicture}
            \caption{Histograms \(h_1\) (left) and \(h_2\) (right)}
            \label{fig:4-1}
        \end{figure}
        For the feature histograms \(h_1, h_2\) compute the dissimilarity measures
        \begin{enumerate}
            \item Hamming distance,
            \item Minkowski distance for \(p = 1, 2, \infty\),
            \item histogram intersection distance \(HI\parentheses*{h_1, h_2}\) and \(HI\parentheses*{h_2, h_1}\),
            \item histogram intersection over union distance.
        \end{enumerate}
        Use your favorite computing environment (e.g. Python) to compute the results.
    \end{quote}

    \begin{enumerate}
        \item
        \item
        \item
        \item
    \end{enumerate}
\end{document}
