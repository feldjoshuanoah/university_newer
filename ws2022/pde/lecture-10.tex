\section{The Riemann problem}

The Riemann problem will become a building block of finite-volume methods for conservation laws.
To introduce it, we will first need some more details on systems of conservation laws.
We will begin by focusing on systems on the real line.


\subsection{Hyperbolic systems}

In this section we will consider systems of conservation laws on the real line, for which the unknown is a vector-valued function
\[
	U: \R \times \R^+ \to \R^N
\]
that solves the initial value problem (i.e., Cauchy problem)
\begin{align*}
	\partial_t U + \partial_x F\parentheses*{U} &= 0, \quad \text{in }\R \times \parentheses*{0, \infty},\\
	U\parentheses*{\cdot, 0} &= U^{\parentheses*{0}}, \quad \text{in }\R,
\end{align*}
for a vector-valued flux function \(F: \R^N \to \R^N\), where \(\partial_t U\) is understood componentwise.
We assume the system is (strictly) hyperbolic, recall that
\[
	\text{system is (stricly) hyperbolic} \iff \text{Jacobian }D_U F\parentheses*{U}\text{is diagonalizable with real (distinct) eigenvalues.}
\]
The concepts of weak solutions, Rankine-Hugoniot conditions, and entropy can be extended to the case of systems of conservation laws in a natural way.

\begin{example}[Hyperbolic system]
	The simplest case is a linear system giben by the flux function \(F\parentheses*{U} = AU\) with a constant matrix \(A \in \R^{N \times N}\).
	Then \(D_U F\parentheses*{U} = A\) and the system reads
	\[
		\partial_t U + A\partial_x U = 0.
	\]
	Hyperbolicity then implies that the matrix \(A\) is diagonalizable with a regular transformation matrix \(T \in \R^{N \times N}\), such that
	\[
		TAT^{-1} = \Lambda = \diag\parentheses*{\lambda_1, \ldots, \lambda_N}
	\]
	where \(\lambda_1, \ldots, \lambda_N \in \R\) are eigenvalues of \(A\).
	We can transform the solution vector \(U\parentheses*{x, t} \in \R^N\) to a new variable (sometimes called ``characteristic variable'')
	\[
		TU\parentheses*{x, t} =: W\parentheses*{x, t} \equiv \parentheses*{w_i\parentheses*{x, t}}_{i = 1, \ldots, N} \in \R^N
	\]
	for which the PDE system reads
	\[
		\partial_t W = T\partial_t U = -TA\partial_x U = -\Lambda T\partial_x U = -\Lambda\partial_x W \iff \partial_t W + \Lambda\partial_x W = 0.
	\]
	Since \(\Lambda\) is diagonal, the system decouples into \(N\) indepedent scalar equations
	\[
		\partial_t w_i + \lambda_i \partial_x w_i = 0, \quad i = 1, \ldots, N.
	\]
	That is, each component \(w_i\) is advected with advection speed \(\lambda_i\).
	The initial condition for \(W\) is \(W^{\parentheses*{0}} = TU^{\parentheses*{0}}\), so that the solution of each component is
	\[
		w_i\parentheses*{x, t} = w_i^{\parentheses*{0}}\parentheses*{x - \lambda_i t}.
	\]
	From this the solution \(U\parentheses*{x, t}\) can be found by back-transforming \(U\parentheses*{x, t} = T^{-1}W\parentheses*{x, t}\).
\end{example}


\subsection{Similarity solution and Riemann problem}

\begin{theorem}[Similarity solution]
	If the intial condition satisfies
	\[
		U^{\parentheses*{0}}\parentheses*{x} = U^{\parentheses*{0}}\parentheses*{\alpha x}, \quad x \in \R,
	\]
	for all \(\alpha > 0\), then the conservation law admits a \emph{similarity} (or \emph{self-similar}) \emph{solution} that satisfies
	\[
		U\parentheses*{x, t} = U\parentheses*{\alpha x, \alpha t}, \quad \parentheses*{x, t} \in \R \times \R^+,
	\]
	for all \(\alpha > 0\).
\end{theorem}

\begin{proof}
	Assuming that \(U\parentheses*{x, t}\) is a solution, we can simply check that \(U\parentheses*{\alpha x, \alpha x}\) is also a solution by inserting this into the conservation law and apply chain rule.
\end{proof}

\begin{remark}
	\begin{enumerate}
		\item A similarity solution \(U\parentheses*{x, t}\) is constant along rays from the origin in the \(\parentheses*{x, t}\)-diagram, that is, along the lines \(\frac{x}{t} = \text{const.}\) with \(t > 0\).
		We have the representation
		\[
			U\parentheses*{x, t} = V\parentheses*{\frac{x}{t}} = V\parentheses*{\xi}
		\]
		with \(\xi = \frac{x}{t}\).
		\item The solution is also called ``self-similar'', because it suffices to specify the solution at any time \(t > 0\) in order to know the solution at any other time.
	\end{enumerate}
\end{remark}

Next we introduce the general Riemann problem.

\begin{definition}[Riemann problem]
	The \emph{Riemann problem} consists of the system of conservation laws
	\[
		\partial_t U + \partial_x F\parentheses*{U} = 0, \quad \text{in }\R \times \parentheses*{0, \infty}
	\]
	equipped with the discontinuous, piecewise constant initial data
	\[
		U^{\parentheses*{0}}\parentheses*{x} = \begin{cases}
			U_L, & \text{if }x < 0,\\
			U_R, & \text{if }x > 0,
		\end{cases}
	\]
	for which the corresponding similarity solutions are sought.
\end{definition}

\begin{remark}
	\begin{enumerate}
		\item In general, the jump of the initial condition does not satisfy the Rankine-Hugoniot conditions.
		The solution will thus be split into several waves due to the initial jump.
		\item The solution to the Riemann problem provides helpful insight into general systems of conservation laws and will be useful for constructing numerical methods.
	\end{enumerate}
\end{remark}


\subsection{Linear Riemann solution}


\subsection{Nonlinear systems}
