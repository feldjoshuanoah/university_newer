\documentclass[english]{exercise}

\DeclareMathOperator{\HI}{HI}
\DeclareMathOperator{\IoU}{IoU}


\title{Homework 3}
\author{Joshua Feld, 406718}
\professor{Prof. Kowalski}
\course{Partial Differential Equations}

\begin{document}
	\maketitle


	\section{}

	\begin{quote}
		Consider the Poisson equation in 1D on the domain \(\Omega = \parentheses*{0, 1}\):
		\begin{align*}
			-\Delta u &= f, \quad \text{in }\Omega,\\
			u &= 0, \quad \text{on }\partial\Omega.
		\end{align*}
		For each of the following choices of the function \(f\), write down the weak formulation of the Poission problem and find the solution to the weak formlation.
		Verify that the proposed weak solution satisfies the weak formulation in each case.
		\begin{enumerate}
			\item \(f \equiv 0\),
			\item \(f = \delta_{\frac{1}{2}}\),
			\item \(f\parentheses*{x} = \begin{cases}
				0, & \text{if }x \le \frac{1}{2},\\
				1, & \text{if }x > \frac{1}{2}.
			\end{cases}\)
		\end{enumerate}
	\end{quote}

	\begin{enumerate}
		\item The PDE is \(-u''\parentheses*{x} = 0\) which implies that the solution has the form \(u\parentheses*{x} = ax + b, a, b \in \R\).
		With the boundary conditions \(u\parentheses*{0} = u\parentheses*{1} = 0\) we obtain \(a = b = 0\) and thus we obtain \(u\parentheses*{x} = 0\) as a classical solution (since it's continuous and differentiable).
		Quite obviuously, it is also a weak solution, because
		\[
			\int_\Omega \nabla u \cdot \nabla v\d x = \int_\Omega 0 \cdot \nabla v\d x = 0 \int_\Omega fv\d x \quad \forall v \in H_0^1\parentheses*{\Omega}.
		\]
		\item Consider the two subintervals \(\parentheses*{0, \frac{1}{2}}\) and \(\parentheses*{\frac{1}{2}, 1}\) on which a classical problem is given because \(f\) is regular.
		We can now apply a linear function to the left and right interval
		\begin{align*}
			u_\ell\parentheses*{x} &= a_\ell x + b_\ell,\\
			u_r\parentheses*{x} &= a_r x + b_r.
		\end{align*}
		The boundary conditions give us \(b_\ell = 0\) and \(a_r + b_r = 0\).
		We now have to connect two straight lines, which are fixed at the points \(0\) and \(1\), such that the functions are continuous and the second derivative results in a negative delta distribution at \(x = \frac{1}{2}\).
		We know that this corresponds to a kink with a slope change of \(-1\) and thus
		\begin{align*}
			u_\ell\parentheses*{\frac{1}{2}} = u_r\parentheses*{\frac{1}{2}} &\iff \frac{1}{2}a_\ell = \frac{1}{2}a_r + b_r,\\
			u_\ell'\parentheses*{\frac{1}{2}} - 1 = u_r'\parentheses*{\frac{1}{2}} &\iff a_\ell - 1 = a_r.
		\end{align*}
		These four equations give us the solution
		\[
			a_\ell = \frac{1}{2}, b_\ell = 0, \quad a_r = -\frac{1}{2}, b_r = \frac{1}{2}.
		\]
		The candidate for the weak solution is
		\[
			u\parentheses*{x} = \begin{cases}
				\frac{1}{2}x, & \text{if }x \le \frac{1}{2},\\
				\frac{1}{2} - \frac{1}{2}x, & \text{if }x > \frac{1}{2}.
			\end{cases}
		\]
		We now only need to verify that this is in fact a weak solution by plugging it into the weak formulation
		\begin{align*}
			\int\nabla u\nabla v\d x &= \int_0^{\frac{1}{2}}\frac{1}{2}\nabla v\d x + \int_{\frac{1}{2}}^1 -\frac{1}{2}\nabla v\d x\\
			&= \frac{1}{2}\parentheses*{v\parentheses*{1}{2} - v\parentheses*{0}} - \frac{1}{2}\parentheses*{v\parentheses*{1} - v\parentheses*{\frac{1}{2}}}\\
			&\stackrel{v\parentheses*{0} = v\parentheses*{1} = 0}{=} v\parentheses*{\frac{1}{2}} = \int_\Omega \delta_{\frac{1}{2}}v\d x \quad \forall v \in H_0^1\parentheses*{\Omega}.
		\end{align*}
		\item Again, consider the two subintervals \(\parentheses*{0, \frac{1}{2}}\) and \(\parentheses*{\frac{1}{2}, 1}\).
		The procedure is very analogous to the previous exercise. We use two functions
		\begin{align*}
			u_\ell\parentheses*{x} &= a_\ell x + b_\ell,\\
			u_r\parentheses*{x} &= -\frac{1}{2}x^2 + a_r x + b_r.
		\end{align*}
		From the boundary conditions we obtain \(b_\ell = 0\) and \(a_r + b_r = \frac{1}{2}\).
		\(u\) and its first derivative need to be continuous.
		We end up with the following two additional equations which help us find the remaining parameters for the candidate of the weak solution:
		\begin{align*}
			u_\ell\parentheses*{\frac{1}{2}} = u_r\parentheses*{\frac{1}{2}} &\iff \frac{1}{2}a_\ell = -\frac{1}{8} + \frac{1}{2}a_r + b_r,\\
			u_\ell'\parentheses*{\frac{1}{2}} = u_r'\parentheses*{\frac{1}{2}} &\iff a_\ell = -\frac{1}{2} + a_r.
		\end{align*}
		Thus
		\[
			a_\ell = \frac{1}{8}, b_\ell = 0, \quad a_r = \frac{5}{8}, b_r = -\frac{1}{8}
		\]
		and
		\[
			u\parentheses*{x} = \begin{cases}
				\frac{1}{8}x, & \text{if }x \le \frac{1}{2},\\
				-\frac{1}{2}x^2 + \frac{5}{8}x - \frac{1}{8}, & \text{if }x > \frac{1}{2}.
			\end{cases}
		\]
		We again verify that this is actually a weak solution
		\begin{align*}
			\int_\Omega \nabla u\nabla v\d x &= \int_0^{\frac{1}{2}}\frac{1}{8}\nabla v\d x + \int_{\frac{1}{2}}^1 \parentheses*{\frac{5}{8} - x}\nabla v\d x\\
			&= \frac{1}{8}\parentheses*{v\parentheses*{\frac{1}{2}} - v\parentheses*{0}} + \brackets*{\parentheses*{\frac{5}{8} - x}v\parentheses*{x}}_{\frac{1}{2}}^1 - \int_{\frac{1}{2}}^1 \nabla\parentheses*{\frac{5}{8} - x}v\d x\\
			&= \frac{1}{8}\parentheses*{v\parentheses*{\frac{1}{2}} - v\parentheses*{0} - 3v\parentheses*{1} - v\parentheses*{\frac{1}{2}}} + \int_{\frac{1}{2}}^1 v\d x\\
			&\stackrel{v\parentheses*{0} = v\parentheses*{1} = 0}{=} \int_{\frac{1}{2}}^1 v\d x = \int_\Omega fv\d x \quad \forall v \in H_0^1\parentheses*{\Omega}.
		\end{align*}
	\end{enumerate}


	\section{}

	\begin{quote}
		Let \(\Omega\) be a bounded set with boundary \(\partial\Omega\), where \(\partial\Omega = \Gamma_1 \cup \Gamma_2, \Gamma_1 \cap \Gamma_2 = \emptyset\).
		Additionally, let \(c > 0\) and let the functions \(f, g, p, q \in C^0\parentheses*{\Omega}\).
		Next, we define the set \(V := \braces*{v \in H^1\parentheses*{\Omega} : \left.v\right|_{\Gamma_1} = 0}.\).
		Assume that the function \(u \in H^1\parentheses*{\Omega}\) with \(\left.u\right|_{\Gamma_1} = g\) satisfies the following weak formulation:
		\[
			\int_\Omega \parentheses*{\nabla u\nabla v + cuv}\d x + \int_{\Gamma_2}\parentheses*{pu - q}v\d s = \int_\Omega fv\d x \quad \forall v \in V.
		\]
		If \(u \in C^2\parentheses*{\Omega} \cap C^1\parentheses*{\bar{\Omega}}\), then what classical PDE does \(u\) solve?
	\end{quote}

	Since \(v = 0\) on \(\Gamma_1\) we obtain using integration by parts that
	\[
		\int_\Omega \parentheses*{-\Delta u + cu}v\d x + \int_{\Gamma_2}\parentheses*{\frac{\partial u}{\partial n} + pu - q}v\d s = \int_\Omega fv\d x
	\]
	for all test functions \(v \in V\).
	In particular for \(v \in H_0^1\parentheses*{\Omega} \subset V\) we obtain
	\[
		\int_\Omega \parentheses*{-\Delta u + cu}v\d x = \int_\Omega fv\d x \iff -\Delta u + cu = f, \quad \text{in }\Omega.
	\]
	Returning to the first equation, we see that we must have
	\[
		\int_{\Gamma_2}\parentheses*{\frac{\partial u}{\partial n} + pu - q}v\d s = 0
	\]
	for all \(v \in V\).
	Naturally, this yields the boundary condition
	\[
		\frac{\partial u}{\partial n} + pu = q, \quad \text{on }\Gamma_2.
	\]
	We conclude that the classical PDE associated with the above weak formulation is given by
	\begin{align*}
		-\Delta u + cu &= f, \quad \text{in }\Omega,\\
		u &= g, \quad \text{on }\Gamma_1,\\
		\frac{\partial u}{\partial n} + pu &= q, \quad \text{on }\Gamma_2.
	\end{align*}


	\section{}

	\begin{quote}
		Let \(\Omega \subset \R\) be an open bounded domain.
		Consider the following Robin boundary (which is a combination of Dirichlet and Neumann boundary conditions) value problem
		\begin{align*}
			-\Delta u + \alpha u &= f, \quad \text{in }\Omega,
			\nabla u \cdot n + \beta u &= g, \quad \text{on }\partial\Omega,
		\end{align*}
		where \(f, g \in L^2\parentheses*{\Omega}\).
		\begin{enumerate}
			\item Derive the weak formulation of the above problem, including the statement of reasonable function spaces for the solution and test functions.
			\item Prove that for any \(\alpha > 0\) and \(\beta \ge 0\) there exists a unique solution.
		\end{enumerate}
	\end{quote}

	\begin{enumerate}
		\item Multiplying the given differential equation by \(v\parentheses*{x} \in H^1\parentheses*{\Omega}\) and integrating over the domain \(\Omega\), we get
		\[
			-\int_\Omega v\Delta u\d x + \int_\Omega \alpha uv\d x = \int_\Omega fv\d x.
		\]
		Integrating by parts, we get
		\begin{align*}
			\int_\Omega \nabla u \cdot \nabla v\d x - \int_{\partial\Omega}\parentheses*{\nabla u \cdot n}v\d x + \int_\Omega \alpha uv\d x &= \int_\Omega fv\d x\\
			\iff \int_\Omega \nabla u \cdot \nabla v\d x - \int_{\partial\Omega}\parentheses*{g - \beta u}v\d x + \int_\Omega \alpha uv\d x &= \int_\Omega fv\d x.
		\end{align*}
		The weak formulation of the given problem can thus be given as:
		Find \(u \in H^1\parentheses*{\Omega}\) such that
		\[
			a\parentheses*{u, v} = \ell\parentheses*{v} \quad \forall v \in H^1\parentheses*{\Omega},
		\]
		where
		\begin{align*}
			a\parentheses*{u, v} &= \int_\Omega \nabla u \cdot \nabla v\d x + \int_\Omega \alpha uv\d x + \int_{\partial\Omega}\beta uv\d x,\\
			\ell\parentheses*{v} &= \int_\Omega fv\d x + \int_{\partial\Omega}gv\d x.
		\end{align*}
		\item We first look at the continuity property:
		\begin{align*}
			\absolute*{a\parentheses*{u, v}} &= \absolute*{\int_\Omega \nabla u \cdot \nabla v\d x + \int_\Omega \alpha uv\d x + \int_{\partial\Omega}\beta uv\d x}\\
			&\le \absolute*{\int_\Omega \nabla u \cdot \nabla v\d x} + \absolute*{\int_\Omega \alpha uv\d x} + \absolute*{\int_{\partial\Omega}\beta uv\d x}\\
			&\le \sqrt{\int_\Omega \absolute*{\nabla u}^2 \d x}\sqrt{\int_\Omega \absolute*{\nabla v}^2 \d x} + \absolute*{\alpha}\sqrt{\int_\Omega \absolute*{u}^2 \d x}\sqrt{\int_\Omega \absolute*{v}^2 \d x} + \absolute*{\beta}\sqrt{\int_{\partial\Omega}\absolute*{u}^2 \d x}\sqrt{\int_{\partial\Omega}\absolute*{v}^2 \d x}\\
			&\le \sqrt{\int_\Omega \parentheses*{\absolute*{\nabla u}^2 + \absolute*{u}^2}\d x}\sqrt{\int_\Omega \parentheses*{\absolute*{\nabla v}^2 + \absolute*{v}^2}\d x} + \absolute*{\alpha}\sqrt{\int_\Omega \parentheses*{\absolute*{\nabla u}^2 + \absolute*{u}^2}\d x}\sqrt{\int_\Omega \parentheses*{\absolute*{\nabla v}^2 + \absolute*{v}^2}\d x}\\
			&\quad + \absolute*{\beta}\sqrt{\int_{\partial\Omega}\parentheses*{\absolute*{\nabla u}^2 + \absolute*{u}^2}\d x}\sqrt{\int_{\partial\Omega}\parentheses*{\absolute*{\nabla v}^2 + \absolute*{v}^2}\d x}\\
			&\le \norm*{u}_1 \norm*{v}_1 + \absolute*{\alpha}\norm*{u}_1 \norm*{v}_1 + \absolute*{\beta}\norm*{u}_1 \norm*{v}_1\\
			&= \parentheses*{1 + \absolute*{\alpha} + \absolute*{\beta}}\norm*{u}_1 \norm*{v}_1.
		\end{align*}
		Therefore, the bilinear form \(a\parentheses*{u, v}\) is continuous for \(\alpha, \beta \ge 0\).
		We now look at the coercive property.
		Notice that if \(\beta \ge 0\) then we have
		\begin{align*}
			a\parentheses*{v, v} &= \int_\Omega \nabla u \cdot \nabla v\d x + \int_\Omega \alpha vv\d x + \int_{\partial\Omega}\beta vv\d x\\
			&\ge \int_\Omega \absolute*{\nabla u}^2 \d x + \alpha\int_\Omega \absolute*{v}^2 \d x\\
			&\ge \min\parentheses*{1, \alpha}\int_\Omega \parentheses*{\absolute*{\nabla v}^2 + \absolute*{v}^2}\d x.
		\end{align*}
		Consequently, the bilinear form is coercive if \(\alpha > 0\) and \(\beta \ge 0\).
		In can easily be verified that the functional \(\ell\parentheses*{v}\) is continuous and further it does not depend on \(\alpha\) and \(\beta\).
		Therfore the given boundary value problem has a unique solution for all \(\alpha \ge 1, \beta > 0\).
	\end{enumerate}
\end{document}
