\documentclass[english]{exercise}

\DeclareMathOperator*{\cond}{cond}
\DeclareMathOperator*{\diag}{diag}
\DeclareMathOperator*{\dist}{dist}
\DeclareMathOperator*{\esssup}{ess\,sup}
\DeclareMathOperator*{\vol}{vol}


\title{Homework 1}
\author{Joshua Feld, 406718}
\professor{Prof. Kowalski}
\course{Partial Differential Equations}

\begin{document}
	\maketitle


	\section{}

	\begin{enumerate}
		\item
		\begin{enumerate}
			\item For all \(n \in \N\) we have that \(u\) is continuous and bounded in \(\Omega\).
			Since \(\Omega\) is a bounded set \(u\) is in each one of the \(L^p\) spaces.
			Clearly \(u\) does not have compact support in \(\Omega\) for any \(n \in \N\).
			Next, \(u\) is continuous for all \(n \in \N\).
			Furthermore we observe that for all \(n \in \N\) it holds that
			\[
				u\parentheses*{x} = \absolute*{x}^n = \begin{cases}
					\absolute*{x}^{2k + 1}, k \in \N, & \text{if }n\text{ is odd},\\
					\absolute*{x}^{2k}, k \in \N, & \text{if }n\text{ is even}.
				\end{cases}
			\]
			Consequently, if \(n\) is even, then \(u\) is just a polynomial so it is infinitely differentiable.
			On the other hand, if \(n\) is odd, then \(u\) is product of a polynomial and the absolute value function.
			Consequently, it cannot be differentiable (by the product rule).
			\item \(u\) has a singularity at \(x = 0\) so it is neither continuous, differentiable nor bounded.
			Clearly the support of \(u\) is not compact in \(\Omega\).
			Finally, the following calculations show that \(u \not\in L^1\parentheses*{\Omega}, u \not\in L^2\parentheses*{\Omega}\) for \(n = 1\), \(u \in L^1\parentheses*{\Omega}, u \not\in L^2\parentheses*{\Omega}\) for \(n = 2\) and \(u \in L^1\parentheses*{\Omega}, u \in L^2\parentheses*{\Omega}\) otherwise.

			\(n = 1\):
			\begin{align*}
				\norm*{u}_{L^1} &= \int_{-2}^2 \frac{1}{\absolute*{x}}\d x = 2\int_0^2 \frac{1}{x}\d x = 2\brackets*{\ln\parentheses*{x}}_0^2 = 2\parentheses*{\ln 2 - \lim_{x \to 0}\ln x} \to \infty \implies u \not\in L^1\parentheses*{\Omega},\\
				\norm*{u}_{L^1}^2 &= \int_{-2}^2 \absolute*{x}^{-2}\d x = 2\int_0^2 x^{-2}\d x = -2\brackets*{\frac{1}{x}}_0^2 = -2\parentheses*{\frac{1}{2} - \lim_{x \to 0}\frac{1}{x}} \to \infty \implies u \not\in L^2\parentheses*{\Omega}.
			\end{align*}
			\(n = 2\):
			\begin{align*}
				\norm*{u}_{L^1} &= \int_{-2}^2 \absolute*{x}^{-\frac{1}{2}}\d x = 2\int_0^2 \frac{1}{\sqrt{x}}\d x = 4\brackets*{\sqrt{x}}_0^2 = 4\parentheses*{\sqrt{2} - \sqrt{0}} < \infty \implies u \in L^1\parentheses*{\Omega},\\
				\norm*{u}_{L^2}^2 &= \int_{-2}^2 \absolute*{\frac{1}{\sqrt{\absolute*{x}}}}^2\d x = 2\int_0^2 \frac{1}{x}\d x = 2\brackets*{\ln\parentheses*{x}}_0^2 = 2\parentheses*{\ln 2 - \lim_{x \to 0}\ln x} \to \infty \implies u \not\in L^2\parentheses*{\Omega}.
			\end{align*}
			\(n = 3\):
			\begin{align*}
				\norm*{u}_{L^1} = \int_{-2}^2 \absolute*{x}^{-\frac{1}{n}}\d x = 2\int_0^2 x^{-\frac{1}{n}}\d x = \frac{2}{1 - \frac{1}{n}}\brackets*{x^{1 - \frac{1}{n}}}_0^2 = \frac{2}{1 - \frac{1}{n}}\parentheses*{2^{1 - \frac{1}{n}} - 0} < \infty \implies u \in L^1\parentheses*{\Omega},\\
				\norm*{u}_{L^2} = \int_{-2}^2 \absolute*{x}^{-\frac{2}{n}}\d x = 2\int_0^2 x^{-\frac{2}{n}}\d x = \frac{2}{1 - \frac{2}{n}}\brackets*{x^{1 - \frac{2}{n}}}_0^2 = \frac{2}{1 - \frac{2}{n}}\parentheses*{2^{1 - \frac{2}{n}} - 0} < \infty \implies u \in L^2\parentheses*{\Omega}.
			\end{align*}
			\item First \(u\) goes to zero at \(\pm 1\) exponentially fast, so \(u\) is continuous.
			Compute
			\[
				u'\parentheses*{x} = \begin{cases}
					-\exp\parentheses*{-\frac{1}{1 - \absolute*{x}^2}} \cdot \frac{2x}{\parentheses*{1 - x^2}^2}, & \text{if }\absolute*{x} < 1,\\
					0, & \text{elsewhere}.
				\end{cases}
			\]
			When \(\absolute*{x} < 1\), this is a product of a rational function with \(u\) itself.
			It is itself hence differentiable.
			Forthermore, \(u'\parentheses*{x}\) is bounded as there are no singularities within \(\parentheses*{-1, 1}\) and the function is continuous therein.
			But is \(u'\parentheses*{x}\) continuous differentiable 
			\item The logarithm is neither continuous nor bounded due to its singularity, but it is integrable in \(L^1\parentheses*{\Omega}\) and \(L^2\parentheses*{\Omega}\).
			This can be shown by a direct calculation.
			\item \(u\) is obviously not continuous or differentiable, but bounded and hence also in \(L^1\parentheses*{\Omega}\) and \(L^2\parentheses*{\Omega}\) as \(\Omega\) is bounded.
		\end{enumerate}
		\item We use a change of variables to the spherical coordinate system \(\parentheses*{r, \phi, \varphi_1, \ldots, \varphi_{n - 2}}\) in \(\R^n, n \ge 3\).
		The determinant of the Jacobian associated with this change in variables is given by
		\[
			r^{n - 1}\sin\parentheses*{\varphi_1}\sin\parentheses*{\varphi_2}^2 \cdots \sin\parentheses*{\varphi_{n - 2}}^{n - 2},
		\]
		where \(\phi \in \left[0, 2\pi\right), \varphi_i \in \left[0, \pi\right)\).
		Consequently, we obtain that
		\begin{align*}
			\int_{B_1\parentheses*{0}}\parentheses*{\frac{1}{\absolute*{x}^q}}^p \d x &= \int_0^{2\pi}\int_0^\pi \cdots \int_0^\pi \int_0^1 \frac{1}{r^{pq}}r^{n - 1}\sin\parentheses*{\varphi_1} \cdots \sin\parentheses*{\varphi_{n - 2}}^{n - 2}\d r\d\varphi_1 \cdots \d\varphi_{n - 2}\d\phi\\
			&= C\parentheses*{n}\int_0^1 r^{n - 1 - pq}\d r,
		\end{align*}
		where \(C\) is some constant depending on the dimension \(n\).
		We immediately see that this integral exists if and only if
		\[
			n - 1 - pq > -1 \iff n > pq.
		\]
		In other words \(u \in L^p\parentheses*{\Omega}\) if and only if \(pq < n\).
		As an example, for the case \(n = 3\), the function \(u\parentheses*{x} := \frac{1}{\norm*{x}_2}\) is in \(L^1\parentheses*{\Omega}\) and \(L^2\parentheses*{\Omega}\) but not in any \(L^p\parentheses*{\Omega}\) for \(p \ge 3\).
		Notice that the function \(u\) has a singularity at the origin but is still integrable.
	\end{enumerate}


	\section{}

	\begin{enumerate}
		\item We first write the Laplace operator in polar coordinates (see formula given in the hint):
		\[
			\frac{\partial^2 u}{\partial r^2} + \frac{1}{r}\frac{\partial u}{\partial r} + \frac{1}{r^2}\frac{\partial^2 u}{\partial\varphi^2} = 0, \quad 0 \le r \le r_{\text{max}}, 0 \le \varphi \le \alpha.
		\]
		We will now be using the seperation of variables approach
		\[
			u\parentheses*{r, \varphi} = R\parentheses*{r}\Phi\parentheses*{\varphi}.
		\]
		Inserting this approach into the partial differential equation yields
		\[
			R''\parentheses*{r}\Phi\parentheses*{\varphi} + \frac{1}{r}R'\parentheses*{r}\Phi\parentheses*{\varphi} + \frac{1}{r^2}R\parentheses*{r}\Phi''\parentheses*{\varphi} = 0,
		\]
		and simplifying this expression gives
		\[
			r^2 \frac{R''\parentheses*{r}}{R\parentheses*{r}} + r\frac{R'\parentheses*{r}}{R\parentheses*{r}} + \frac{\Phi''\parentheses*{\varphi}}{\Phi\parentheses*{\phi}} = 0.
		\]
		Since this equation mus be fulfilled for all \(r\) and \(\varphi\), the terms \(r^2 \frac{R''\parentheses*{r}}{R\parentheses*{r}} + r\frac{R'\parentheses*{r}}{R\parentheses*{r}}\) and \(\frac{\Phi''\parentheses*{\varphi}}{\Phi\parentheses*{\varphi}}\) must each be equal to some constant.
		Consequently, we obtain two ordinary differential equations:
		\begin{align}
			\Phi''\parentheses*{\varphi} + \lambda^2 \Phi\parentheses*{\varphi} &= 0, \label{eq:1}\\
			r^2 R''\parentheses*{r} + r R'\parentheses*{r} - \lambda^2 R\parentheses*{r} &= 0. \label{eq:2}
		\end{align}
		Here, the contant \(\lambda\) is as of yet unknown and must be determined.
		\item In order to solve the angular ordinary differential equation \eqref{eq:1} we use the approach
		\begin{equation}\label{eq:3}
			\Phi\parentheses*{\varphi} = C_1 \sin\parentheses*{\mu\varphi} + C_2 \cos\parentheses*{\mu\varphi}.
		\end{equation}
		Inserting approach \eqref{eq:3} into the ordinary differential equation yields \(\mu = \lambda\).
		Furthermore, the boundary conditions imply that \(\Phi\parentheses*{0} = 0\) so that \(C_2 = 0\).
		Next, since \(\sin\parentheses*{\lambda\alpha} = 0\) we obtain \(\lambda = \frac{\pi}{\alpha}\).
		Consequently, the function \(\Phi\) is given by
		\[
			\Phi\parentheses*{\varphi} = C_1 \sin\parentheses*{\frac{\pi}{\alpha}\varphi}.
		\]
		In order to solve the radial ordinary differential equation \eqref{eq:2}, we use the approach \(R\parentheses*{r} = r^\beta\).
		Inserting this in equation \eqref{eq:2} immediately yields \(\beta = \lambda = \frac{\pi}{\alpha}\).
		Consequently, it holds that
		\[
			u\parentheses*{r, \varphi} = C_1 \sin\parentheses*{\frac{\pi}{\alpha}\varphi}r^{\frac{\pi}{\alpha}},
		\]
		where the constant \(C_1\) must be determined from the boundary condition \(u\parentheses*{r_{\text{max}}, \varphi} = \sin\parentheses*{\frac{\pi}{\alpha}\varphi}\).
		Indeed, a simple solution yields \(C_1 = \parentheses*{\frac{1}{r_{\text{max}}}}^{\frac{\pi}{\alpha}}\).
		The complete solution is thus given by
		\[
			u\parentheses*{r, \varphi}\sin\parentheses*{\frac{\pi}{\alpha}\varphi}\parentheses*{\frac{r}{r_{\text{max}}}}^{\frac{\pi}{\alpha}}.
		\]
	\end{enumerate}


	\section{}

	\begin{enumerate}
		\item We look at the Taylor expansion in \(x\)-direction.
		We implicitly assume a discretization of the domain with a grid of width \(h\).
		\begin{align*}
			u\parentheses*{x_i + h, y_i} &= u\parentheses*{x_i, y_i} + h\left.\frac{\partial u}{\partial x}\right|_{x_i, y_i} + \frac{1}{2}h^2 \left.\frac{\partial^2 u}{\partial x^2}\right|_{x_i, y_i} + \mathcal{O}\parentheses*{h^3},\\
			u\parentheses*{x_i - h, y_i} &= u\parentheses*{x_i, y_i} - h\left.\frac{\partial u}{\partial x}\right|_{x_i, y_i} + \frac{1}{2}h^2 \left.\frac{\partial^2 u}{\partial x^2}\right|_{x_i, y_i} + \mathcal{O}\parentheses*{h^3}.
		\end{align*}
		A summation and rearrangement gives
		\[
			\left.\frac{\partial^2 u}{\partial x^2}\right|_{x_i, y_i} \approx \frac{u\parentheses*{x_i + h, y_i} - 2u\parentheses*{x_i, y_i} + u\parentheses*{x_i - h, y_i}}{h^2}.
		\]
		Analogously for the second derivative with respect to \(y\) we get
		\[
			\left.\frac{\partial^2 u}{\partial y^2}\right|_{x_i, y_i} \approx \frac{u\parentheses*{x_i, y_i + h} - 2u\parentheses*{x_i, y_i} + u\parentheses*{x_i, y_i - h}}{h^2}.
		\]
		This way we obtain a discretization of the Laplacian on the whole domain.
		For the Neumann boundary conditions we need a one-sided stencil.
		A rearrangement of the expression
		\[
			u\parentheses*{x_i - h, y_i} = u\parentheses*{x_i, y_i} - h\left.\frac{\partial u}{\partial x}\right|_{x_i, y_i} + \mathcal{O}\parentheses*{h^2}
		\]
		gives
		\[
			\left.\frac{\partial u}{\partial x}\right|_{x_i, y_i} \approx \frac{u\parentheses*{x_i, y_i} - u\parentheses*{x_i - h, y_i}}{h}.
		\]
		\item
	\end{enumerate}
\end{document}
