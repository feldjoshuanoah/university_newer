\documentclass[english]{exercise}

\DeclareMathOperator*{\cond}{cond}
\DeclareMathOperator*{\diag}{diag}
\DeclareMathOperator*{\dist}{dist}
\DeclareMathOperator*{\esssup}{ess\,sup}
\DeclareMathOperator*{\vol}{vol}


\title{Self Exercise 4}
\author{Joshua Feld (406718)}
\professor{Prof. Kowalski}
\course{Partial Differential Equations}

\begin{document}
	\maketitle


	\section{}

	\begin{quote}
		Consider the following boundary value problem over the domain \(\Omega = \parentheses*{0, 1} \subset \R\)
		\begin{align*}
			\frac{\d}{\d x}\parentheses*{\parentheses*{1 + x}\frac{\d u}{\d x}} &= 1,\\
			u\parentheses*{0} = u\parentheses*{1} &= 0,
		\end{align*}
		with \(u \in H_0^1\parentheses*{0, 1}\).
		\begin{enumerate}
			\item Derive the continuous and discrete variational formulations.
			\item Choose the trigonometric functions, \(\psi_j\parentheses*{x} = \sin\parentheses*{j\pi x}, j = 1, \ldots, N\) as the basis functions.
			Derive the Ritz-Galerkin equations and find the discrete solution \(u_h \in H_h \subset H_0^1\parentheses*{0, 1}\).
			\item Find the Ritz-Galerkin solution for the cases of \(N = 2\) (which results in two basis functions \(\psi_1\) and \(\psi_2\)).
			Plot the exact solution and the approximate solution over the domain \(\parentheses*{0, 1}\).
		\end{enumerate}
	\end{quote}

	\begin{enumerate}
		\item
		\item
		\item
	\end{enumerate}


	\section{}

	\begin{quote}
		We want to approximate the discontinuous function
		\[
			f\parentheses*{x} = \begin{cases}
				1, & \text{if }x < 0,\\
				-1, & \text{if }x \ge 0,
			\end{cases},
		\]
		for \(x \in \brackets*{-L, L}\) by a piecewise linear function.
		The approximation has the form
		\[
			s\parentheses*{x} = \sum_{i = -N}^{N + 1}\alpha_i \phi_i\parentheses*{x}
		\]
		with hat functions \(\phi_i\) at the grid points
		\[
			x_i = \parentheses*{i - \frac{1}{2}}\Delta x, \quad \Delta x = \frac{2L}{2N + 1}, \quad -N \le i \le N + 1
		\]
		and the boundary conditions \(\alpha_{-N} = f\parentheses*{x_{-N}}\) and \(\alpha_{N + 1} = f\parentheses*{x_{N + 1}}\).
		The approximation should be optimal in the sense that the error \(\norm*{f - s}\) is minimal in a suitable norm.
		\begin{enumerate}
			\item Consider the \(L^2\) norm and solve the minimization problem
			\[
				\norm*{f - s}_s^2 \to \min
			\]
			in a general setting.
			\item Pick \(L = 3, N = 1, 2, 3\) and plot the solution next to the function \(f\).
			Would you call this an optimal solution?
			\item Write down the minimization problem for the \(L^1\) norm.
			What is the difficulty here?
			How would you solve this problem?
			\item Show for \(N = 1\) assuming point symmetry that the interpolating spline \(\alpha_i = f\parentheses*{x_i}\) minimizes the \(L^1\) norm.
		\end{enumerate}
	\end{quote}

	\begin{enumerate}
		\item
		\item
		\item
		\item
	\end{enumerate}
\end{document}
