\section{Theory of Weak Solutions}

The classical solution theory of the Poisson problem \(-\Delta u = f\) in \(\Omega\) with boundary conditions \(u = 0\) on \(\partial\Omega\) requires two classical derivatives.
That is, to at least make sense of the Laplace operator \(\Delta\), we require \(u \in C^2\parentheses*{\Omega}\).
We have seen, that this might be too restrictive in general.
To remedy this shortcoming we need to reformulate the problem statement and allow for a ``weaker'' solution concept.


\subsection{Variational Formulation}

We multiply the Poisson problem by an arbitrary, sufficiently smooth test function \(v \in V\) defined on \(\Omega\) and integrate over \(\Omega\):
\[
	-\int_\Omega v \cdot \Delta u\d x = \int_\Omega v \cdot f\d x.
\]
Partial integration for \(\Delta u = \div\parentheses*{\grad u}\) yields
\begin{align*}
	-\int_\Omega v \cdot \Delta u\d x &= -\int_\Omega \parentheses*{\div\parentheses*{v\grad u} - \grad u \cdot \grad v}\d x\\
	&= -\int_\Omega \div\parentheses*{v\grad u}\d x + \int_\Omega \grad u \cdot \grad v\d x\\
	&= -\int_{\partial\Omega}vn \cdot \grad u\d s + \int_\Omega \grad u \cdot \grad v\d x
\end{align*}
where we used Gauss' theorem, denoting the outward pointing normal by \(n\) on the domain boundary \(\partial\Omega\).
Rearranging terms, we arrive at
\[
	\int_\Omega \grad u \cdot \grad v\d x - \int_{\partial\Omega}vn \cdot \grad u\d s = \int_\Omega v \cdot f\d x.
\]
We observe that the boundary integral is ready to accommodate Neumann boundary conditions (comes later).
Moreover, homogeneous Dirichlet conditions \(u = 0\) on \(\partial\Omega\) are commonly ``built'' into the function space, that is, we simply require the test functions to satisfy \(\left.v\right|_{\partial\Omega} = 0\), so that the boundary integral vanishes.

\begin{definition}
	The problem statement
	\begin{quote}
		``Find the function \(u: \Omega \to \R\) with \(\left.u\right|_{\partial\Omega} = 0\) such that the relation
		\[
			\int_\Omega \nabla u \cdot \nabla v\d x = \int_\Omega v \cdot f\d x
		\]
		is satisfied for all test functions \(v \in V\).''
	\end{quote}
	is called the \emph{weak} or \emph{variational formulation} of the PDE problem
	\[
		-\Delta u = f\text{ in }\Omega, \quad \text{and} \quad u = 0\text{ on }\partial\Omega.
	\]
	Its solution is called \emph{weak solution}.
\end{definition}

\begin{remark}
	\begin{enumerate}
		\item In, e.g., 2D, we have \(\nabla u \cdot \nabla v = \grad u \cdot \grad v = \partial_x u \partial_x v + \partial_y u\partial_y v\) for the scalar product.
		\item Every classical, ``strong'' solution of the Poisson problem is also a weak solution of the variational formulation.
		Just insert that strong solution into the formulation and proceed the above derivation backwards.
		\item We need to specify the function spaces that the solution \(u\) and the test function \(v\) are supposed to live in.
		They  should be large enough, complete, and with scalar product.
		Foremost, it should be such that the integrals in the formulation make sense.
	\end{enumerate}
\end{remark}

\begin{theorem}\label{theorem:3-3}
	If \(u, v \in H^1\parentheses*{\Omega}\), it holds that
	\[
		\absolute*{\int_\Omega \nabla u \cdot \nabla v\d x} \le \norm*{u}_1 \norm*{v}_1 < \infty.
	\]
\end{theorem}

\begin{proof}
	Using the Cauchy-Schwartz inequality we can split the integral into two integrals and use the definition of the Sobolev norm
	\[
		\absolute*{\int_\Omega \nabla u \cdot \nabla v\d x}^2 \le \int_\Omega \absolute*{\nabla u}^2 \d x \int_\Omega \absolute*{\nabla v}^2 \d x = \norm*{\nabla u}_0^2 \norm*{\nabla v}_0^2 \le \norm*{u}_1^2 \norm*{v}_1^2,
	\]
	which is bounded because \(u, v \in H^1\parentheses*{\Omega}\).
\end{proof}

\begin{remark}
	\begin{enumerate}
		\item Hence, we will require \(u, v \in V := H_0^1\parentheses*{\Omega}\), with functions vanishing at the boundary and the variational formulation for the Poisson problem with homogeneous Dirichlet boundary conditions reads:
		\begin{quote}
			``Find \(u \in H_0^1\parentheses*{\Omega}\) such that \(\int_\Omega \nabla u \cdot \nabla v\d x = \int_\Omega vf\d x\) für alle \(v \in H_0^1\parentheses*{\Omega}\).'' 
		\end{quote}
		\item This formulation allows more solutions to the Poisson problem than the classical statement.
	\end{enumerate}
\end{remark}

\begin{example}
	For \(n = 1\) and \(\Omega = \brackets*{-1, 1}\) the function
	\[
		u\parentheses*{x} = \frac{1}{2}\parentheses*{1 - \absolute*{x}}
	\]
	is a weak solution of the Poisson problem with irregular right-hand side \(f\parentheses*{x} = \delta_0\parentheses*{x}\) (Dirac).
	That is, we need to verify that the variational formulation
	\[
		\int_{-1}^1 u'v'\d x = \int_{-1}^1 fv\d x,
	\]
	holds for all \(v \in H_0^1\parentheses*{\brackets*{-1, 1}}\).
	For the right-hand side we immediately find
	\[
		\int_{-1}^1 fv\d x = \int_{-1}^1 \delta_{-1}^1 \delta_0\parentheses*{x}v\parentheses*{x}\d x = v\parentheses*{0}
	\]
	using the properties of the Dirac distribution.
	For the left-hand side we insert the piecewise derivative of the solution
	\[
		\int_{-1}^1 u'v'\d x = \int_{-1}^0 \frac{1}{2}v'\parentheses*{x}\d x + \int_0^1 -\frac{1}{2}v'\parentheses*{x}\d x = \frac{1}{2}\parentheses*{v\parentheses*{0} - v\parentheses*{-1}} - \frac{1}{2}\parentheses*{v\parentheses*{1} - v\parentheses*{0}}.
	\]
	We conclude by noting that on the boundary \(v\parentheses*{1} = v\parentheses*{-1} = 0\) holds for all test functions \(v \in H_0^1\parentheses*{\brackets*{-1, 1}}\).
	Thus \(u\) is a weak solution of the problem, but not a classical, strong solution, because \(u \not\in C^2\parentheses*{\brackets*{-1, 1}}\).
\end{example}


\subsection{Lax-Milgram Theorem}

To formulate the variational formulation in a more compact form, we introduce the following operators (for a fixed function \(f\)):
\begin{align*}
	a: H_0^1\parentheses*{\Omega} \times H_0^1\parentheses*{\Omega} \to \R, &\parentheses*{u, v} \mapsto a\parentheses*{u, v} := \int_\Omega \nabla u \cdot \nabla v\d x,\\
	\ell: H_0^1\parentheses*{\Omega} \to \R, &v \mapsto \ell\parentheses*{v} := \int_\Omega fv\d x.
\end{align*}
With this \(a\parentheses*{\cdot, \cdot}\) is linear in both arguments, maps to a real number, and it holds that \(a\parentheses*{u, v} = a\parentheses*{v, u}\).
Hence, \(a\) is a symmetric bilinear form.
Similarly, \(\ell\parentheses*{\cdot}\) is linear and maps to a real number; hence, \(\ell\) is a linear functional.

If we replace \(H_0^1\parentheses*{\Omega}\) by a general space placeholder \(V\), the variational formulation reads:
\begin{quote}
	``Find \(u \in V\) such that \(a\parentheses*{u, v} = \ell\parentheses*{v}\) for all \(v \in V\).''
\end{quote}
It turns out that many problems (not just the Poisson problem) can be cast as abstract variational formulations.

So far we have not yet discussed whether or not theses abstract variational formulations are well-posed, in the sense of providing a unique solution.
The answer is affirmative for a large class of variational formulations.

\begin{theorem}
	Let \(V\) be a Hilbert space with norm \(\norm*{\cdot}_V\), \(a: V \times V \to \R\) a bilinear form, and \(\ell: V \to \R\) a linear functional.
	Suppose that
	\begin{enumerate}
		\item \(a\) is continuous, i.e., \(\absolute*{a\parentheses*{u, v}} \le C_1\norm*{u}_V\norm*{v}_V\),
		\item \(\ell\) is continuous, i.e., \(\absolute*{\ell\parentheses*{v}} \le C_1\norm*{v}_V\),
		\item \(a\) is coercive, i.e., \(a\parentheses*{u, u} \ge \gamma\norm*{u}_V^2\) with \(\gamma > 0\)
	\end{enumerate}
	for all \(u, v \in V\).
	Then the variational formulation
	\begin{quote}
		``Find \(u \in V\) such that \(a\parentheses*{u, v} = \ell\parentheses*{v}\) for all \(v \in V\).''
	\end{quote}
	has a unique solution, which satisfies
	\[
		\norm*{u}_V \le \frac{1}{\gamma}\norm*{\ell}_{V}.
	\]
\end{theorem}

\begin{proof}
	The existence proof of a unique weak solution can be found in any decent book on PDEs, e.g., the one by Evans.
	Here, we only show the norm estimate, which is obviously true for \(u = 0\).
	Without losss of generality we therefore assume that \(u \ne 0\) solves the variational formulation.
	Together with the coercivity condition we find
	\[
		\gamma\norm*{u}_V^2 \le a\parentheses*{u, u} = \absolute*{\ell\parentheses*{u}} \iff \gamma\norm*{u}_V \le \frac{\absolute*{\ell\parentheses+{u}}}{\norm*{u}_V},
	\]
	from which the claim follows.
\end{proof}

\begin{remark}
	\begin{enumerate}
		\item Notice that the Lax-Milgram theorem does not require symmetry of \(a\).
		Consequently, it is possible to show existence of a unique weak solution for many different elliptic PDEs.
		\item Coercivity (sometimes called \(V\)-ellipticity) is crucial and often harder to check than continuity.
		In the finite dimensional case \(V = \R^m\), \(u, v\) are \(\R^m\) vectors and the bilinear form \(a\) would be represented by a quadratic form \(a\parentheses*{u, v} = u^T Av\) with matrix \(A \in \R^{m \times m}\).
		In that case the coercivity condition reads
		\[
			u^T Au \ge \gamma u^T u,
		\]
		for some \(\gamma > 0\), which is satisfied if \(A\) is strictly positive definite.
		\item The functional \(\ell: V \to \R\) contains the right-hand side \(f\) (in the Poisson case) which can be considered the ``input'' or ``data'' to the problem.
		The Lax-Milgram theorem states
		\[
			\norm*{u}_V \le \frac{1}{\gamma}\norm*{\ell}_{V'} = \frac{1}{\gamma}\sup_{v \in V, v \ne 0}\frac{\absolute*{\ell\parentheses*{v}}}{\norm*{v}_V}
		\]
		for any abstract variational formulation.
		This means the solution \(u\) is ``controlled'' by the right-hand side, or ``\(u\) depends continuously on the data'', or ``the problem is stable''.
		\item This is an important, general concept:
		If the solution of a mathematical problem exists, is unique, and the problem is stable, then the problem is called ``well-posed'' (in the sense of J. Hadamard).
	\end{enumerate}
\end{remark}


\subsection{Well-posedness for the Possion problem}

We now use the Lax-Milgram theorem to show that the variational formulation of the Poisson problem
\begin{quote}
	``Find \(u \in H_0^1\parentheses*{\Omega}\) such that \(\int_\Omega \nabla v \cdot \nabla u\d x = \int_\Omega fv\d x\) for all \(v \in H_0^1\parentheses*{\Omega}\).''
\end{quote}
has a unique solution.

\begin{example}
	First we identify
	\[
		V = H_0^1\parentheses*{\Omega}, \quad a\parentheses*{u, v} = \int_\Omega \nabla u \cdot \nabla v\d x, \quad \ell\parentheses*{v} = \int_\Omega fv\d x.
	\]
	Next we need to check the requirements of the Lax-Milgram theorem:
	\begin{enumerate}
		\item \(H_0^1\parentheses*{\Omega}\) is a Hilbert space and we use its norm \(\norm*{\cdot}_1\) and the corresponding scalar product \(\angles*{\cdot, \cdot}_1\).
		\item \(a\parentheses*{\cdot, \cdot}\) is a bilinear form.
		\item \(\ell\parentheses*{\cdot}\) is a linear functional.
		\item \(a\parentheses*{\cdot, \cdot}\) is continuous, because
		\[
			\absolute*{\int_\Omega \nabla u \cdot \nabla v\d x} \le \norm*{u}_1 \norm*{v}_1
		\]
		see Theorem \ref{theorem:3-3}.
		\item \(\ell\parentheses*{\cdot}\) is continuous, because (again with the Cauchy-Schwartz inequality) provided that \(f \in L^2\parentheses*{\Omega}\)
		\[
			\absolute*{\int_\Omega fv\d x} \le \norm*{v}_0 \norm*{f}_0 \le C\norm*{v}_1
		\]
		with \(C = \norm*{f}_0 < \infty\).
		\item However, it is not immediately clear why \(a\) should be coercive, that is, why
		\[
			\absolute*{\int_\Omega \nabla u \cdot \nabla v\d x} \stackrel{?}{\ge} \gamma\norm*{u}_1^2 \quad \forall u \in H_0^1\parentheses*{\Omega}.
		\]
		Formally, this inequality is not satisfied for constant functions (\(\ne 0\)).
	\end{enumerate}
\end{example}

\begin{theorem}
	Let \(\Omega \subset \R^n\) be bounded.
	There is a constant \(C\) depending only on \(\Omega\), such that
	\[
		\norm*{u}_{L^2\parentheses*{\Omega}} \le C\norm*{\nabla u}_{L^2\parentheses*{\Omega}}
	\]
	for all \(u \in H^1\parentheses*{\Omega}\) that vanish on the boundary \(\partial\Omega\), that is, for all \(u \in H_0^1\parentheses*{\Omega}\).
\end{theorem}

\begin{proof}
	We show this property for mooth functions \(u \in C_0^\infty\parentheses*{\Omega}\), because \(C_0^\infty\parentheses*{\Omega}\) is dense in \(H_0^1\parentheses*{\Omega}\) so that any \(H_0^1\parentheses*{\Omega}\) function can be approximated by those.
	We restrict ourselves to \(n = 1\) (\(\nabla u = u'\)) and \(\Omega = \parentheses*{a, b}\).
	First, consider any \(x \in \Omega\) and look at
	\[
		\absolute*{u\parentheses*{x}}^2 = \absolute*{u\parentheses*{x} - u\parentheses*{a}}^2 = \absolute*{\int_a^x u'\parentheses*{\xi}\d\xi}^2 = \absolute*{\int_a^x u'\parentheses*{\xi}\d\xi}^2 \le \int_a^x \d\xi\int_a^x \absolute*{u'\parentheses*{\xi}}^2 \d\xi.
	\]
	The first integral can be computed and gives \(x - a \le b - a\), and the second integral can be estimated by the integral from \(a\) to \(b\), hence
	\[
		\absolute*{u\parentheses*{x}}^2 \le \parentheses*{b - a}\int_a^b \absolute*{u'\parentheses*{x}}^2 \d\xi = \parentheses*{b - a}\norm*{u'}_{L^2\parentheses*{\Omega}}^2.
	\]
	Both sides are positive and can be integrated again
	\[
		\norm*{u}_{L^2\parentheses*{\Omega}}^2 = \int_a^b \absolute*{u\parentheses*{x}}^2 \d x \le \int_a^b \parentheses*{b - a}\norm*{u'}_{L^2\parentheses*{\Omega}}^2 \d x = \parentheses*{b - a}^2 \norm*{u'}_{L^2\parentheses*{\Omega}}^2
	\]
	where we identify \(C = \parentheses*{b - a}^2\).
	For \(n > 1\) one would proceed coordinate-wise, using \(u\parentheses*{x} = u\parentheses*{y} + \sum_{i = 1}^n \int_{y_i}^{x_i}\partial_{x_i} u\parentheses*{x_1, \ldots, x_{i - 1}, z, x_{i + 1}, \ldots, x_n}\d z\) for \(x = \parentheses*{x_1, \ldots, x_n}^T, y = \parentheses*{y_1, \ldots, y_n}^T \in \R^n\).
\end{proof}

\begin{example}
	To show the coercivity of \(a\) we add \(\norm*{\nabla u}_0^2\) to both sides of the square of the Poincare inequality to find
	\[
		\norm*{u}_0^2 + \norm*{\nabla u}_0^2 \le C^2 \norm*{\nabla u}_0^2 + \norm*{\nabla u}_0^2 \iff \norm*{u}_1^2 \le \parentheses*{1 + C^2}\norm*{\nabla u}_0^2 \equiv \tilde{C}\int_\Omega \nabla u \cdot \nabla u\d x.
	\]
	Reading backwards we obtain as desired
	\[
		a\parentheses*{u, u} = \int_\Omega \nabla u \cdot \nabla u\d x \ge \frac{1}{\tilde{C}}\norm*{u}_1^2.
	\]
	Hence, the variational or weak formulation of the Poisson problem
	\begin{align*}
		-\Delta u = 0, \quad \text{in }\Omega,\\
		u &= f, \quad \text{on }\partial\Omega
	\end{align*}
	has a unique weak solution \(u \in H_0^1\parentheses*{\Omega}\) for any \(f \in L^2\parentheses*{\Omega}\).
\end{example}


\subsection{Regularity}

Next, we assess the regularity of the weak solution for the Poisson problem.

\begin{theorem}
	Let \(\Omega \subset \R^n\) be a domain and \(u \in H_0^1\parentheses*{\Omega}\) the weak solution of the Poisson problem
	\begin{align*}
		-\Delta u &= f, \quad \text{in }\Omega,\\
		u &= 0, \quad \text{on }\partial\Omega,
	\end{align*}
	for \(f \in H^k\parentheses*{\Omega}, k \in \N_0\).
	\begin{enumerate}
		\item\label{theorem:3-11:1} If \(\partial\Omega\) is a smooth boundary with a parametrization in \(C^{k + 2}\parentheses*{\Omega}\), then \(u \in H^{k + 2}\parentheses*{\Omega}\) and \(\norm*{u}_{k + 2} \le C\norm*{f}_k\).
		\item\label{theorem:3-11:2} If \(\Omega\) is a (Lipschitz) domain and \(G \subset \Omega\) a strictly contained subdomain with \(\bar{G} \subset \Omega\), then \(u \in H^{k + 2}\parentheses*{G}\) and \(\norm*{u}_{k + 2, G} \le C\norm*{f}_{k, \Omega}\), with \(C = C\parentheses*{G}\).
		\item If \(\Omega\) is a convex Lipschitz domain, then \(u \in H^2\parentheses*{\Omega}\) for all \(f \in L^2\parentheses*{\Omega}\).
	\end{enumerate}
\end{theorem}

\begin{proof}
	In any decent book on elliptic PDEs, e.g., the book by Evans.
\end{proof}

\begin{remark}
	\begin{enumerate}
		\item The case \ref{theorem:3-11:1} describes a property that we already know from classical solutions:
		The solution has two orders of ``differentiability'' more than the right-hand side \(f\).
		Notice, however, that these are weak derivatives here.
		Results of this type, where a certain regularity of \(f\) yields a higher regularity in \(u\), are called \emph{shift theorems}.
		\item The case \ref{theorem:3-11:2} shows that full regularity as in \ref{theorem:3-11:1} can only be expected in some interior \(G\) of the domain \(\Omega\), so that the smoothness of \(\partial\Omega\) plays no role.
		\item If the domain has reentrant corners, it is possible to have only \(u \in H^{1 + \alpha}\parentheses*{\Omega}\) with \(0 < \alpha < 1\); c.f. Pacman domain.
		\item Question:
		For case \ref{theorem:3-11:1}, is the solution \(u\) smooth enough to be a classical solution, i.e., to be in \(C^2\parentheses*{\Omega}\)?
		The answer is given by the Sobolev embedding theorem \ref{theorem:2-12}.
		For example, for \(n = 2\) we find
		\[
			f \in H^2\parentheses*{\Omega} \implies u \in H^4\parentheses*{\Omega} \implies u \in C^2\parentheses*{\Omega},
		\]
		because \(4 - \frac{n}{2} > 2\).
		\item Similar regularity results can be proved for weak solutions of more general elliptic PDEs.
	\end{enumerate}
\end{remark}


\subsection{Neumann problem}

We consider the Poisson problem with inhomogeneous Neumann boundary conditions
\begin{align*}
    -\Delta u &= f, \quad \text{in }\Omega,\\
    \partial_n u &= g_N, \quad \text{on }\partial\Omega.
\end{align*}
As before we construct the variational formulation by multiplying with the test function \(v\), integrating over \(\Omega\) and using partial integration
\[
    -\int_\Omega v \cdot \Delta u\d x = \int_\Omega v \cdot f\d x \iff \int_\Omega \nabla u \cdot \nabla v\d x - \oint_{\partial\Omega}\partial_n uv\d A = \int_\Omega v \cdot f\d x,
\]
where we now use the boundary conditions for \(\partial_n u\) directly in the boundary integral.
We use \(H^1\parentheses*{\Omega}\) as the approach-space for \(u\) and test-space for \(v\).
Find \(u \in H^1\parentheses*{\Omega}\) such that
\[
    \int_\Omega \nabla u \cdot \nabla v\d x = \int_\Omega v \cdot f\d x + \oint_{\partial\Omega}vg_N\d A \quad \forall v \in H^1\parentheses*{\Omega}.
\]

\begin{remark}
    \begin{enumerate}
        \item Inspection shows that the solution is only determined up to a constant, i.e., if \(u\) is a solution, then \(u + c\) with \(c \in \R\) is too.
        A single condition is needed to eliminate this non-uniqueness.
        One approach is the requirement of zero mean
        \[
            \int_\Omega u\d x = 0.
        \]
        Analogously, to the Dirichlet case we build this requirement into our function space and define
        \[
            \tilde{H}^1\parentheses*{\Omega} = \braces*{u \in H^1\parentheses*{\Omega} : \int_\Omega u\d x = 0}.
        \]
        Consequently, we ask for \(u, v \in \tilde{H}^1\parentheses*{\Omega}\) in the variational formulation of the Neumann-Poisson problem.
        \item The bilinear form \(a\parentheses*{u, v}\) is coercive on \(\tilde{H}^1\parentheses*{\Omega}\) (second Poincare inequality) and according to Lax-Milgram the Neumann-Poisson problem has a unique solution.
        \item Inhomogeneous Dirichlet conditions \(u = g_D\) on the boundary \(\partial\Omega\) can be ``weakly'' imposed by writing them as Robin-type conditions \(-\delta\partial_n u = u - g_D\) with a small artificial parameter \(\delta \ll 1\).
        The variational formulation then reads:
        \begin{quote}
            ``Find \(u \in H^1\parentheses*{\Omega}\) such that
            \[
                \int_\Omega \nabla u \cdot \nabla v\d x + \delta^{-1}\oint_{\partial\Omega}uv\d A = \int_\Omega v \cdot f\d x + \delta^{-1}\oint_{\partial\Omega}vg_N\d x
            \]
            for all \(v \in H^1\parentheses*{\Omega}\).''
        \end{quote}
    \end{enumerate}
\end{remark}


\subsection{Minimization problem}

\begin{theorem}
    Like in Lax-Milgram, let \(V\) be a Hilbert space, \(a: V \times V \to \R\) a continuous, coercive, symmetric bilinear form, and \(\ell: V \to \R\) a continuous, linear functional.
    Then we have the equivalence
    \[
        u \in V\text{ minimizes }J\parentheses*{v} := \frac{1}{2}a\parentheses*{v, v} - \ell\parentheses*{v} \iff a\parentheses*{v, u} = \ell\parentheses*{v}\text{ for all }v \in V.
    \]
\end{theorem}

\begin{proof}
    Let \(u \in V\) be the minimum of \(J: V \to \R, v \mapsto \frac{1}{2}a\parentheses*{v, v} - \ell\parentheses*{v}\).
    We choose a so-called variation \(v \in V\) and scalar parameter \(t \in \R, t > 0\).
    Obviously, \(J\parentheses*{u + tv} \ge J\parentheses*{u}\), because \(J\parentheses*{u}\) is the minimal value.
    Hence,
    \begin{align*}
        0 &\le J\parentheses*{u + tv} - J\parentheses*{u}\\
        &= \frac{1}{2}a\parentheses*{u + tv, u + tv} - \ell\parentheses*{u + tv} - \parentheses*{\frac{1}{2}a\parentheses*{u, u} - \ell\parentheses*{u}}\\
        &= t\parentheses*{a\parentheses*{v, u} - \ell\parentheses*{v}} + t^2 \frac{1}{2}a\parentheses*{v, v}
    \end{align*}
    due to linearity of \(a\) and \(\ell\).
    Because \(t > 0\) and \(a\) is coercive we find
    \[
        a\parentheses*{v, u} - \ell\parentheses*{v} \le 0.
    \]
    However, the same calculation holds for \(t < 0\) which gives
    \[
        a\parentheses*{v, u} - \ell\parentheses*{v} \ge 0,
    \]
    such that \(a\parentheses*{u, v} = \ell\parentheses*{v}\) for any variation \(v\).
    If, conversely, \(u \in V\) is the solution of \(a\parentheses*{v, u} = \ell\parentheses*{v}\) for all \(v \in V\), we compute
    \begin{align*}
        J\parentheses*{u + v} - J\parentheses*{u} &= \frac{1}{2}a\parentheses*{u + v, u + v} - \ell\parentheses*{u + v} - \parentheses*{\frac{1}{2}a\parentheses*{u, u} - \ell\parentheses*{u}}\\
        &= a\parentheses*{u, v} - \ell\parentheses*{v} + \frac{1}{2}a\parentheses*{v, v}\\
        &= \frac{1}{2}a\parentheses*{v, v} \ge 0.
    \end{align*}
    Thus, \(J\parentheses*{u + v} \ge J\parentheses*{u}\) for all \(v\) and \(u\) must be a minimizer of \(J\).
\end{proof}

\begin{remark}
    \begin{enumerate}
        \item With this insight the notation ``variational formulation'' makes more sense, because we vary the solution \(u\) with a variation \(v\) (see variational calculus).
        \item In the finite dimensional case \(V = \R^m\) we have \(a\parentheses*{v, u} = v^T Au\) with matrix \(A \in \R^{m \times m}\) and \(\ell\parentheses*{v} = v^T f\) with some fixed vector \(f \in \R^m\).
        The variational formulation reads
        \[
            v^T Au = v^T f \quad \forall v \in \R^m,
        \]
        which implies \(Au = f\) and the solution \(u = A^{-1}f\).
        This solution minimizes the quadratic form
        \[
            J\parentheses*{v} = \frac{1}{2}v^T Av - v^T f.
        \]
    \end{enumerate}
\end{remark}
