\section{Theory of Weak Solutions}

The classical solution theory of the Poisson problem \(-\Delta u = f\) in \(\Omega\) with boundary conditions \(u = 0\) on \(\partial\Omega\) requires two classical derivatives.
That is, to at least make sense of the Laplace operator \(\Delta\), we require \(u \in C^2\parentheses*{\Omega}\).
We have seen, that this might be too restrictive in general.
To remedy this shortcoming we need to reformulate the problem statement and allow for a ``weaker'' solution concept.


\subsection{Variational Formulation}

We multiply the Poisson problem by an arbitrary, sufficiently smooth test function \(v \in V\) defined on \(\Omega\) and integrate over \(\Omega\):
\[
	-\int_\Omega v \cdot \Delta u\d x = \int_\Omega v \cdot f\d x.
\]
Partial integration for \(\Delta u = \div\parentheses*{\grad u}\) yields
\begin{align*}
	-\int_\Omega v \cdot \Delta u\d x &= -\int_\Omega \parentheses*{\div\parentheses*{v\grad u} - \grad u \cdot \grad v}\d x\\
	&= -\int_\Omega \div\parentheses*{v\grad u}\d x + \int_\Omega \grad u \cdot \grad v\d x\\
	&= -\int_{\partial\Omega}vn \cdot \grad u\d s + \int_\Omega \grad u \cdot \grad v\d x
\end{align*}
where we used Gauss' theorem, denoting the outward pointing normal by \(n\) on the domain boundary \(\partial\Omega\).
Rearranging terms, we arrive at
\[
	\int_\Omega \grad u \cdot \grad v\d x - \int_{\partial\Omega}vn \cdot \grad u\d s = \int_\Omega v \cdot f\d x.
\]
We observe that the boundary integral is ready to accommodate Neumann boundary conditions (comes later).
Moreover, homogeneous Dirichlet conditions \(u = 0\) on \(\partial\Omega\) are commonly ``built'' into the function space, that is, we simply require the test functions to satisfy \(\left.v\right|_{\partial\Omega} = 0\), so that the boundary integral vanishes.

\begin{definition}
	The problem statement
	\begin{quote}
		``Find the function \(u: \Omega \to \R\) with \(\left.u\right|_{\partial\Omega} = 0\) such that the relation
		\[
			\int_\Omega \nabla u \cdot \nabla v\d x = \int_\Omega v \cdot f\d x
		\]
		is satisfied for all test functions \(v \in V\).''
	\end{quote}
	is called the \emph{weak} or \emph{variational formulation} of the PDE problem
	\[
		-\Delta u = f\text{ in }\Omega, \quad \text{and} \quad u = 0\text{ on }\partial\Omega.
	\]
	Its solution is called \emph{weak solution}.
\end{definition}

\begin{remark}
	\begin{enumerate}
		\item In, e.g., 2D, we have \(\nabla u \cdot \nabla v = \grad u \cdot \grad v = \partial_x u \partial_x v + \partial_y u\partial_y v\) for the scalar product.
		\item Every classical, 
	\end{enumerate}
\end{remark}