\section{Finite element method}

The goal of the finite element method (FEM) is to find a simple space \(V_N\) such that the resulting matrix \(A\) is sparsely occupied.
This is, for example, the case if the test functions \(\braces*{\varphi_i}_{i = 1, 2, \ldots}\) overlap with each other in a minimal way.


\subsection{Hat-functions and FEM in 1D}

For an interval \(\Omega = \brackets*{a, b}\) we consider the positions \(\braces*{x_i}_{i = 0, \ldots, N + 1} \subset \brackets*{a, b}\) with \(a = x_0 < x_i < x_{N + 1} = b\) and step size \(h_j = x_j - x_{j - 1}\) and define the ``hat-functions'' \(\varphi_j \in H^1\parentheses*{\brackets*{a, b}}\) by
\[
    \varphi_j\parentheses*{x} = \begin{cases}
        \frac{x - x_{j - 1}}{h_j}, & \text{if }x_{j - 1} \le x \le x_j,\\
        \frac{x_{j + 1} - x}{h_{j + 1}}, & \text{if }x_j \le x \le x_{j + 1},\\
        0, & \text{otherwise}.
    \end{cases}
\]

\begin{theorem}
    Let \(u \in H^1\parentheses*{\Omega}\) with \(\Omega = \parentheses*{a, b}\) and the interior positions \(\braces*{x_i}_{i = 1, \ldots, N} \subset \Omega\) be given.
    The function
    \[
        s^{\parentheses*{N}}\parentheses*{x} = \sum_{j = 1}^N u\parentheses*{x_j}\varphi_j\parentheses*{x},
    \]
    with hat-functions \(\varphi_j\), satisfies \(s^{\parentheses*{N}} \in H_0^1\parentheses*{\Omega}\) and is a linear spline interpolating \(u\parentheses*{x_k}\) for \(k = 1, \ldots, N\).
\end{theorem}

\begin{proof}
    The hat-functions are continuous piecewise linear functions, hence \(\varphi_j \in H^1\parentheses*{\brackets*{a, b}}\).
    They have the property \(\varphi_j\parentheses*{x_k} = 1\) if \(j = k\) and \(\varphi_j\parentheses*{x_k} = 0\) if \(j \ne k\), i.e., \(\varphi_j\parentheses*{x_k} = \delta_{jk}\).
    Hence, \(s^{\parentheses*{N}}\) interpolates \(u\) because for \(k = 1, \ldots, N\)
    \[
        s^{\parentheses*{N}}\parentheses*{x_k} = \sum_{j * 1}^N u\parentheses*{x_j}\varphi_j\parentheses*{x_k} = u\parentheses*{x_k}.
    \]
    By definition a linear spline is a continuous piecewise linear function.
    On each interval \(\parentheses*{x_j, x_{j + 1}}\) the function \(s^{\parentheses*{N}}\) is a superposition of two hat-functions \(\varphi_j\) and \(\varphi_{j + 1}\), which gives a linear function connecting \(u\parentheses*{x_j}\) and \(u\parentheses*{x_{j + 1}}\).
    Because the function values coincide on each side of the intervals \(s^{\parentheses*{N}}\) is continuous.
    At the boundaries \(x_0 = a\) and \(x_{N + 1} = b\) we find
    \[
        s^{\parentheses*{N}}\parentheses*{x_0} = u\parentheses*{x_1}\varphi_1\parentheses*{x_0} = 0, \quad s^{\parentheses*{N}}\parentheses*{x_{N + 1}} = u\parentheses*{x_N}\varphi_N\parentheses*{x_{N + 1}} = 0.
    \]
    Alltogether, \(s^{\parentheses*{N}} \in H_0^1\parentheses*{\brackets*{a, b}}\).
\end{proof}

For a fixed grid we choose the linear splines as approximation space \(V_N \in H_0^1\parentheses*{\brackets*{a, b}}\) and the hat-functions \(\braces*{\varphi_j}_{j = 1, \ldots, N}\) as basis.
For the Ritz-Galerkin solution of the Poisson problem we compute
\[
    a\parentheses*{\varphi_j, \varphi_k} = \begin{cases}
        \frac{2}{h}, & \text{if }j = k,\\
        -\frac{1}{h}, & \text{if }j = k + 1\text{ or }j = k - 1,\\
        0, & \text{otherwise}
    \end{cases}
\]
for an equidistant grid with \(h_j = h = \text{const.}\) and the resulting matrix of the linear system reads
\[
    A = \frac{1}{h}\begin{pmatrix}
        2 & -1 & 0 & \cdots & 0\\
        -1 & 2 & \ddots & \ddots & \vdots\\
        0 & \ddots & \ddots & \ddots & 0\\
        \vdots & \ddots & \ddots & 2 & -1\\
        0 & \cdots & 0 & -1 & 2
    \end{pmatrix}.
\]

\begin{remark}
    \begin{enumerate}
        \item The matrix is equivalent to the finite difference case.
        This is a coincidence.
        \item As above the solution of the Poisson problem has the form \(u_N = \sum_{k = 1}^N \alpha_k \varphi_k\) with \(\varphi_k\) being hat-functions.
        In this case, the basis coefficients \(\alpha_k\) approximate the point values of \(u\) at \(x_k\) such that \(\alpha_k = u_N\parentheses*{x_k} \approx u\parentheses*{x_k}\).
    \end{enumerate}
\end{remark}


\subsection{Triangulation in 2D}

After the basic motivation in 1D, we can now move on to some simple examples in 2D.
First we will ned the following concepts for a spatial decomposition.

\begin{definition}
    A \emph{triangulation} \(\mathcal{T} = \braces*{K_i}_{i = 1, 2, \ldots}\) of a polygonal domain \(\Omega \subset \R^2\) is a decomposition of \(K_i, i = 1, 2, \ldots\).
    A triangulation \(\mathcal{T}\) is called \emph{regular}, if the intersection of two triangles \(K, K' \in \mathcal{T}\) is either empty, a corner (vertex/node), or an entire edge.
    The \emph{set of vertices} of \(\mathcal{T}\) is denoted by
    \[
        \mathcal{N}\parentheses*{\mathcal{T}} = \braces*{p \in \R^2 : p\text{ is a vertex of some }K \in \mathcal{T}}.
    \]
\end{definition}

\begin{remark}
    \begin{enumerate}
        \item Here we focus on the subdomains \(K_i\) being triangles (i.e., simplices), but one could, in principle, also combine triangle and quadrilaterals, as long as \(\bar{\Omega} = \bigcup_{i = 1}^M \bar{K_i}\) for some \(M \in \N\) with \(K_i \cap K_j = \emptyset\) for \(i \ne j\).
        \item Regular means that every vertex in \(K\) is also a corner of \(K\).
        That is, there are no hanging nodes.
        \item On a computer \(\mathcal{T}\) is typically stored in two parts: a list of coordinates for the vertices \(p_j \in \mathcal{N}\parentheses*{\mathcal{T}}, j = 1, 2, \ldots\) and a connectivity list consisting of the three vertex numbers \(\parentheses*{j_1^{\parentheses*{K}}, j_2^{\parentheses*{K}}, j_3^{\parentheses*{K}}} \in \N^3\) (with respect to a global numbering) for each triangle \(K \in \mathcal{T}\).
    \end{enumerate}
\end{remark}

\begin{example}
    Let's consider the following simple triangulation:
    \begin{figure}
        \centering
        \begin{tikzpicture}
            \draw[<->] (-.5,2.5) node[right] {\(y\)} -- (-.5, -.5) -- (4.5,-.5) node[above] {\(x\)};
        \end{tikzpicture}
        \caption{3 triangles build from 5 vertices}
        \label{fig:5-1}
    \end{figure}
\end{example}
