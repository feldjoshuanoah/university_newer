\documentclass[english]{exercise}

\DeclareMathOperator*{\cond}{cond}
\DeclareMathOperator*{\diag}{diag}
\DeclareMathOperator*{\dist}{dist}
\DeclareMathOperator*{\esssup}{ess\,sup}
\DeclareMathOperator*{\vol}{vol}


\title{Homework 4}
\author{Joshua Feld, 406718}
\professor{Prof. Kowalski}
\course{Partial Differential Equations}

\begin{document}
	\maketitle


	\section{}

	\begin{quote}
		Show that the Poincare's inequality
		\[
			\int_\Omega \absolute*{u}^2 \d x \le c\parentheses*{\Omega}\int_\Omega \absolute*{\nabla u}^2 \d x, \quad c\parentheses*{\Omega} > 0
		\]
		holds on the space \(W\parentheses*{\Omega}\) defined by
		\[
			W\parentheses*{\Omega} = \braces*{u \in H^1\parentheses*{\Omega} : \int_\Omega u\d x = 0}
		\]
		where \(\Omega = \parentheses*{a, b} \subset \R\) is a bounded domain.
	\end{quote}

	Let \(u\parentheses*{x}\) belong to the space \(W\parentheses*{\Omega}\), defined by
	\[
		W\parentheses*{\Omega} := \braces*{u \in H^1\parentheses*{\Omega} : \int_\Omega u\d x = 0}.
	\]
	From the fundamental theorem of calculus, we have
	\[
		u\parentheses*{x} - u\parentheses*{a} = \int_a^x u'\parentheses*{s}\d s.
	\]
	From the Schwartz inequality, the above equation reduces to
	\[
		u\parentheses*{x} - u\parentheses*{a} \le \sqrt{\int_a^x \absolute*{1}^2 \d s}\sqrt{\int_a^x \absolute*{u'\parentheses*{s}}^2 \d s} = \sqrt{x - a}\sqrt{\int_a^x \absolute*{u'\parentheses*{s}}^2 \d s}.
	\]
	Taking the square on both sides, we get
	\[
		\parentheses*{u\parentheses*{x} - u\parentheses*{a}}^2 \le \parentheses*{x - a}\int_a^x \absolute*{u'\parentheses*{s}}^2 \d s \le \parentheses*{b - a}\int_a^b \absolute*{u'\parentheses*{s}}^2 \d s
	\]
	Integrating on both sides over the domain \(\Omega = \parentheses*{a, b}\) gives
	\[
		\int_a^b \parentheses*{u\parentheses*{x} - u\parentheses*{a}}^2 \d x \le \parentheses*{b - a}\int_a^b \absolute*{u'\parentheses*{s}}^2 \d s\int_a^b \d x.
	\]
	It implies that
	\[
		\int_a^b u^2\parentheses*{x}\d x - 2u\parentheses*{a}\int_a^b u\parentheses*{x}\d x + \int_a^b u^2\parentheses*{x}\d x \le \parentheses*{b - a}^2 \int_a^b \absolute*{u'\parentheses*{s}}^2 \d s.
	\]
	Since \(u\parentheses*{x} \in W\parentheses*{\Omega}\), the second term in the left-hand side of the above equation becomes zero, while the term is a positive constant.
	The above equation can then be casted as
	\[
		\int_a^b \absolute*{u\parentheses*{x}}^2 \d x \le c\parentheses*{\Omega}\int_a^b \absolute*{u'\parentheses*{x}}^2 \d x,
	\]
	where \(c\parentheses*{\Omega} = \parentheses*{b - a}^2\).


	\section{}

	\begin{quote}
		Consider the boundary value problem
		\begin{align*}
			-\Delta u\parentheses*{x, y} &= 1, \quad \text{in }\Omega = \parentheses*{0, 1}^2 \subset \R^2,\\
			u\parentheses*{x, y} &= 0, \quad \text{on }\partial\Omega.
		\end{align*}
		Choose the basis functions as
		\[
			\psi_{i, j}\parentheses*{x, y} = \sin\parentheses*{i\pi x}\sin\parentheses*{j\pi y}, \quad 1 \le i, j \le N.
		\]
		\begin{enumerate}
			\item Derive the Ritz-Galerkin equations and find the stiffness matrix.
			\item Let the basis functions \(\psi_{1, 1}\), \(\psi_{1, 3}\), \(\psi_{3, 1}\) and \(\psi_{3, 3}\) form a space \(V\).
			Find the approximate solution in the space \(V\).
		\end{enumerate}
	\end{quote}

	\begin{enumerate}
		\item Multiplying the given differential equation by a test function \(v\parentheses*{x, y} \in H_0^1\parentheses*{\Omega}\) and integrating over the domain \(\Omega = \parentheses*{0, 1}^2\), we get
		\[
			-\int_\Omega v\Delta u\d\Omega = \int_\Omega v\d\Omega \iff \int_\Omega \nabla u \cdot \nabla v\d \Omega = \int_\Omega v\d\Omega.
		\]
		The weak formulation of the given boundary value poribem is then given as follows:
		Find \(u\parentheses*{x, y} \in H_0^1\parentheses*{\Omega}\) such that
		\[
			a\parentheses*{u, v} = \ell\parentheses*{v} \quad \forall v \in H_0^1\parentheses*{\Omega}
		\]
		where
		\[
			a\parentheses*{u, v} = \int_\Omega \nabla u \cdot \nabla v\d\Omega, \quad \text{and} \quad \ell\parentheses*{v} = \int_\Omega v\d\Omega.
		\]
		Let us derive the discrete weak formulation.
		Let \(u_h\parentheses*{x, y} \in H_h \subset H_0^1\parentheses*{\Omega}\) be a discrete approximation of the weak solution \(u\parentheses*{x, y}\).
		Let \(\psi_{i, j}\parentheses*{x, y}, i, j = 1, \ldots, N\) be the basis functions of the space \(H_h\).
		The discrete approximate solution is then given by
		\[
			u_h = \sum_{i, j = 1}^N a_{i, j}\psi_{i, j}\parentheses*{x, y}.
		\]
		The discrete weak formulation of the given problem can be given as
		\begin{align*}
			a\parentheses*{u_h, v} &= \ell\parentheses*{v} \quad \forall v \in H_0^1\parentheses*{\Omega},\\
			\iff a\parentheses*{u_h, v_h} &= \ell\parentheses*{v_h} \quad \forall v_h \in H_h \subset H_0^1\parentheses*{\Omega},\\
			\iff a\parentheses*{u_h, \psi_{k, l}} &= \ell\parentheses*{\psi_{k, l}} \quad \forall k, l = 1, \ldots, N,\\
			\iff \sum_{i, j = 1}^N a\parentheses*{\psi_{i, j}, \psi_{k, l}}\alpha_{i, j} &= \ell\parentheses*{\psi_{k, l}} \quad \forall k, l = 1, \ldots, N.
		\end{align*}
		In the matrix form, the above system of equations can be written as
		\[
			Ac = b,
		\]
		where
		\begin{align*}
			A &= A_{\parentheses*{k, l}, \parentheses*{i, j}} = a\parentheses*{\psi_{i, j}, \psi_{k, l}} = \int_\Omega \nabla\psi_{i, j} \cdot \nabla\psi_{k, l}\d\Omega,\\
			c &= c_{i, j} = \alpha_{i, j},\\
			b &= b_{k, l} = \int_\Omega \psi_{k, l}\d\Omega.
		\end{align*}
		Using the basis functions \(\psi_{i, j}\parentheses*{x, y} = \sin\parentheses*{i\pi x}\sin\parentheses*{j\pi y}\), the stiffness matrix in the Ritz-Galerkin equations can be given as
		\[
			A = A_{\parentheses*{k, l}, \parentheses*{i, j}} = \int_\Omega \nabla\psi_{i, j} \cdot \nabla\psi_{k, l}\d\Omega = \begin{cases}
				\frac{\parentheses*{i^2 + j^2}\pi^2}{4}, & \text{if }\parentheses*{i, j} = \parentheses*{k, l},\\
				0, & \text{otherwise}
			\end{cases}
		\]
		and the vector \(b\) is given by
		\[
			b = b_{k, l} = b\parentheses*{\psi_{k, l}} = \int_\Omega \psi_{k, l}\d\Omega = \int_\Omega \sin\parentheses*{k\pi x}\sin\parentheses*{l\pi y}\d x\d y.
		\]
		Such an integral is zero if one of \(k\) and \(l\) is even.
		In the odd case we have
		\[
			b_{k, l} = \frac{4}{kl\pi^2}.
		\]
		\item We choose the basis functions as \(\psi_{i, j}, i, j = 1, 3\).
		That is the four basis functions are given by \(\psi_{1, 1}\), \(\psi_{1, 3}\), \(\psi_{3, 1}\) and \(\psi_{3, 3}\).
		The Ritz-Galerkin equations can then be written as
		\begin{align*}
			a\parentheses*{\psi_{1, 1}, \psi_{1, 1}}c_{1, 1} + a\parentheses*{\psi_{1, 3}, \psi_{1, 1}}c_{1, 3} + a\parentheses*{\psi_{3, 1}, \psi_{1, 1}}c_{3, 1} + a\parentheses*{\psi_{3, 3}, \psi_{1, 1}}c_{3, 3} &= b\parentheses*{\psi_{1, 1}},\\
			a\parentheses*{\psi_{1, 1}, \psi_{1, 3}}c_{1, 1} + a\parentheses*{\psi_{1, 3}, \psi_{1, 3}}c_{1, 3} + a\parentheses*{\psi_{3, 1}, \psi_{1, 3}}c_{3, 1} + a\parentheses*{\psi_{3, 3}, \psi_{1, 3}}c_{3, 3} &= b\parentheses*{\psi_{1, 3}},\\
			a\parentheses*{\psi_{1, 1}, \psi_{3, 1}}c_{1, 1} + a\parentheses*{\psi_{1, 3}, \psi_{3, 1}}c_{1, 3} + a\parentheses*{\psi_{3, 1}, \psi_{3, 1}}c_{3, 1} + a\parentheses*{\psi_{3, 3}, \psi_{3, 1}}c_{3, 3} &= b\parentheses*{\psi_{3, 1}},\\
			a\parentheses*{\psi_{1, 1}, \psi_{3, 3}}c_{1, 1} + a\parentheses*{\psi_{1, 3}, \psi_{3, 3}}c_{1, 3} + a\parentheses*{\psi_{3, 1}, \psi_{3, 3}}c_{3, 1} + a\parentheses*{\psi_{3, 3}, \psi_{3, 3}}c_{3, 3} &= b\parentheses*{\psi_{3, 3}}.
		\end{align*}
		The diagonal terms \(a\parentheses*{\psi_{i, j}, \psi_{k, l}}, \parentheses*{i, j} = \parentheses*{k, l}\) are given by
		\[
			a\parentheses*{\psi_{1, 1}, \psi_{1, 1}} = \frac{\pi^2}{2}, \quad a\parentheses*{\psi_{1, 3}, \psi_{1, 3}} = \frac{5\pi^2}{2}, \quad a\parentheses*{\psi_{3, 1}, \psi_{3, 1}} = \frac{5\pi^2}{2}, \quad a\parentheses*{\psi_{3, 3}, \psi_{3, 3}} = \frac{9\pi^2}{2},
		\]
		while the values of off-diagonal terms are \(a\parentheses*{\psi_{i, j}, \psi_{k, l}} = 0, \parentheses*{i, j} \ne \parentheses*{k, l}\).
		Solving the above system of equations, we get
		\[
			c_{1, 1} = \frac{8}{\pi^4}, \quad c_{1, 3} = c_{3, 1} = \frac{8}{15\pi^4}, \quad c_{3, 3} = \frac{8}{81\pi^4}.
		\]
		The discrete solution is given by
		\[
			u_h\parentheses*{x, y} = \psi_{1, 1}c_{1, 1} + \psi_{1, 3}c_{1, 3} + \psi_{3, 1}c_{3, 1} + \psi_{3, 3}c_{3, 3}.
		\]
	\end{enumerate}
\end{document}