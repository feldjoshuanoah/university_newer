\documentclass[english]{exercise}

\DeclareMathOperator*{\cond}{cond}
\DeclareMathOperator*{\diag}{diag}
\DeclareMathOperator*{\dist}{dist}
\DeclareMathOperator*{\esssup}{ess\,sup}
\DeclareMathOperator*{\vol}{vol}


\title{Self Exercise 1}
\author{Joshua Feld, 406718}
\professor{Prof. Kowalski}
\course{Partial Differential Equations}

\begin{document}
	\maketitle


	\section{}

	\begin{enumerate}
		\item The problem does not have a unique solution:
		If \(u\) is a solution to the problem and \(c\) is a constant, then \(u + c\) is also a solution to the problem, because
		\[
			-\Delta\parentheses*{u + c} = -\Delta u = f \quad \text{and} \quad \frac{\partial\parentheses*{u + c}}{\partial n} = \frac{\partial u}{\partial n} = g.
		\]
		\item We define \(w := \nabla u\).
		This transforms the problem into
		\begin{align*}
			-\div w &= f, \quad \text{in }\Omega,\\
			w \cdot n &= g, \quad \text{on }\partial\Omega
		\end{align*}
		and the theorem of Gauss gives
		\[
			\int_\Omega \div w\d x = \int_{\partial\Omega}w \cdot n\d s.
		\]
		Thus
		\[
			\int_\Omega f\d x + \int_{\partial\Omega}g\d s = 0
		\]
		has to be fulfilled.
		However, this condition is only necessary if a Neumann condition is set on the complete boundary.
		If a Dirichlet boundary condition is set on a part of the boundary, this condition is no longer necessary.
	\end{enumerate}


	\section{}

	\begin{enumerate}
		\item
		\[
			\int_0^\infty \absolute*{f\parentheses*{x}}\d x = \int_0^1 f\parentheses*{x}\d x + \int_1^\infty f\parentheses*{x}\d x = 1 + \lim_{x \to \infty}3x^{\frac{1}{3}} = \infty.
		\]
		Therefore \(f \not\in L^1\parentheses*{\Omega}\).
		\[
			\int_0^\infty \absolute*{f\parentheses*{x}}^2 \d x = \int_0^1 f^2\parentheses*{x}\d x + \int_1^\infty f^2\parentheses*{x}\d x = 1 + \int_1^\infty x^{-\frac{4}{3}}\d x = 1 + \brackets*{-3x^{-\frac{1}{3}}}_0^\infty = 4.
		\]
		We have \(\norm*{f}_{L^2\parentheses*{\Omega}} = 2\) and thus \(f \in L^2\parentheses*{\Omega}\).
		\item
		\[
			\int_0^\infty \absolute*{g\parentheses*{x}}\d x = \int_0^1 g\parentheses*{x}\d x + \int_1^\infty g\parentheses*{x}\d x = \int_0^1 x^{-\frac{3}{4}}\d x = \brackets*{4x^{\frac{1}{4}}}_0^1 = 4.
		\]
		So \(\norm*{g}_{L^1\parentheses*{\Omega}} = 4\) and therefore \(g \in L^1\parentheses*{\Omega}\).
		\[
			\int_0^\infty \absolute*{g\parentheses*{x}}^2 \d x = \int_0^1 g^2\parentheses*{x}\d x + \int_1^\infty g^2\parentheses*{x}\d x = \int_0^1 x^{-\frac{3}{2}}\d x = -2 + \lim_{x \to 0}2x^{-\frac{1}{2}} = \infty.
		\]
		We showed \(g \not\in L^2\parentheses*{\Omega}\).
		\item
		\[
			\int_\Omega \frac{1}{\sqrt[3]{\parentheses*{x + 2}^4}}e^{-\frac{2}{3}x}\d x \le \parentheses*{\int_0^\infty \parentheses*{e^{-\frac{2}{3}x}}^3 \d x}^{\frac{1}{3}}\parentheses*{\int_0^\infty \parentheses*{\parentheses*{x + 2}^{-\frac{4}{3}}}^{\frac{3}{2}}\d x}^{\frac{2}{3}} = \parentheses*{\frac{1}{2}}^{\frac{1}{3}} \cdot \parentheses*{\frac{1}{2}}^{\frac{2}{3}} = \frac{1}{2}.
		\]
	\end{enumerate}
\end{document}
