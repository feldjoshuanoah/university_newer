\section{Finite volume methods}

So far we have discussed numerical methods for conservation laws based on grid values, which had a finite difference character.
Now we turn our attention to numerical schemes based on cell mean values, which will lead to finite volume methods.
A prime application for finite volume methods is, for example, Euler's equations of gas dynamics, which is a system of conservation laws.
Indeed, it contains the sought-after fields density \(\rho\parentheses*{x, t}\), velocity \(v\parentheses*{x, t}\), and thermodynamic pressure \(p\parentheses*{x, t}\) and the system in one spatial dimension reads
\begin{align*}
	\partial_t \rho + \partial_x \parentheses*{\rho v} &= 0,\\
	\partial_t U + \partial_x F\parentheses*{U} = 0 :\iff \qquad\qquad \partial_t \parentheses*{\rho v} + \partial_x \parentheses*{\rho v^2 + p} &= 0,\\
	\partial_t \parentheses*{\rho e + \rho\frac{v^2}{2}} + \partial_x \parentheses*{\parentheses*{\rho e + \rho\frac{v^2}{2} + p}v} &= 0.
\end{align*}
Here, the internal energy \(e \equiv e\parentheses*{\rho, p}\) is a function of (i.e., it is computed from) \(\rho\) and \(p\).
The Jacobian \(D_U F\parentheses*{U}\) of the flux is diagonalizable with real eigenvalues \(\lambda_k\parentheses*{U}, k = 1, \ldots, N\), for any \(U \in \R^N, N = 3\) for \(\rho, p > 0\).


\subsection{Conservative form}

We begin with introducing the finite volume method, before studying some insightful examples.
As we have done before, concepts and observations made for the advection equations will carry over to more general settings.

\begin{definition}[Finite volume method]
	Consider an initial value problem for \(U: \R \times \R^+ \to \R^N\)
	\begin{align*}
		\partial_t U + \partial_x F\parentheses*{U} &= 0, \quad \text{with }x \in \R, t > 0,\\
		U\parentheses*{x, 0} &= U_0\parentheses*{x}, \quad \text{with }x \in \R
	\end{align*}
	where the flux \(F\) and the initial datum \(U_0\) are given.
	Let the \(\parentheses*{x, t}\)-plane be discretized as before.
	A time stepping method for \emph{cell mean values} that is of the form
	\[
		\bar{U}_j^{n + 1} = \bar{U}_j^n + \frac{\Delta t}{\Delta x}\parentheses*{\tilde{F}_{j - \frac{1}{2}}^n - \tilde{F}_{j + \frac{1}{2}}^n}
	\]
	is called \emph{finite volume method} (in conservation form); FV method for short.
	Here, \(\tilde{F}\) is called the \emph{numerical flux}.
\end{definition}

\begin{example}
	It turns out that all time stepping methods that we have introduced above can be written as finite volume methods.
	For example, the upwind method for advection with \(a > 0\) can be written as a finite volume (FV) update with
	\[
		\tilde{F}_{j + \frac{1}{2}}^n = au_j^n,
	\]
	so that \(\tilde{F}_{j - \frac{1}{2}}^n = au_{j - 1}^n\) and we can indeed identify the upwind method.
	For general nonlinear systems one can generalize this by
	\[
		\tilde{F}_{j + \frac{1}{2}}^n = F\parentheses*{U_j^n},
	\]
	but one has to change \(U_j^n \to U_{j + 1}^n\) for negative (state dependent) advection \(a < 0\).
	Similarly, the Lax-Friedrichs can be written as FV-update with
	\[
		\tilde{F}_{j + \frac{1}{2}}^n = \frac{a}{2}\parentheses*{u_j^n + u_{j + 1}^n} + \frac{\Delta x}{2\Delta t}\parentheses*{u_j^n - u_{j + 1}^n},
	\]
	which is easily checked after inserting.
	A possible nonlinear generalization is
	\[
		\tilde{F}_{j + \frac{1}{2}} n = \frac{1}{2}\parentheses*{F\parentheses*{U_j^n} + F\parentheses*{U_{j + 1}^n}} + \frac{\Delta x}{2\Delta t}\parentheses*{U_j^n - U_{j + 1}^n}.
	\]
	It turns out that this method is stable under the standard CFL condition, as we will see below.
\end{example}

\begin{remark}
	\begin{enumerate}
		\item The conservative finite-volume update follows without any approximation from the integral formulation of the conservation law.
		Indeed, choosing the domain \(\brackets*{x_{j - \frac{1}{2}}, x_{j + \frac{1}{2}}} \times \brackets*{t_n, t_{n + 1}}\) the integral formulation states that
		\begin{align*}
			\int_{x_{j - \frac{1}{2}}}^{x_{j + \frac{1}{2}}}&U\parentheses*{x, t_{n + 1}}\d x - \int_{x_{j - \frac{1}{2}}}^{x_{j + \frac{1}{2}}}&U\parentheses*{x, t_n}\d x\\
			&+ \int_{t_n}^{t_{n + 1}}F\parentheses*{U\parentheses*{x_{j + \frac{1}{2}}, t}}\d t - \int_{t_n}^{t_{n + 1}}F\parentheses*{U\parentheses*{x_{j - \frac{1}{2}}, t}}\d t = 0.
		\end{align*}
		We thus identify
		\[
			\bar{U}_j^n = \frac{1}{\Delta x}\int_{x_{j - \frac{1}{2}}}^{x_{j + \frac{1}{2}}}U\parentheses*{x, t_n}\d x \quad \text{and} \quad \tilde{F}_{j + \frac{1}{2}}^n = \frac{1}{\Delta t}\int_{t_n}^{t_{n + 1}}F\parentheses*{U\parentheses*{x_{j + \frac{1}{2}}, t}}\d t
		\]
		so that the FV update formula follows.
		\begin{center}
			\begin{tikzpicture}
				\draw[<->] (0,4) node[right] {\(t\)} -- (0,0) -- (6,0) node[above] {\(x\)};
				\draw (.125,1) -- (-.125,1) node[left] {\(t_n\)};
				\draw (.125,3) -- (-.125,3) node[left] {\(t_{n + 1}\)};
				\draw (2,.125) -- (2,-.125) node[below] {\(x_{j - \frac{1}{2}}\)};
				\draw (3.5,.125) -- (3.5,-.125) node[below] {\(x_j\)};
				\draw (5,.125) -- (5,-.125) node[below] {\(x_{j + \frac{1}{2}}\)};
				\draw (2,1) rectangle (5,3);
				\node[anchor=south] at (3.5,1) {\(\bar{U}_j^n\)};
				\node[anchor=south] at (3.5,3) {\(\bar{U}_j^{n + 1}\)};
				\draw[->] (1.5,2) node[left] {\(\tilde{F}_{j - \frac{1}{2}}^n\)} -- (2.5,2);
				\draw[->] (4.5,2) -- (5.5,2) node[right] {\(\tilde{F}_{j + \frac{1}{2}}^n\)};
			\end{tikzpicture}
		\end{center}
		\item The expression for the numerical flux \(\tilde{F}_{j + \frac{1}{2}}^n\) at time \(t_n\) cannot be used directly in the numerical method because it requires knowledge of the solution at location \(x_{j + \frac{1}{2}}\) in the entire time interval \(\brackets*{t_n, t_{n + 1}}\).
		In practice we use numerical approximations for the numerical flux \(\tilde{F}_{j + \frac{1}{2}}^n\) at \(x_{j + \frac{1}{2}}\) that in general depend on ``nearby'' numerical cell mean values
		\[
			\bar{U}_{j - p}^n, \bar{U}_{j - p + 1}^n, \ldots, \bar{U}_{j + q}^n, \quad p, q \in \N.
		\]
		For example in the simplest case we use the direct neighbors so that
		\[
			\tilde{F}_{j + \frac{1}{2}}^n \equiv F_{j + \frac{1}{2}}^n\parentheses*{\bar{U}_j^n, \bar{U}_{j + 1}^n}.
		\]
		\item For equations from continuum physics FV updates automatically ensure mass, momentum, and energy conservation in the grid cells.
		\item We recall that \(\bar{U}_{\Delta t} \equiv U_{\Delta t}\) denotes a (piecewise constant) grid function that is defined by the cell mean values \(\parentheses*{\bar{U}_j^n}_{j \in \text{grid}}\) at times \(t_n, n \ge 1\).
		Sometimes, when the time level \(t_n\) is fixed, then we will also use the notation \(\bar{U}_{\Delta x}^n \equiv U_{\Delta x}^n := \bar{U}_{\Delta t}\parentheses*{\cdot, t_n}\) for the spatial grid function at time level \(n\).
	\end{enumerate}

	\begin{theorem}[Numerical Conservation]
		The numerical approximation \(U_{\Delta x}^n\) (either with compact support or periodic) obtained by finite volume method satisfies
		\[
			\int_\R \bar{U}_{\Delta x}^n\d x = \sum_{j \in \text{grid}}\Delta x\bar{U}_j^n = \sum_{j \in \text{grid}}\Delta x\bar{U}_j^0 = \int_\R \bar{U}_{\Delta x}^0 \parentheses*{x}\d x,
		\]
		which means that the initial ``mass'' is conserved.
	\end{theorem}

	\begin{proof}
		A direct computation using the FV update yields
		\begin{align*}
			\sum_{j \in \text{grid}}\Delta x\bar{U}_j^{n + 1} &= \sum_{j \in \text{grid}}\Delta x\parentheses*{\bar{U}_j^n + \frac{\Delta t}{\Delta x}\parentheses*{\tilde{F}_{j - \frac{1}{2}}^n - \tilde{F}_{j + \frac{1}{2}}^n}}\\
			&= \sum_{j \in \text{grid}}\Delta x\bar{U}_j^n + \Delta t\sum_j \parentheses*{\tilde{F}_{j - \frac{1}{2}}^n - \tilde{F}_{j + \frac{1}{2}}^n}.
		\end{align*}
		The last term is a telescoping sum, thus
		\[
			\sum_{j \in \text{grid}}\parentheses*{\tilde{F}_{j - \frac{1}{2}}^n - \tilde{F}_{j + \frac{1}{2}}^n} = 0.
		\]
	\end{proof}
\end{remark}


\subsection{Theorem of Lax-Wendroff}

To better understand theoretical properties of finite volume methods, we require some theoretical results.

\begin{theorem}[Lax-Wendroff]
	Consider the \(N\)-dimensional system of conservation laws \(\partial_t U + \partial_x F\parentheses*{U} = 0\). Suppose that a finite volume method is given whose numerical flux \(\tilde{F}\) is
	\begin{enumerate}
		\item \emph{consistent}, in the sense that
		\[
			\tilde{F}\parentheses*{U, \ldots, U} = F\parentheses*{U},
		\]
		\item Lipschitz-continuous in all arguments, that is,
		\[
			\norm*{\tilde{F}\parentheses*{\ldots, U, \ldots} - \tilde{F}\parentheses*{\ldots, V, \ldots}} \le L\norm*{U - V} \quad \forall U, V \in \R^N,
		\]
		with some \(L > 0\).
	\end{enumerate}
	Moreover, let the numerical solution \(\bar{U}_{\Delta x}^n\) at time \(t_n\) (either with compact support or periodic) be bounded total variation.
	If \(\bar{U}_{\Delta x}^n\) converges in \(L^1\) to a function \(U\parentheses*{x, t_n}\) as \(\Delta x, \Delta t \to 0\), then \(U\) is a weak solution of the system of conservation laws.
\end{theorem}

\begin{proof}
	Too technical, no time.
\end{proof}

\begin{remark}
	\begin{enumerate}
		\item The conservative form is crucial for the correct weak solution.
		\item The actual convergence must be studied separately.
		\item If a discrete entropy condition can be guaranteed, then the solution is also an entropy solution.
	\end{enumerate}
\end{remark}

\begin{example}[Non-conservative methods]

\end{example}


\subsection{Godunov-method}


\subsection{Approximate Riemann-solver}