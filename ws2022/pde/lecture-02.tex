\section{Theory of Sobolev-spaces}

To come up with methodologies that overcome these limitations, we require some further theoretical concepts.
One important one being Sobolev spaces, introduced by S. L. Sobolev at Moscow State University around 1950.


\subsection{The necessity of additional function spaces}

\begin{definition}
	We recall the following function spaces:
	\begin{enumerate}
		\item \(C^k\parentheses*{\Omega} := \braces*{f: \Omega \to \R : k\text{-times continuous differentiable}}\),
		\item \(C^\infty\parentheses*{\Omega} := \braces*{f: \Omega \to \R : \text{any derivative is continuous}}\),
		\item \(C_0^\infty\parentheses*{\Omega} := \braces*{f: \Omega \to \R : f \in C^\infty\parentheses*{\Omega}\text{ and }\supp f \subset K \subset \Omega, K\text{ compact}}\),
		\item \(L^p\parentheses*{\Omega} := \braces*{f: \Omega \to \R : \text{Lebesgue-integrable and }\int_\Omega \absolute*{f\parentheses*{x}}^p \d x < \infty}\)
		\item \(L^\infty\parentheses*{\Omega} := \braces*{f: \Omega \to \R : \text{Lebesgue-integrable and }\absolute*{f\parentheses*{x}} < \infty\text{ for almost every }x \in \Omega}\)
	\end{enumerate}
	and the \(L^p\) norms
	\[
		\norm*{f}_1 := \int_\Omega \absolute*{f\parentheses*{x}}\d x, \quad \norm*{f}_2 := \parentheses*{\int_\Omega \absolute*{f\parentheses*{x}}^2 \d x}^{\frac{1}{2}}, \quad \norm*{f}_p := \parentheses*{\int_\Omega \absolute*{f\parentheses*{x}}^p \d x}^{\frac{1}{p}}
	\]
	\[
		\norm*{f}_\infty := \inf\braces*{C \ge 0 : \absolute*{f\parentheses*{x}} \le C\text{ for almost every }x \in \Omega}.
	\]
\end{definition}

\begin{remark}
	\begin{enumerate}
		\item Notice that \(\norm*{f}_\infty = \max_{x \in \Omega}\absolute*{f\parentheses*{x}}\), if \(f \in C\parentheses*{\Omega}\).
		\item \(L^2\parentheses*{\Omega}\) with \(\norm*{\cdot}_2\) is complete (contains all limits of Cauchy sequences) and has a scalar product
		\[
			\angles*{f, g} = \int_\Omega f\parentheses*{x}g\parentheses*{x}\d x
		\]
		such that \(\absolute*{\angles*{f, g}} \le \norm*{f}_2 \norm*{g}_2\).
		Consequently, \(L^2\parentheses*{\Omega}\) is a Hilbert space.
		This is not possible with \(\norm*{\cdot}_\infty\).
		\item We have the following situation:

		\begin{table}[h]
			\centering
			\begin{tabular}{r|ccc}
				\toprule
				& completeness & norm induced by scalar product & strictly continuous functions\\
				\midrule
				\(C^0\parentheses*{\Omega}\) with \(\norm*{\cdot}_\infty\) & yes & no & yes\\
				\(L^2\parentheses*{\Omega}\) with \(\norm*{\cdot}_2\) & yes & yes & no\\
				\bottomrule
			\end{tabular}
			\caption{}
			\label{tab:2-1}
		\end{table}

		Sobolev spaces fix this situation to some extend, as we will see below.
	\end{enumerate}
\end{remark}


\subsection{Weak derivatives}

\begin{definition}
	Let \(\Omega \subset \R^n\) and \(u \in L^2\parentheses*{\Omega}\).
	A function \(v \in L^2\parentheses*{\Omega}\) is called \emph{weak derivative} of order \(\parentheses*{\alpha_1, \ldots, \alpha_n} \in \N_0^n\) if it satisfies
	\[
		\int_\Omega v\parentheses*{x}\varphi\parentheses*{x}\d x = \parentheses*{-1}^{\alpha_1}\cdots\parentheses*{-1}^{\alpha_n}\int_\Omega u\parentheses*{x}\partial_{x_1}^{\alpha_1}\cdots\partial_{x_n}^{\alpha_n}\varphi\parentheses*{x}\d x \quad \forall\varphi \in C_0^\infty\parentheses*{\Omega},
	\]
	which is equivalent to saying that ``\(v\) satisfies the partial integration relation with \(u\)''.
\end{definition}

\begin{remark}
	\begin{enumerate}
		\item For \(n = 1\) with \(\Omega = \brackets*{a, b}\) we find
		\[
			\int_a^b v\parentheses*{x}\varphi\parentheses*{x}\d x = -\int_a^b u\parentheses*{x}\varphi'\parentheses*{x}\d x \quad \forall \varphi \in C_0^\infty\parentheses*{\brackets*{a ,b}}
		\]
		for the first derivative.
		\item If the classical derivative \(u'\) exists, it is also the weak derivative.
		\item A weak derivative \(v\) corresponds to the distributional derivative, but we also ask for \(v \in L^2\parentheses*{\Omega}\).
	\end{enumerate}
\end{remark}

\begin{example}
	Let \(u: \brackets*{a, b} \to \R\) be a continuous, piecewise differentiable function with \(N\) kinks at \(a < x_1 < \cdots < x_N < b\) such that a \(u\) is obviously not in \(C^1\parentheses*{\brackets*{a, b}}\) but has a weak derivative.
	Indeed, to show that, we have to find a function \(v \in L^2\parentheses*{\brackets*{a, b}}\) such that
	\[
		\int_a^b v\parentheses*{x}\varphi\parentheses*{x}\d x \stackrel{!}{=} -\int_a^b u\parentheses*{x}\varphi'\parentheses*{x}\d x \quad \forall \varphi \in C_0^\infty\parentheses*{\brackets*{a ,b}}.
	\]
	Respecting the kinks and using the notation \(x_0 := a\) and \(x_{N + 1} := b\), we can compute the right-hand side
	\begin{align*}
		-\int_a^b u\parentheses*{x}\varphi'\parentheses*{x}\d x &= -\sum_{i = 1}^N \int_{x_i}^{x_{i + 1}}u\parentheses*{x}\varphi'\parentheses*{x}\d x\\
		&= \sum_{i = 0}^N \parentheses*{\int_{x_i}^{x_{i + 1}}u'\parentheses*{x}\varphi\parentheses*{x}\d x - \brackets*{u\parentheses*{x}\varphi\parentheses*{x}}_{x_i}^{x_{i + 1}}}\\
		&= \sum_{i = 0}^N \int_{x_i}^{x_{i + 1}}u'\parentheses*{x}\varphi\parentheses*{x}\d x - \sum_{i = 0}^N \brackets*{u\parentheses*{x}\varphi\parentheses*{x}}_{x_i}^{x_{i + 1}}\\
		&= \sum_{i = 0}^N \int_{x_i}^{x_{i + 1}}u'\parentheses*{x}\varphi\parentheses*{x}\d x,
	\end{align*}
	where the classical derivative \(u'\) is well-defined on each open sub-interval \(\parentheses*{x_i, x_{i + 1}}\) and the boundary terms vanish because \(\varphi\) is continuous and \(\varphi\parentheses*{a} = \varphi\parentheses*{b} = 0\).
	Consequently, we identify the weak derivative
	\[
		v\parentheses*{x} = \begin{cases}
			u'\parentheses*{x}, & \text{if }x \in \parentheses*{x_i, x_{i + 1}}, i = 0, \ldots, N,\\
			\text{undefined}, & \text{if }x = x_i, i = 1, \ldots, N.
		\end{cases}
	\]
	Obviously \(v\) is piecewise continuous, but discontinuous (as undefined) at the positions \(x_1, \ldots, x_N\).
	In total, \(v \in L^2\parentheses*{\brackets*{a, b}}\), as sets f zero measure are irrelevant.
\end{example}

\begin{example}
	The Heaviside function
	\[
		u: \R \to \R, \quad x \mapsto \begin{cases}
			1, & \text{if }x > 0,\\
			0, & \text{if }x < 0
		\end{cases}
	\]
	is not weakly differentiable.
	The distributional derivative is \(u' = \delta_0\) (the Dirac distribution localized at \(0\)) satisfies the partial integration relation, but \(\delta \not\in L^2\parentheses*{\Omega}\).
\end{example}

\begin{remark}
	\begin{enumerate}
		\item The weak derivative generalizes the classical derivative in a natural way.
		\item We will abbreviate the general derivative \(\partial_{x_1}^{\alpha_1}\cdots\partial_{x_n}^{\alpha_n} =: D^\alpha\) with \(\alpha = \parentheses*{\alpha_1, \ldots, \alpha_n} \in \N_0^n\) denoting a multi-index with \(\absolute*{\alpha} = \sum_{i = 1}^n \alpha_i\) and \(D^0 u = u\).
		\item From now on we will always mean weak derivative when we write \(u'\) or \(D^\alpha\).
	\end{enumerate}
\end{remark}


\subsection{Introduction of Sobolev-spaces}

Noew we are in the position to introduce Sobolev-spaces.

\begin{definition}
	Let \(\Omega \subset \R^n\) and \(m \ge 1\).
	We define by
	\[
		H^m\parentheses*{\Omega} := \braces*{u \in L^2\parentheses*{\Omega} : D^\alpha u \in L^2\parentheses*{\Omega}\text{ for all }\alpha \in \N_0^n\text{ with }\absolute*{\alpha} \le m}
	\]
	a linear function space with the norm
	\[
		\norm*{u}_m := \parentheses*{\sum_{\absolute*{\alpha} \le m}\norm*{D^\alpha u}_{L^2\parentheses*{\Omega}}^2}^{\frac{1}{2}}
	\]
	for \(u \in H^m\parentheses*{\Omega}\) and associated scalar product
	\[
		\angles*{u, v}_m := \sum_{\absolute*{\alpha} \le m}\angles*{D^\alpha u, D^\alpha v}_{L^2\parentheses*{\Omega}}
	\]
	for \(u, v \in H^m\parentheses*{\Omega}\).
	This space is called \emph{Sobolev-space} of order \(m\).
	We define \(H^0\parentheses*{\Omega}\) as \(L^2\parentheses*{\Omega}\).
\end{definition}

\begin{remark}
	\begin{enumerate}
		\item The norm \(\norm*{\cdot}_{L^2\parentheses*{\Omega}}\) and the scalar product \(\angles*{\cdot, \cdot}_{L^2\parentheses*{\Omega}}\) are the standard norm and scalar product of the \(L^2\)-space, see above.
		From now on we will write \(\norm*{\cdot}_0\) and \(\angles*{\cdot, \cdot}_0\) for these items, as \(L^2\parentheses*{\Omega} \equiv H^0\parentheses*{\Omega}\).
		\item Sobolev-spaces are complete.
		They ensure that most ``intuitive'' limits of sequences of functions \(\braces*{u_j}_{j = 1, 2, \ldots}\) are elements of the same, shared function space in the sense that \(\lim_{j \to \infty}u_j = u \in H^m\parentheses*{\Omega}\).
		\item Every \(u \in H^m\parentheses*{\Omega}\) can be approximated to any accuracy by a sequence of \(C_0^\infty\parentheses*{\Omega}\)-functions \(\braces*{u_j}_{j = 1, 2, \ldots}\).
		This property is called the density of \(C_0^\infty\) in \(H^m\).
		This is analogous to the relation between the numbers in \(\Q\) and in \(\R\).
		\item One important case for pedagogical convenience is \(n = 1\) and one weak derivative \(m = 1\).
		For \(\Omega = \brackets*{a, b}\) we have
		\[
			H^1\parentheses*{\brackets*{a, b}} = \braces*{u \in L^2\parentheses*{\brackets*{a, b}} : u' \in L^2\parentheses*{\brackets*{a, b}}\text{ weakly}}
		\]
		and
		\begin{align*}
			\norm*{u}_1 &:= \sqrt{\norm*{u}_{L^2\parentheses*{\Omega}}^2 + \norm*{u'}_{L^2\parentheses*{\Omega}}^2} = \sqrt{\angles*{u, u}_{L^2\parentheses*{\Omega}} + \angles*{u', u'}_{L^2\parentheses*{\Omega}}} = \sqrt{\int_a^b u^2\parentheses*{x}\d x + \int_a^b u'^2\parentheses*{x}\d x}\\
			\angles*{u, v}_1 &:= \angles*{u, v}_{L^2\parentheses*{\Omega}} + \angles*{u', v'}_{L^2\parentheses*{\Omega}} = \int_a^b u\parentheses*{x}v\parentheses*{x}\d x + \int_a^b u'\parentheses*{x}v'\parentheses*{x}\d x.
		\end{align*}
	\end{enumerate}
\end{remark}


\subsection{Regularity}

Question: How regular are functions in \(H^m\parentheses*{\Omega}\)?
Continuous?
Differntiable in the classical sense?
Unsurprisingly, the answer depends crucially on the domain \(\Omega\).
Here, we restrict ourselves to the following case.

\begin{definition}
	An open, bounded, connected subset \(\Omega \subset \R^n\) (i.e., a domain) is called \emph{Lipschitz domain}, if \(\partial\Omega\) is locally (in a suitable direction) the graph of a Lipschitz function.
\end{definition}

\begin{example}
	Common polygons in \(\R^2\) and polyhedrons in \(\R^3\) are Lipschitz domains, so are smooth domains, e.g., circles and spheres.
	Cusped corners are not Lipschitz, for example:
	\begin{figure}[h]
		\centering
		\begin{tikzpicture}
			\clip (-2,-2) rectangle (2,2);
			\draw (2,2) circle (2);
			\draw (2,-2) circle (2);
			\draw (-2,-2) circle (2);
			\draw (-2,2) circle (2);
			\node at (0,0) {\(\Omega\)};
		\end{tikzpicture}
		\caption{A domain with cusped corners that is not a Lipschitz domain}
		\label{fig:2-1}
	\end{figure}
\end{example}

\begin{theorem}\label{theorem:2-12}
	Let \(\Omega \subset \R^n\) be a Lipschitz domain.
	If \(m, k \in \N_0\) satisfy
	\[
		m - \frac{n}{2} > k,
	\]
	then \(C^\infty\parentheses*{\Omega} \subset H^m\parentheses*{\Omega} \subset C^k\parentheses*{\Omega}\).
	That is, every function \(u \in H^m\parentheses*{\Omega}\) corresponds (almost everywhere) to a function from \(C^k\parentheses*{\Omega}\).
	Moreover the following embedding \(\norm*{u}_\infty \le C\norm*{u}_m\) holds.
\end{theorem}

\begin{remark}
	\begin{enumerate}
		\item The higher the value of \(m\) (i.e., the more weak derivatives), the more classical derivatives \(k\) Sobolev functions have.
		However, in higher space dimensions \(n\) this becomes more restrictive.
		\item A function \(u \in H^m\parentheses*{\Omega}\) is certainly continuous, i.e. \(u \in C^k\parentheses*{\Omega}\) for \(k = 0\), if
		\[
			u \in \begin{cases}
				H^1\parentheses*{\Omega}, & \text{for 1D (i.e., }n = 1\text{)},\\
				H^2\parentheses*{\Omega}, & \text{for 2D and 3D (i.e., }n = 2, 3\text{)}.
			\end{cases}
		\]
	\end{enumerate}
\end{remark}

\begin{example}
	We saw in that for continuous, piecewise differentiable functions \(u\) on \(\Omega = \brackets*{a, b} \subset \R\) the weak derivative exist, hence \(u \in H^1\parentheses*{\brackets*{a, b}}\).
	However, on, e.g., \(\brackets*{-1, 1}\), the function \(u\parentheses*{x} = \absolute*{x}^{\frac{3}{4}}\) is also in \(H^1\parentheses*{\brackets*{-1, 1}}\), even though the classical derivative is singular at the origin.
	Indeed, we have \(u \in L^2\parentheses*{\brackets*{-1, 1}}\) (obvious, as \(u\) is continuous and \(\Omega\) bounded), and for its first derivative
	\[
		u'\parentheses*{x} = \frac{3}{4}\absolute*{x}^{-\frac{1}{4}}.
	\]
	The squared integral of \(u'\parentheses*{x}\) on \(\brackets*{-1, 1}\) is thus
	\[
		\int_{-1}^1 \absolute*{u'\parentheses*{x}}^2 \d x = \frac{9}{16}\int_{-1}^1 \absolute*{x}^{-\frac{1}{2}}\d x = \frac{9}{8}\int_0^1 \frac{1}{\sqrt{x}}\d x = \frac{9}{8}\brackets*{2\sqrt{x}}_0^1 = \frac{9}{4} < \infty,
	\]
	verifying that \(u' \in L^2\parentheses*{\brackets*{-1, 1}}\) and thus \(u \in H^1\parentheses*{\brackets*{-1, 1}}\).
	This example shows that \(H^1\parentheses*{\Omega}\) also contains functions that are not Lipschitz.
\end{example}

\begin{example}
	For \(n = 2\) and \(\Omega = \braces*{\parentheses*{x, y} \in \R^2 : \sqrt{x^2 + y^2} \le \frac{1}{2}}\), the Sobolev-space \(H^1\parentheses*{\Omega}\) contains discontinuous (unbounded) functions.
	For example, in polar coordinates \(\parentheses*{r, \varphi}\), consider the isotropic function \(u\parentheses*{r, \varphi} = \absolute*{\ln\parentheses*{r}}^\gamma\) with \(\gamma < \frac{1}{2}\), which is unbounded at \(0\) as \(\lim_{r \to 0}u\parentheses*{r} = \infty\).
	However, it holds that
	\begin{align*}
		u \in L^2\parentheses*{\Omega}:& \quad \int_\Omega u^2 \d x = \int_0^{2\pi}\int_0^{\frac{1}{2}}\parentheses*{\absolute*{\ln r}^\gamma}^2 r\d r\d\varphi < \infty,\\
		\partial_r u \in L^2\parentheses*{\Omega}:& \quad \int_\Omega \parentheses*{\partial_r u}^2 \d x = \int_0^{2\pi}\int_0^{\frac{1}{2}}\parentheses*{\partial_r\parentheses*{\absolute*{\ln r}^\gamma}}^2 r\d r\d\varphi < \infty.
	\end{align*}
	The first integral is difficult, for the second one we use the classic chain rule \(\partial\parentheses*{\absolute*{\ln r}^\gamma} = \gamma\absolute*{\ln r}^{\gamma - 1}\sgn\parentheses*{\ln r}\frac{1}{r}\) for almost every \(r\) to eventually obtain
	\[
		\int_\Omega \parentheses*{\partial_r u}^2 \d x = 2\pi\frac{\gamma^2}{2\gamma - 1}\parentheses*{\lim_{r \to 0}\absolute*{\ln r}^{2\gamma - 1} - \ln\frac{1}{2}} < \infty
	\]
	for \(2\gamma - 1 < 0\).
	Similarly, for \(n = 3\) with \(\Omega = \braces*{\parentheses*{x, y, z} \in \R^3 : \sqrt{x^2 + y^2 + z^2} \le \frac{1}{2}}\) we find that the discontinuous function \(u\parentheses*{r} = \ln r\) (in spherical coordinates \(\parentheses*{r, \theta, \varphi}\)) is in \(H^1\parentheses*{\Omega}\).
	Indeed as above
	\begin{align*}
		u \in L^2\parentheses*{\Omega}:& \quad \int_\Omega u^2 \d x = 4\pi\int_0^{\frac{1}{2}}\parentheses*{\ln r}^2 r^2 \d r < \infty,\\
		\partial_r u \in L^2\parentheses*{\Omega}:& \quad \int_\Omega \parentheses*{\partial_r u}^2 \d x = 4\pi\int_0^{\frac{1}{2}}\parentheses*{\partial_r\parentheses*{\absolute*{\ln r}}}^2 r^2 \d r < \infty.
	\end{align*}
\end{example}

\begin{remark}
	\begin{enumerate}
		\item Obviously, we have \(H^l\parentheses*{\Omega} \subset H^k\parentheses*{\Omega}\) for \(l > k\) and \(\norm*{u}_k \le \norm*{u}_l\).
		\item For vector-valued function \(u\parentheses*{x} = \parentheses*{u_1\parentheses*{x}, \ldots, u_N\parentheses*{x}}^T \in \R^N\) we write \(u \in \parentheses*{H^m\parentheses*{\Omega}}^N\) if for all components \(u_i \in H^m\parentheses*{\Omega}, i = 1, \ldots, N\).
	\end{enumerate}
\end{remark}


\subsection{Traces}

For (unique) solutions of PDEs on bounded domains \(\Omega \subset \R^n\), we need to prescribe conditions on the boundary \(\partial\Omega \subset \R^{n - 1}\).
However, it is (a-priori) not clear if even point evaluations on \(\partial\Omega\) for \(u \in L^2\parentheses*{\Omega}\) are well-defined, because \(\partial\Omega\) is a null-set of \(\Omega\).

\begin{theorem}
	Let \(\Omega \subset \R^n\) be a Lipschitz domain and \(u \in H^1\parentheses*{\Omega}\).
	The restriction \(v: \partial\Omega \to \R\) of \(u\) to the boundary \(\partial\Omega\) of \(\Omega\) is called \emph{trace} of \(u\).
	It holds that \(v \in L^2\parentheses*{\partial\Omega}\) and \(\norm*{v}_{L^2\parentheses*{\partial\Omega}} \le C\norm*{u}_1\).
\end{theorem}

\begin{definition}
	For \(m \ge 1\) we write
	\[
		H_0^m\parentheses*{\Omega} := \braces*{u \in H^m\parentheses*{\Omega} : \left.u\right|_{\partial\Omega} = 0\text{ (in the trace sense)}}.
	\]
\end{definition}
