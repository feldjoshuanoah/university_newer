\documentclass[english]{exercise}

\DeclareMathOperator{\HI}{HI}
\DeclareMathOperator{\IoU}{IoU}


\title{Homework 2}
\author{Joshua Feld (406718)}
\professor{Prof. Kowalski}
\course{Partial Differential Equations}

\begin{document}
	\maketitle


	\section{}

	Clearly both functions are not bounded as \(x \to 0\).
	\begin{enumerate}
		\item Notice that \(\Omega_2 \subset \Omega_1\).
		Thus it suffices to show that \(u \in L^2\parentheses*{\Omega_1}\).
		A direct calculation shows that
		\begin{align*}
			\int_{\Omega_1}\absolute*{u\parentheses*{x}}^2 \d x &= C\int_0^1 \parentheses*{-\ln\parentheses*{r}}^{2k}r\d r\\
			&= C\parentheses*{\int_{\frac{1}{e}}^1 \parentheses*{-\ln\parentheses*{r}}^{2k}r\d r + \int_0^{\frac{1}{e}}\parentheses*{-\ln\parentheses*{r}}^{2k}r\d r}\\
			&\le C\parentheses*{\int_{\frac{1}{e}}^1 r\d r + \int_0^{\frac{1}{e}}\parentheses*{-\ln\parentheses*{r}}r\d r}
		\end{align*}
		and using integration by parts we obtain
		\[
			\int_0^{\frac{1}{e}}r\ln\parentheses*{r}\d r = \parentheses*{\frac{1}{2}r^2 \ln\parentheses*{r}}_0^{\frac{1}{e}} - \int_0^{\frac{1}{e}}\frac{1}{2}r\d r < \infty.
		\]
		Thus \(u \in L^2\parentheses*{\Omega_1}\).
		On the other hand we claim \(\absolute*{\nabla u}^2 \not\in L^1\parentheses*{\Omega_1}\).
		Indeed,
		\[
			\absolute*{\nabla u} = -k\parentheses*{-\ln\parentheses*{\absolute*{x}}}^{k - 1}\frac{1}{\absolute*{x}}
		\]
		and due to the radial symmetry of \(u\) we have
		\begin{align*}
			\int_{\Omega_1}k^2\parentheses*{-\ln\parentheses*{\absolute*{x}}}^{2k - 2}\frac{1}{\absolute*{x}^2}\d x &= Ck^2 \int_0^1 \parentheses*{-\ln\parentheses*{r}}^{2k - 2}\frac{1}{r}\d r\\
			&= C\frac{k^2}{1 - 2k}\int_0^1 \frac{\d}{\d r}\parentheses*{-\ln\parentheses*{r}}^{2k - 1}\d r\\
			&= C\frac{k^2}{1 - 2k}\brackets*{\parentheses*{-\ln\parentheses*{r}}^{2k - 1}}_0^1.
		\end{align*}
		Clearly, this integral does not exist.
		Consequently, \(u\) is not an \(H^1\parentheses*{\Omega_1}\) function.
		On the other hand, \(u \in H^1\parentheses*{\Omega_2}\) because \(2k - 1 < 0\) and therefore \(\lim_{r \to 0}\parentheses*{\ln\parentheses*{r}}^{2k - 1}\) exists.
		\item By definition we have \(v\parentheses*{x} = \ln\parentheses*{-\ln\parentheses*{\absolute*{x}}}\) and for all \(x \in \Omega_2\) it holds that \(-\ln\parentheses*{\absolute*{x}} > 1\).
		Consequently, the function \(v\) only has a singularity at the origin.
		We now show that \(v \in L^2\parentheses*{\Omega_2}\).
		Indeed, using the radial symmetry of \(v\) and the substitution \(\xi = -\ln\parentheses*{r}\) we obtain
		\[
			\int_{\Omega_2}\absolute*{\ln\parentheses*{-\ln\parentheses*{\absolute*{x}}}}^2 \d x = C\int_0^{\frac{1}{e}}\absolute*{\ln\parentheses*{-\ln\parentheses*{r}}}^2 r\d r = C\int_1^\infty \absolute*{\ln\parentheses*{\xi}}^2 e^{-2\xi}\d\xi.
		\]
		Since \(\ln\parentheses*{\xi} \le \xi\) for \(\xi \in \left[1, \infty\right)\) we further obtain
		\[
			\norm*{v}_{0, 2}^2 \le C\int_1^\infty < C\int_1^\infty \xi^2 e^{-2\xi}\d\xi = C\frac{5}{4}e^{-2}.
		\]
		Let us now examine the gradient of \(v\).
		With the chain rule we get
		\[
			\partial_{x_i}v\parentheses*{x} = \frac{1}{-\ln\parentheses*{\absolute*{x}}}\frac{-1}{\absolute*{x}}\frac{x_i}{\absolute*{x}}
		\]
		and therefore
		\[
			\absolute*{\nabla v} = \frac{1}{\absolute*{x}\ln\parentheses*{\absolute*{x}}}.
		\]
		Using the radial symmetry of \(v\) once again we obtain
		\[
			\norm*{\nabla v}_{0, 2}^2 = C\int_0^{\frac{1}{e}}\frac{1}{r\ln^2\parentheses*{r}}\d r = -C\brackets*{\frac{1}{\ln\parentheses*{r}}}_0^{\frac{1}{e}} = C.
		\]
		We conclude that \(v \in H^1\parentheses*{\Omega_2}\).
		We now consider the domain \(\Omega_1\).
		Clearly we have
		\[
			\norm*{v}_{L^2\parentheses*{\Omega_1}} = \norm*{v}_{L^2\parentheses*{\Omega_2}} + \norm*{v}_{L^2\parentheses*{\Omega_1 \setminus \Omega_2}}.
		\]
		It therefore suffices to show that \(v \in L^2\parentheses*{\Omega_1 \setminus \Omega_2}\).
		We have
		\begin{align*}
			\norm*{v}_{L^2\parentheses*{\Omega_1 \setminus \Omega_2}} &= C\int_{\frac{1}{e}}^1 \absolute*{\ln\parentheses*{-\ln\parentheses*{r}}}^2 r\d r\\
			&= C\int_0^1 \absolute*{\ln\parentheses*{\xi}}^2 \underbrace{e^{-2\xi}}_{\le 1}\d\xi\\
			&\le C\int_0^1 \absolute*{\ln\parentheses*{\xi}}^2 \d\xi\\
			&= C\brackets*{\xi\ln^2\parentheses*{\xi} - 2\xi\ln\parentheses*{\xi} + 2\xi}_0^1\\
			&= 2C < \infty.
		\end{align*}
		On the other hand it holds that \(\nabla v \not\in L^2\parentheses*{\Omega_1}\) since
		\[
			\norm*{\nabla v}_0^2 = C\int_0^1 \frac{1}{r\ln^2\parentheses*{r}}\d r = -C\brackets*{\frac{1}{\ln\parentheses*{r}}}_0^1 = \infty
		\]
		and consequently \(v \not\in H^1\parentheses*{\Omega_1}\).
	\end{enumerate}


	\section{}

	Let us first recall the definition of weak differentiability.
	A function \(u \in L^p\parentheses*{\Omega}\) is weakly differentiable to the index \(\alpha = \parentheses*{\alpha_1, \ldots, \alpha_n}\) if there exists a function \(v \in L^p\parentheses*{\Omega}\) such that
	\begin{equation}\label{eq:2-1}
		\int_\Omega v\parentheses*{x}\varphi\parentheses*{x}\d x = \parentheses*{-1}^{\alpha_1} \cdots \parentheses*{-1}^{\alpha_n}\int_\Omega u\parentheses*{x}D^\alpha \varphi\parentheses*{x}\d x, \quad \forall\varphi \in C_0^\infty\parentheses*{\Omega}.
	\end{equation}
	\begin{enumerate}
		\item For this case, the right-hand side of equation \eqref{eq:2-1} for \(\alpha = 1\) can be written as
		\begin{align*}
			 \int_{-1}^1 u\parentheses*{x}D^1 \varphi\parentheses*{x}\d x &= -\parentheses*{\int_{-1}^0 \parentheses*{1 + x}\varphi'\parentheses*{x}\d x + \int_0^1 \parentheses*{1 - x}\varphi'\parentheses*{x}\d x}\\
			&= -\parentheses*{\brackets*{\parentheses*{1 + x}\varphi\parentheses*{x}}_{-1}^0 - \int_{-1}^0 \varphi\parentheses*{x}\d x + \brackets*{\parentheses*{1 - x}\varphi\parentheses*{x}}_0^1 - \int_0^1 -\varphi\parentheses*{x}\d x}\\
			&= \int_{-1}^0 \varphi\parentheses*{x}\d x - \int_0^1 \varphi\parentheses*{x}\d x\\
			&= \int_{-1}^1 v\parentheses*{x}\varphi\parentheses*{x}\d x.
		\end{align*}
		Therefore, there exists a function \(v\parentheses*{x} = u'\parentheses*{x}\) satisfying the definition in equation \eqref{eq:2-1}.
		For the function \(u\parentheses*{x}\) the weak derivative exists, and is given by
		\[
			v\parentheses*{x} = \begin{cases}
				-1, & \text{if }x > 0,\\
				1, & \text{if }x < 0.
			\end{cases}
		\]
		\item In this case we have
		\[
			\parentheses*{-1}^1 \int_{-1}^1 u\parentheses*{x}D^1 \varphi\parentheses*{x}\d x = -\int_0^\infty x\varphi'\parentheses*{x}\d x = \int_0^\infty \varphi\parentheses*{x}\d x - \brackets*{x\varphi\parentheses*{x}}_0^\infty \stackrel{\varphi\parentheses*{\infty} = 0}{=} \int_{-\infty}^\infty v\parentheses*{x}\varphi\parentheses*{x}\d x.
		\]
		For this function the weak derivative also exists and is given by
		\[
			v\parentheses*{x} = \begin{cases}
				1, & \text{if }x > 0,\\
				0, & \text{if }x \le 0.
			\end{cases}
		\]
		\item Following the definition of weak differentiability, we get
		\begin{align*}
			-\int_0^2 u\parentheses*{x}\varphi'\parentheses*{x}\d x &= -\parentheses*{\int_0^1 x\varphi'\parentheses*{x}\d x + 2\int_1^2 \varphi'\parentheses*{x}\d x}\\
			&= -\parentheses*{\brackets*{x\varphi\parentheses*{x}}_0^1 - \int_0^1 \varphi\parentheses*{x}\d x + 2\brackets*{\varphi\parentheses*{x}}_1^2}\\
			&\stackrel{\varphi\parentheses*{2} = 0}{=} \int_0^1 \varphi\parentheses*{x}\d x + \varphi\parentheses*{1}.
		\end{align*}
		We can't simplify further and find a function \(v\parentheses*{x}\) like in the last two examples.
		We need to follow some other strategy to find whether the weak derivative exists or not.
		For the sake of argument assume that the weak derivative exists.
		We then have
		\begin{equation}\label{eq:2-2}
			-\int_0^2 u\parentheses*{x}\varphi'\parentheses*{x}\d x = \int_0^2 v\parentheses*{x}\varphi\parentheses*{x}\d x.
		\end{equation}
		Here \(v\parentheses*{x}\) is the weak derivative.
		Consider a sequence of smooth functions \(\braces*{\varphi_m}_{m = 1}^\infty\) satisfying
		\[
			0 \le \varphi_m\parentheses*{x} \le 1, \quad \lim_{m \to \infty}\varphi_m\parentheses*{1} = 1 \quad \text{and} \quad \lim_{m \to \infty}\varphi_m\parentheses*{x} = 0 \quad \forall x \ne 1.
		\]
		Now replacing \(\varphi\) by \(\varphi_m\) in equation \eqref{eq:2-2}, we get
		\[
			\lim_{m \to \infty}\int_0^2 v\parentheses*{x}\varphi_m\parentheses*{x}\d x = \lim_{m \to \infty}-\int_0^2 u\parentheses*{x}\varphi_m'\parentheses*{x}\d x = \lim_{m \to \infty}\int_0^1 \varphi\parentheses*{x}\d x + \lim_{m \to \infty}\varphi_m\parentheses*{1} = 1,
		\]
		which is a contradiction since
		\[
			\lim_{m \to \infty}\int_0^2 v\parentheses*{x}\varphi_m\parentheses*{x}\d x = 0.
		\]
		Therefore, the assumption that the weak derivative exists is not correct.
		Hence, for this function the weak derivative does not exist.
		\item If we use the definition again, we see that
		\[
			\parentheses*{-1}^1 \int_{-\infty}^\infty u\parentheses*{x}\varphi'\parentheses*{x}\d x = -\int_0^\infty \varphi'\parentheses*{x}\d x = -\brackets*{\varphi\parentheses*{x}}_0^\infty \stackrel{\varphi\parentheses*{\infty} = 0}{=} \varphi\parentheses*{0}.
		\]
		Obviously, there exists no function \(v\parentheses*{x} \in L^p\parentheses*{\Omega}\) such that
		\[
			\int_{-\infty}^\infty v\parentheses*{x}\varphi\parentheses*{x}\d x = \varphi\parentheses*{0}, \quad \forall\varphi \in C_0^\infty\parentheses*{\Omega}.
		\]
		Therefore, the weak derivative does not exists in this case as well.
	\end{enumerate}


	\section{}

	Let us take a look at the partial derivatives of the solution \(u\parentheses*{r, \varphi}\).
	First of all, the input parameters \(r\) and \(\varphi\) are separable as
	\[
		u\parentheses*{r, \varphi} = R\parentheses*{r}\Phi\parentheses*{\varphi}.
	\]
	Hence, any partial derivative is a product of derivatives of \(R\parentheses*{r}\) and \(\Phi\parentheses*{\varphi}\)
	\[
		\frac{\partial^{i_1 + i_2}u\parentheses*{r, \varphi}}{\partial r^{i_1}\partial\varphi^{i_2}} = R^{\parentheses*{i_1}}\parentheses*{r}\Phi^{\parentheses*{i_2}}\parentheses*{\varphi}.
	\]
	To compute the \(L^2\)-norm of the above mentioned derivative, we adjust the integral by Jacobian of polar coordinate system (which is a simple \(r\) multiplier):
	\[
		\norm*{\frac{\partial^{i_1 + i_2}u\parentheses*{r, \varphi}}{\partial r^{i_1}\partial\varphi^{i_2}}}_{L^2\parentheses*{\Omega}}^2 = \int_0^\alpha \parentheses*{\varphi^{\parentheses*{i_2}}\parentheses*{\varphi}}^2 \d\varphi \int_0^1 \parentheses*{R^{\parentheses*{i_1}}\parentheses*{r}}^2 r\d r.
	\]
	If \(\parentheses*{R^{\parentheses*{i_1}}\parentheses*{r}}^2 r\) is \(L^2\)-integrable on \(\brackets*{0, 1}\), then the derivative of the solution is \(L^2\)-integrable on \(\Omega\).
	Now we look at the first and second derivate of \(R\parentheses*{r}\):
	\begin{align*}
		R'\parentheses*{r} &= \frac{\pi}{\alpha}r^{\frac{\pi}{\alpha} - 1},\\
		R''\parentheses*{r} = \frac{\pi}{\alpha}\parentheses*{\frac{\pi}{\alpha} - 1}r^{\frac{\pi}{\alpha} - 2}
	\end{align*}
	and the corresponding integrals
	\begin{align*}
		\int_0^1 \parentheses*{R'\parentheses*{r}}^2 r\d r &= \int_0^1 \parentheses*{\frac{\pi}{\alpha}}^2 r^{\frac{2\pi}{\alpha} - 1}\d r = \frac{1}{2}\frac{\pi}{\alpha}\brackets*{r^{\frac{2\pi}{\alpha}}}_0^1,\\
		\int_0^1 \parentheses*{R''\parentheses*{r}}^2 r\d r &= \int_0^1 \parentheses*{\frac{\pi}{\alpha}}^2 r^{\frac{2\pi}{\alpha} - 3}\d r = \frac{1}{2}\parentheses*{\frac{\pi}{\alpha}}^2 \parentheses*{\frac{\pi}{\alpha} - 1}\brackets*{r^{\frac{2\pi}{\alpha} - 2}}_0^1.
	\end{align*}
	So, we get to the answer:
	\begin{enumerate}
		\item For any \(\alpha > 0\) the first derivative of the solution \(u\parentheses*{r, \varphi}\) is \(L^2\parentheses*{\Omega}\)-integrable.
		So we obtain \(\alpha_{\text{max}} = 2\pi\).
		\item The second derivative is \(L^2\parentheses*{\Omega}\)-integrable if \(\frac{2\pi}{\alpha} - 2 > 0\) or \(\frac{\pi}{\alpha} - 1 = 0\).
		Thus \(\alpha_{\text{max}} = \pi\).
	\end{enumerate}
\end{document}