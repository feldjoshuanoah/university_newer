\section{Lineare Regelkreisglieder}


\subsection{Allgemeines}

Im vorangehenden Abschnitt sind wiederholt einzelne einfache Regelkreisglieder und ihre dynamischen Eigenschaften als Beispiele behandelt worden.
Dies hatte vor allem das Ziel, die neu eingeführten Beschreibungsmittel zu erläutern.
Im Folgenden soll eine systematische Übersicht über häufig auftretende, relativ einfach zu beschreibende Regelkreisglieder gegeben werden.
Damit und mit den Tabellen \ref{tab:4-2} bis \ref{tab:4-4} soll besonders betont werden, dass die analytischen Ausdrücke für Differentialgleichung und Übertragungsfunktion, die Übergangsfunktion, die Pole und Nullstellen der Übetragungsfunktion, das Bode-Diagramm und die Ortskurve des Frequenzgangs prinzipiell gleichberechtigte Beschreibungsmittel sind.
Dies bedeutet auch, dass das dynamische Verhalten ein und desselben Regelkreisgliedes durch unterschiedliche Mittel dargestellt und dass für unterschiedliche Anwendungsfälle die jeweils geeignetste Darstellungsform ausgewählt werden kann.

Hier, wie auch im vorherigen Abschnitt, werden nur linear, zeitinvariante Übertragungsglieder mit einer Eingangs- und einer Ausgangsgröße behandelt, die durch lineare Differentialgleichungen mit konstanten reellen Koeffizienten, z.B. \eqref{eq:3-1} und einem damit in Reihe geschalteten Totzeitglied, z.B. \eqref{eq:3-109}, beschrieben werden können.
Damit werden all hier behandelten Übertragungsglieder durch die allgemeine Differentialgleichung
\begin{equation}\label{eq:4-1}
	a_n y^{\parentheses*{n}}\parentheses*{t} + \cdots + a_1 \dot{y}\parentheses*{t} + a_0 y\parentheses*{t} = b_0 u\parentheses*{t - T_t} + b_1 \dot{u}\parentheses*{t - T_t} + \cdots + b_m u^{\parentheses*{m}}\parentheses*{t - T_t}
\end{equation}
oder durch den zugehörigen Frequenzgang
\begin{equation}\label{eq:4-2}
	G\parentheses*{j\omega} = \frac{\underline{y}}{\underline{u}} = \frac{b_0 + b_1 j\omega + \cdots + b_m \parentheses*{j\omega}^m}{a_n \parentheses*{j\omega}^n + \cdots + a_1 j\omega + a_0}e^{-j\omega T_t}
\end{equation}
beschrieben.
Alle im Folgenden betrachteten und in den Tabellen \ref{tab:4-2} bis \ref{tab:4-4} dargestellten Übertragungsglieder sind Sonderformen der Gleichungen \eqref{eq:4-1} und \eqref{eq:4-2}.
Um den Umfang der Tabellen zu begrenzen, sind nur häufig vorkommende Übertragungsglieder berücksichtigt worden.
Die Graphiken für Übergangsfunktion, Ortskurve und Bode-Diagramm sollen den prinzipiellen Verlauf dieser Funktionen veranschaulichen.


\subsection{Anmerkungen zu den Tabellen}


\subsubsection{$P$, $I$, $D$}

Übertragungsglieder mit proportionalem, integrierendem oder differenzierendem Verhalten sind in Abschnitt 3 behandelt worden.
Sie sind in Regelstrecken, Mess- und Stellgeräten sowie in Reglern anzutreffen.
Auf die Unterschiede im statischen Verhalten von \(P\)- und \(I\)-Reglern ist in Abschnitt 2 hingewiesen worden; hier ist noch zu ergänzen, dass \(D\)-Glieder als Regler im engeren Sinne nur selten eingesetzt werden.


\subsubsection{$PI$, $PD$, $PID$}

Übertragungsglieder mit \(PI\)-, \(PD\)- und \(PID\)-Verhalten treten hauptsächlich als Regler auf.
Umgekehrt sind mit ganz wenigen Ausnahmen nahezu alle praktisch eingesetzten linearen Regler vom so genannten \(PID\)-Typ, d.h. sie lassen sich als Vereinfachungen des \(PID\)-Reglers auffassen.

Das \(PD\)-Glied entsteht durch Parallelschalten eines proportional und eines differenzierend wirkenden Gliedes nach Bild \ref{fig:4-3}.
Als \(PD\)-Regler (Proportional-Differential-Regler) eingesetzt, nutzt es beben der Regelabweichung auch noch deren Änderungsgeschwindigkeit zum Bilden der Stellgröße aus.
Die Differentialgleichung dieser Parallelschaltung
\begin{equation}
	y = K_R \cdot u + K_D \cdot \dot{u}
\end{equation}
wird üblicherweise in der Form
\[
	y = K_R \parentheses*{u + T_v \cdot \dot{u}}
\]
geschrieben, mit der sogenannten Vorhaltzeit
\[
	T_v = \frac{K_D}{K_R}.
\]
Die Übergangsfunktion (Tabelle \ref{tab:4-2}) entsteht durch Addition der Übergangsfunktionen des \(P\)- und des \(D\)-Gliedes.

Die Übertragungsfunktion des \(PD\)-Reglers
\begin{equation}
	G\parentheses*{s} = K_R\parentheses*{1 + sT_v}
\end{equation}
wird durch die Vorhaltzeit \(T_v\) und die entsprechende Nullstelle \(s_N = -\frac{1}{T_v}\) gekennzeichnet.
Sie entspricht der inversen Übertragungsfunktion des \(PT_1\)-Gliedes.

Der Frequenzgang
\begin{equation}
	G\parentheses*{j\omega} = K_R\parentheses*{1 + j\omega T_v}
\end{equation}
weist einen konstanten Realteil \(K_R\) und einen kreisfrequenzanhängigen Imaginärteil \(\omega K_R T_v\) auf, seine Ortskurve ist eine Parallele zur positiv-imaginären Achse.



\subsubsection{$PT_1$, $PT_2$, $PT_n$}


\subsubsection{$IT_1$}


\subsubsection{$DT_1$}


\subsubsection{$PIT_1$}


\subsubsection{$PDT_1$, $PPT_1$}


\subsubsection{$PT_t$, $PT_1 T_t$}


\subsubsection{$PA_1$}


\subsection{Minimalphasenglieder, Phasenminimumsysteme}
