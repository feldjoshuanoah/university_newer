\documentclass{exercise}

\DeclareMathOperator{\HI}{HI}
\DeclareMathOperator{\IoU}{IoU}


\title{Übung 1}
\author{Joshua Feld, 406718}
\professor{Prof. Abel}
\course{Regelungstechnik}

\begin{document}
	\maketitle


	\section{}

	\begin{quote}
		Bei der skizzierten Geschwindigkeitsregelung eines Kraftfahrzeuges liefert ein Tachgenerator eine Spannung \(U_x\), die der Raddrehzahl \(N_R\) und damit der Geschwindigkeit \(V\) proportional ist.
		In Abhängigkeit von der Regelabweichung liefert ein Regler, der proportionales Übertragungsverhalten haben soll, ein Signal \(U_y\), das auf den Öffnungswinkel \(\Theta\) der Drosselklappe wirkt.
		Damit wird das Motormoment \(M_M\) beeinflusst, das außerdem noch von der Motordrehzahl \(N_M\) abhängt (mit steigender Motordrehzahl \(N_M\) nimmt das Motormoment \(N_M\) ab).

		Zeichnen Sie einen Wirkungsplan für kleine Abweichungen on einem Arbeitspunkt.
		Verwenden Sie als Ausgangsgröße die Abweichungsgröße \(x\) der Fahrzeuggeschwindigkeit.
		Berücksichtigen Sie alle im Gerätebild aufgeführten Variablen.
		Tragen Sie die Übergangsfunktionen in die Übertragungsblöcke ein.
		Kennwerte an den Übertragungsblöcken sind nicht erforderlich.
		\begin{center}
			\begin{tikzpicture}
				\filldraw[fill=red] (0,0) rectangle (5,5) node[pos=.5] {\emph{TODO}};
 			\end{tikzpicture}
		\end{center}
		Alle nicht gegebenen Massen werden vernachlässigt.
		\begin{center}
			\begin{tabular}{ll}
				\(F_L\) & Luftwiderstand\\
				\(F_R\) & Rollwiderstand\\
				\(F_G\) & Hangantriebskraft\\
				\(F_A\) & Antriebskraft\\
				\(V = X\) & Fahrzeuggeschwindigkeit (\(X\): Regelgröße)\\
				\(\Phi = Z\) & Straßenneigungswinkel (\(Z\): Störgröße)\\
				\(G\) & Gewichtskraft\\
				\(M\) & Fahrzeugmasse\\
				\(W\) & Führungsgröße des Reglers\\
				\(N_M\) & Motordrehzahl\\
				\(M_M\) & Motormoment\\
				\(N_R\) & Raddrehzahl\\
				\(M_R\) & Moment an der Antriebsachse\\
				\(U_x\) & Tachogeneratorspannung\\
				\(U_y\) & Spannung am Reglerausgang\\
				\(I\) & Spulenstrom\\
				\(S_1\) & Position Schubgestänge\\
				\(S_2\) & Position Eisenkern\\
				\(\Theta = Y\) & Drosselklappenwinkel (\(Y\): Stellgröße)
			\end{tabular}
		\end{center}
		\(M, G, F_R\) sind konstant.
	\end{quote}

	Wirkungspläne gelten für kleine Abweichungen von einem Arbeitspunkt.
	Es werden daher lediglich Abweichungsgrößen (kleine Buchstaben) eingetragen.
	Konstante Eingangsgrößen haben keinen Einfluss auf das zu modellierende System und tauchen deshalb nicht im Wirkungsplan auf.

	Zur Darstellung von Zusammenhängen in Wirkungsplänen sind zu bestimmen:
	\begin{enumerate}[label=\arabic*.]
		\item Struktur: Abhängigkeit der Signale feststellen

		Frage: Wovon hängt das betrachtete Signal \emph{direkt} ab?

		Bsp.:
		\begin{equation}\label{eq:1-1}
			x = x\parentheses*{f_{\text{res}}, t}
		\end{equation}
		\item Vorzeichen: Wirkung tendenziell feststellen

		Frage: Das Eingangssignal wird größer.
		Wird das Ausgangssignal größer (positives Vorzeichen) oder kleiner (negatives Vorzeichen)?
		\item Übergangsfunktion: Dynamisches Verhalten bestimmen

		Frage: Wie ist der zeitliche Verlauf des Ausgangssignals, wenn sich das Eingangssignal sprungförmig ändert?

		Bsp.: Die funktionale Abhängigkeit \ref{eq:1-1} wird durch die Newtonsche Bewegungsgleichung beschrieben
		\[
			M\dot{x} = f_{\text{res}} \iff \dot{x} = \frac{1}{M}f_{\text{res}} \iff x = \frac{1}{M}\int f_{\text{res}}\d t
		\]
	\end{enumerate}
	\begin{center}
		\begin{tikzpicture}
			\filldraw[fill=red] (0,0) rectangle (5,5) node[pos=.5] {\emph{TODO}};
		\end{tikzpicture}
	\end{center}
\end{document}
