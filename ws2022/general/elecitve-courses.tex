\documentclass{lecture}

\title{Vertiefung}
\author{Joshua Feld, 406718}
\professor{alle Berufsfeldbetreuer}
\course{Allgemeines}

\begin{document}
	\maketitle


	\section{Vorgaben}

	Es müssen Vertiefungsfächer aus sechs Katalogen im Umfang von 24 bis 28 Credit Points (kurz: CP) ausgewählt werden.
	Dabei gibt es noch zwei Beschränkungen:
	\begin{itemize}
		\item Aus dem Katalog Mathematik-Informatik dürfen max. 12 CP belegt werden.
		\item Die Wahl der Fächer muss in sich schlüssig sein, d.h. es sollten ein oder zwei fachliche Schwerpunkte erkennbar sein.
		Dies wird vom Berufsfeldbetreuer und von der Fakultät geprüft.
	\end{itemize}


	\section{Was kann ich wählen?}

	Die Liste der Vertiefungsfächer aktuell zu halten ist nicht ganz einfach, da regelmäßig neue Fächer dazu kommen oder alte abgeschafft werden.
	Die aktuellste Liste ist auf der Website der Fakultät im Studienverlaufsplan zu finden.
	Außerdem ist es hilfreich die dazugehörigen Einträge in RWTHonline im aktuellen und alten (oder falls schon verfügbar, nächsten) Semester anzuschauen (5. bzw 6. Semester, Wahlpflichtbereich).
	Hast du interessante Module gefunden, kannst du auf den dazugehörigen Institutsseiten oder im Maschboard mehr über diese Module erfahren und ggf. auch schon in Material reinschauen.
	Ältere Kommilitonen geben auch gerne Auskunft, allerdings ist diese nicht immer objektiv.

	\begin{remark}
		\begin{enumerate}
			\item Es ist auch möglich außerhalb des Kataloges zu wählen, schaue dir dabei die Kataloge ähnlicher Studiengänge an oder schau auf den Institusseiten nach, welche Module sie anbieten.
			Ein Modul aus dem Master Katalog muss als ``nicht im Wahlkatalog'' markiert werden.
			\item Solltest du mit dem Gedanken spielen auch den Master CES an der RWTH zu machen, schaue in den Studienplan des Masters.
			Diverse Fächer kann man regulär sowohl im Bachelor, als auch im Master belegen.
			Dies ist angenehm wenn man sich für mehr Fächer interessiert als man CP im Bachelor zur Verfügung hat.
			Auch gibt es Module, die aufeinander aufbauen, d.h. das vorherige Modul sollte bestanden sein um das nächste belegen zu können.
		\end{enumerate}
	\end{remark}


	\section{Wo muss ich meinen Antrag abgeben?}

	Hast du dich für eine Modulkombination entschieden, lade das Studienplanformular von der Fakultät herunter und fülle es aus.
	Unterschreibe es und bringe es dann zu deinem Berufsfeldbetreuer.
	Nachdem du deinen Studienplan bei deinem Berufsfeldbetreuer hast unterzeichnen lassen, musst du ihn beim Prüfungsausschuss einreichen.
	Dies ist der rote Briefkasten in der Kackertstr. 9, 2. Stock.
	Falls es knapp werden sollte, kannst du den Antrag auch in den Fristenbriefkasten am Hauptgebäude werfen.


	\section{Wann muss ich meinen Antrag abgeben?}

	Es gibt keine Frist, d.h. du kannst den Antrag abgeben wann du möchtest.
	Du solltest aber den bewilligten Antrag von der Fakultät zurück bekommen haben, bevor du Klausuren aus dem Wahlpflichtbereich anmeldest oder gar schreibst.
	Einmal angemeldete Prüfungen sind nicht ohne Aufwand aus dem ZPA löschbar und geschriebene Prüfungen müssen auf jeden Fall bestanden werden, egal ob sie angerechnet werden oder nicht.
	Ebenfalls solltest du einkalkulieren, dass der ganze Prozess mehrere Wochen dauern kann, da der Betreuer zuerst den Plan fachlich bestätigen muss und danach der Prüfungsausschuss formal den Antrag prüft, also frühzeitig abgeben!


	\section{Was wenn ich mich umentscheide?}

	Solltest du im Verlauf des Semesters merken, dass eines oder mehrere deiner Wahlmodule dir doch nicht gefallen, muss du einen neuen Studienplan einreichen.
	Beachte aber, dass du in den Modulen, die du wechseln willst, noch keine Prüfung abgelegt haben darfst!
	Sobald eine Prüfungsleistung erbracht wurde, musst du das Modul behalten.


	\section{Berufsfeldbetreuer}

	Dein Berufsfeldbetreuer ist derjenige Professor, in dessen Modulkatalog du am meisten belegt hast.
	Wer das momentan ist und wo sein Büro ist, findest du in der Liste der Berufsfeldbetreuer.

	\begin{center}
		\begin{tabular}{ll}
			\toprule
			Berufsfeld & Betreuer\\
			\midrule
			Energie- und Verfahrenstechnik & Prof. Mitsos\\
			Informatik & Prof. Naumann\\
			Mathematik & Prof. Torrilhon\\
			Mechanische Systeme & Prof. Behr\\
			Strömung und technische Verbrennung & Prof. Pitsch\\
			\bottomrule
		\end{tabular}
	\end{center}


	\section{Pflicht für alle}

	\begin{center}
		\begin{tabular}{lcccccc}
			\toprule
			\multirow{3}{*}{Pflichtveranstaltungen} & \multicolumn{3}{c}{5. Semester} & \multicolumn{3}{c}{6. Semester}\\
			& \multicolumn{3}{c}{SWS im WS} & \multicolumn{3}{c}{SWS im SS}\\
			& V & U & CP & V & Ü & CP\\
			\midrule
			Regelungstechnik & 3 & 2 & 6 & & & \\
			Modellgestützte Schätzmethoden & & & & 2 & 2 & 5\\
			Numerische Strömungssimulation & & & & 1 & 3 & 5\\
			Partielle Differentialgleichungen & 4 & 2 & 9 & & & \\
			Data Analysis and Visualization & 2 & 1 & 4 & & & \\
			Softwareentwicklungspraktikum & 0 & 3 & 3 & & & \\
			Projektaufgabe & & & & & & 5\\
			\midrule
			Zwischensumme & \multicolumn{3}{c}{22} & \multicolumn{3}{c}{15}\\
			Wahlbereich & \multicolumn{3}{c}{10} & \multicolumn{3}{c}{14}\\
			Gesamt & \multicolumn{3}{c}{32} & \multicolumn{3}{c}{29}\\
			\bottomrule
		\end{tabular}
	\end{center}
\end{document}