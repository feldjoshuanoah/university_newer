\documentclass{lecture}

\DeclareMathOperator*{\cond}{cond}
\DeclareMathOperator*{\diag}{diag}
\DeclareMathOperator*{\dist}{dist}
\DeclareMathOperator*{\esssup}{ess\,sup}
\DeclareMathOperator*{\vol}{vol}


\title{Vorlesung 1}
\author{Joshua Feld, 406718}
\professor{Prof. Stamm}
\course{Mathematische Grundlagen III}

\begin{document}
	\maketitle


	\tableofcontents


	\section{Einfache Einschrittverfahren}

	Zunächst wollen wir kurz einige Notationen klären und nützliche Resultate zur Theorie gewöhnlicher Differentialgleichungen rekapitulieren, die wir im Folgenden benutzen werden.

	Wir betrachten Anfangswertprobleme für gewöhnliche Differentialgleichungen:
	\begin{equation}
		\begin{alignedat}{2}
			\dot{x}\parentheses*{t} &= f\parentheses*{t, x\parentheses*{t}}, \quad & &a \le t \le b,\\
			x\parentheses*{t_0} &= x_0, & &a \le t_0 \le b.
		\end{alignedat}
	\end{equation}
	Dabei ist die rechte Seite gegeben durch
	\[
		f: \brackets*{a, b} \times \R^d \to \R^d, f = \begin{pmatrix}
			f_1\\
			\vdots\\
			f_d
		\end{pmatrix}
	\]
	und die Lösung hat die Form
	\[
		x: \brackets*{a, b} \to \R^d, x = \begin{pmatrix}
			x_1\\
			\vdots\\
			x_d
		\end{pmatrix}.
	\]

	\begin{definition}
		Die Funktion \(f: \brackets*{a, b} \times \R^d \to \R^d\) ist \emph{global Lipschitz-stetig} bzgl. \(x\), falls
		\begin{equation}
			\norm*{f\parentheses*{t, x_1} - f\parentheses*{t, x_2}} \le L\norm*{x_1 - x_2}
		\end{equation}
		für alle \(t \in \brackets*{a, b}\) und \(x_1, x_2 \in \R^d\).
	\end{definition}

	Für global Lipschitz-stetige rechte Seiten gilt der Existenz- und Eindeutigkeitssatz von Picard-Lindelöf.

	\begin{proposition}[Picard-Lindelöf]
		Sei \(f\) global Lipschitz-stetig bzgl. \(x\).
		Dann existiert zu jedem \(x_0 \in \R^d\) und \(a \le t_0 \le b\) genau eine Lösung \(x \in C^1\parentheses*{\brackets*{a, b}}\) des Anfangswertproblems 
	\end{proposition}

	Für die weitere Untersuchung führen wir folgende Sprechweise ein:
	Sei \(x\) die Lösung des AWP \(\dot{x} = f\parentheses*{t, x}, x\parentheses*{t_0} = x_0\) auf \(\brackets*{a, b}\).
	Dann ist durch
	\[
		\phi^{t, t_0}x_0 = x\parentheses*{t; t_0, x_0}
	\]
	die \emph{Evolution der Differentialgleichung} gegeben.
	Für die Evolution gilt:
	\begin{itemize}
		\item \(\phi^{t, t}x_0 = x_0\),
		\item \(\left.\frac{\d}{\d\tau}\phi^{t + \tau, t}x\right|_{\tau = 0} = f\parentheses*{t, x}\),
		\item \(\phi^{t, \sigma}\phi^{\sigma, s}x = \phi^{t, s}x\).
	\end{itemize}

	\begin{remark}
		Betrachtet man autonome Probleme
		\[
			\dot{x} = f\parentheses*{x}, \quad x\parentheses*{t_0} = x_0,
		\]
		dann entsteht durch Translation um \(\tau \in \R\) eine Lösung \(\hat{x}\parentheses*{t} = x\parentheses*{t - \tau}\) des AWP
		\[
			\frac{\d}{\d t}\hat{x} = f\parentheses*{\hat{x}}, \quad \hat{x}\parentheses*{t_0 + \tau} = x_0.
		\]
		D.h. bei autonomen Problemen spielt der Startpunkt keine Rolle.
		Wir wählen immer \(t_0 = 0\) und \(a \le 0 \le b\), und schreiben \(\phi^t\) statt \(\phi^{t, t_0}\).
	\end{remark}

	Will man die Auswirkung von Störungen in den Anfangswerten für \emph{lange} Zeiten untersuchen, so muss man die Stabilität von Lösungen betrachten.
	Seien \(\parentheses*{t_0, x_0}\) und die Evolution \(\phi^{t, t_0}x_0\) der DGL \(\dot{x} = f\parentheses*{t, x}\) für alle \(t \ge t_0\) gegeben.
	Die Lösung \(\phi^{t, t_0}x_0\) heißt
	\begin{itemize}
		\item \emph{stabil (Ljapunov-stabil)}, falls für alle \(\varepsilon > 0\) ein \(\delta > 0\) existiert, so dass
		\[
			\phi^{t, t_0}x \in B_\varepsilon\parentheses*{\phi^{t, t_0}x_0}
		\]
		für alle \(t > t_0\) für alle gestörten Anfangswerte \(x \in B_\delta\parentheses*{x_0}\),
		\item \emph{asymptotisch stabil}, falls zusätzlich ein \(\delta_0 > 0\) existiert, so dass
		\[
			\lim_{t \to \infty}\absolute*{\phi^{t, t_0}x_0 - \phi^{t, t_0}x} = 0
		\]
		für alle \(x \in B_{\delta_0}\parentheses*{x_0}\),
		\item \emph{instabil} in allen anderen Fällen.
	\end{itemize}
	Wir untersuchen zunächst homogene, lineare autonome Systeme:
	\[
		\dot{x} = Ax, \quad A \in \R^{d \times d}.
	\]
	Dann ist \(\phi^t: \R^d \to \R^d\) eine lineare Abbildung.
	Die Stabilität eines linearen autonomen Systems kann leicht überprüft werden.

	\begin{definition}
		\begin{enumerate}
			\item Das \emph{Spektrum} einer Matrix \(A \in \C^{d \times d}\) ist
			\[
				\sigma\parentheses*{A} = \braces*{\lambda \in \C : \det\parentheses*{\lambda I - A} = 0}.
			\]
			\item Die \emph{Spektralabszisse} von \(A\) ist der maximale Realteil der Eigenwerte
			\[
				\nu\parentheses*{A} = \max_{\lambda \in \sigma\parentheses*{A}}\Re\parentheses*{\lambda}.
			\]
			\item Der \emph{Index} \(i\parentheses*{\lambda}\) eines Eigenwertes \(\lambda \in \sigma\parentheses*{A}\) ist die maximale Dimension der zu \(\lambda\) gehörenden Jordanblöcke von \(A\).
		\end{enumerate}
	\end{definition}

	\begin{proposition}
		Das lineare Anfangswertproblem \(\dot{x} = Ax, A \in \R^{d \times d}\) ist genau dann stabil, falls \(\nu\parentheses*{A} \le 0\) und alle Eigenwerte \(\lambda \in \sigma\parentheses*{A}\) mit \(\Re\parentheses*{\lambda} = 0\) den Index \(i\parentheses*{\lambda} = 1\) besitzen.
		Es ist asymptotisch stabil, falls \(v\parentheses*{A} < 0\) gilt.
	\end{proposition}

	Jetzt wollen wir uns der numerischen Lösung von Anfangswertproblemen zuwenden.
	Eine Idee dazu ist uns bereits beim Beweis des Existenzsatzes von Peano über den Weg gelaufen, nämlich das sogenannte Polygonzugverfahren.
	Man benutzt zur Approximation der Lösung \(x\parentheses*{t}\) von
	\[
		\dot{x}\parentheses*{t} = f\parentheses*{t, x\parentheses*{t}}, \quad x\parentheses*{t_0} = x_0
	\]
	ein Polygon, das man wie folgt erhält:
	Wähle eine Unterteilung (oder auch Gitter) \(\Delta = \braces*{t_0, \ldots, t_n : t_0 < \cdots < t_n = T}\) des Intervalls \(\brackets*{t_0, T}\) und konstruiere eine Approximation \(x_\Delta\) durch folgende rekursive Vorschrift:
	\begin{align*}
		x_\Delta\parentheses*{t_0} &= x_0,\\
		x_\Delta\parentheses*{t} &= x_0 + \parentheses*{t - t_0}\underbrace{f\parentheses*{t_0, x_0}}_{\sim \dot{x}\parentheses*{t_0}} &&\text{für }t \in \brackets*{t_0, t_1},\\
		x_\Delta\parentheses*{t} &= x_\Delta\parentheses*{t_j} + \parentheses*{t - t_j}f\parentheses*{t_j, x_\Delta\parentheses*{t_j}} &&\text{für }t \in \brackets*{t_j, t_{j + 1}}, j = 0, \ldots, n - 1.
	\end{align*}
	D.h. \(x_j := x_\Delta\parentheses*{t_j}\) ist gegeben durch die Iterationsvorschrift
	\begin{equation}
		x_{j + 1} = x_j + \parentheses*{t_{j + 1} - t_j}f\parentheses*{t_j, x_j},
	\end{equation}
	wobei \(x_0\) gegeben ist.
	Diese Iteration ist einfach auf einem Rechner implementierbar.
	Das Verfahren heißt \emph{explizites Euler-Verfahren}.
	``Explizit'', da die Iterationsvorschrift direkt benutzt werden kann, um \(x_{j + 1}\) zu berechnen.

	Das explizite Euler-Verfahren lässt sich auch dahingehend interpretieren, als dass die Ableitung durch einen Differenzenquotienten ersetzt wurde.
	Aus
	\[
		\dot{x}\parentheses*{t} = f\parentheses*{t, x\parentheses*{t}}
	\]
	wird
	\[
		\frac{x_{j + 1} - x_j}{t_{j + 1} - t_j} = f\parentheses*{t_j, x_j}.
	\]
	Es sind auch andere Differenzenquotienten denkbar, z.B. der andere einseitige Quotient
	\[
		\frac{x_j - x_{j - 1}}{t_j - t_{j - 1}} = f\parentheses*{t_j, x_j},
	\]
	bzw. nach Indexverschiebung
	\[
		\frac{x_{j + 1} - x_j}{t_{j + 1} - t_j} = f\parentheses*{t_{j + 1}, x_{j + 1}} \quad \text{oder} \quad x_{j + 1} = x_j + \parentheses*{t_{j + 1} - t_j}f\parentheses*{t_{j + 1}, x_{j + 1}}.
	\]
	Dieses Verfahren heißt \emph{implizites Euler-Verfahren}.
	``Implizit'' daher, dass der neu zu bestimmende Wert \(x_{j + 1}\) auch in der rechten Seite \(f\) auftaucht, und daher ein Gleichungssystem gelöst werden muss.

	Eine weitere Möglichkeit, Verfahren herzuleiten, ist durch Quadraturformeln.
	Wir integrieren die gewöhnliche Differentiagleichung über das Intervall \(\brackets*{t_j, t_{j + 1}}\) und erhalten
	\[
		\int_{t_j}^{t_{j + 1}}\dot{x}\parentheses*{t}\d t = \int_{t_j}^{t_{j + 1}}f\parentheses*{t, x\parentheses*{t}}\d t \iff x\parentheses*{t_{j + 1}} - x\parentheses*{t_j} = \int_{t_j}^{t_{j + 1}}f\parentheses*{t, x\parentheses*{t}}\d t.
	\]
	Dann verwenden wir für die rechte Seite eine Quadraturformel, z.B. die Trapezregel
	\[
		x\parentheses*{t_{j + 1}} - x\parentheses*{t_j} \approx \frac{t_{j + 1} - t_j}{2}\parentheses*{f\parentheses*{t_j, x\parentheses*{t_j}} + f\parentheses*{t_{j + 1}, x\parentheses*{t_{j + 1}}}}.
	\]
	Es ist nicht überraschend, dass diese Methode Trapezverfahren heißt.
	Das Trapezverfahren ist ebenfalls implizit.

	\begin{example}
		Wir betrachten die einfache gewöhnliche Differentialgleichung \(\dot{x} = \lambda x\) it einem \(\lambda \in \R\).
		Wir wenden die obigen drei Verfahren an mit einer festen Schrittweite \(\tau\).
		Die impliziten Gleichungen kann man leicht auflösen und erhält
		\begin{align*}
			x_{j + 1} &= \parentheses*{1 + \tau\lambda}x_j, &&\text{explizites Eulerverfahren},\\
			x_{j + 1} &= \frac{1}{1 - \tau\lambda}x_j, &&\text{implizites Eulerverfahren},\\
			x_{j + 1} &= \frac{1 + \frac{\tau\lambda}{2}}{1 - \frac{\tau\lambda}{2}}x_j, &&\text{Trapezverfahren}.
		\end{align*}
	\end{example}
\end{document}
