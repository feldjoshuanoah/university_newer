\documentclass{exercise}

\DeclareMathOperator*{\cond}{cond}
\DeclareMathOperator*{\diag}{diag}
\DeclareMathOperator*{\dist}{dist}
\DeclareMathOperator*{\esssup}{ess\,sup}
\DeclareMathOperator*{\vol}{vol}


\title{Selbstrechenübung 2}
\author{Joshua Feld (406718)}
\professor{Prof. Torrilhon \& Dr. Speck}
\course{Mathematische Grundlagen III}

\begin{document}
	\maketitle


	\section{}

	\begin{quote}
		Consider a system of a finite number (say \(n\)) of point masses \(M_i\) (with respective masses \(m_i\)), located at positions \(\vec{x}\parentheses*{t} = \parentheses*{x_1\parentheses*{t}, \ldots, x_n\parentheses*{t}}\).
		We seek the critical points of the action
		\[
			I\parentheses*{\vec{x}} = \int_{t_0}^{t_1}L\parentheses*{t, \vec{x}\parentheses*{t}, \dot{\vec{x}}\parentheses*{t}}\d t \quad \text{with} \quad L\parentheses*{t, \vec{x}\parentheses*{t}, \dot{\vec{x}}\parentheses*{t}} = T\parentheses*{t, \vec{x}\parentheses*{t}, \dot{\vec{x}}\parentheses*{t}} - V\parentheses*{t, \vec{x}\parentheses*{t}, \dot{\vec{x}}\parentheses*{t}},
		\]
		where \(L: \brackets*{a, b} \times \R^{3 \times n} \times \R^{3 \times n} \to \R\) is the Lagrange function and \(T: \brackets*{a, b} \times \R^{3 \times n} \times \R^{3 \times n}\) and \(V: \brackets*{a, b} \times \R^{3 \times n} \times \R^{3 \times n}\) denotes the (total) kinetic and potential energy of the system respectively.
		Now, consider the simple system \(n = 2\) with the restriction that the two mass points are connected to each other by a wire of length \(l\) as inidcated in figure \ref{fig:1-1}.
		Hereby, \(M_1\) can move in the \(\parentheses*{x, y}\)-plane, which has a hole in the origin, through which the wire connects \(M_1\) to \(M_2\), which can move along the \(z\)-axis.
		\begin{enumerate}
			\item The kinetic energy of the system is given by \(T\parentheses*{x_1, x_2, \dot{x}_1, \dot{x}_2} = \frac{1}{2}m_1 \absolute*{\dot{x}_1}^2 + \frac{1}{2}m_2 \absolute*{\dot{x}_2}^2\).
			Rewrite this expression explicitly in cylindrical coordinates, i.e.
			\[
				x_i = \begin{pmatrix}
					x_i\\
					y_i\\
					z_i
				\end{pmatrix} = \begin{pmatrix}
					r_i \cos\phi_i\\
					r_i \sin\phi_i\\
					z_i
				\end{pmatrix}.
			\]
			\item The potential energy is given by \(V\parentheses*{x_1, x_2} = m_1 g\parentheses*{x_1^T \vec{e}_z} + m_2 g\parentheses*{x_2^T \vec{e}_z}\).
			Also rewrite this expression in cylindrical coordinates, i.e., as \(T\parentheses*{z_2, \dot{\phi}_i, \dot{z}_2}\).
			\item State the Euler-Lagrange equations, which are satisfied for critical points of the action \(I\parentheses*{\tilde{x}}\).
		\end{enumerate}
	\end{quote}

	\begin{enumerate}
		\item
		\item
		\item
	\end{enumerate}


	\section{}

	\begin{quote}
		Für welche Werte von \(\beta\) besitzt
		\[
			F\parentheses*{y} = \int_{-1}^1 \parentheses*{y^2 + \frac{2}{3}x^3 y' - 2xy'}\d x
		\]
		mindestens ein Extremum mit \(y\parentheses*{-1} = 0, y\parentheses*{1} = \beta\)?
	\end{quote}

	Eine Lösung erfüllt notwendigerweise die Euler-Lagrange Differentialgleichung
	\[
		0 = \frac{\partial f}{\partial y} - \frac{\d}{\d x}\frac{\partial f}{\partial y'} = 2y - \frac{\d}{\d x}\parentheses*{\frac{2}{3}x^3 - 2x} = 2y - 2x^2 + 2 = 2\parentheses*{y - x^2 + 1}
	\]
	woraus \(y = x^2 - 1\) folgt.
	Dann ist \(y\parentheses*{-1} = 0\) wie gefordert und \(\beta = y\parentheses*{1} = 0\).


	\section{}

	\begin{quote}
		Gegeben sei ein Einschrittverfahren mit
		\[
			\Phi_f\parentheses*{t, y, h} = af\parentheses*{t, y} + \frac{1}{4}f\parentheses*{t + bh, y + chf\parentheses*{t, y}}.
		\]
		\begin{enumerate}
			\item Geben Sie das Runge-Kutta Tableau für dieses Verfahren an.
			Ist das Verfahren explizit oder implizit?
			\item Für welche Werte von \(\parentheses*{a, b, c}\) ist die Konsistenzordnung des Verfahrens mindestens \(2\)?
			\item Existieren für die Parameter \(\parentheses*{a, b, c}\) Werte, so dass das Schema mit der speziellen Flussfunktion \(f\parentheses*{t, y} = t^2\) von dritter Ordnung ist?
		\end{enumerate}
	\end{quote}

	\begin{enumerate}
		\item
		\item
		\item
	\end{enumerate}


	\section{}

	\begin{quote}
		Sei \(f: \R^+ \times \R^d \to \R^d\) stetig differenzierbar und bezüglich \(y\) global Lipschitz-stetig.
		Ist \(y\) die eindeutige Lösung des Anfangswertproblems
		\[
			y'\parentheses*{t}, f\parentheses*{t, y}, \quad y\parentheses*{0} = y_0, \quad 0 \le t \le T,
		\]
		so erfüllen die Approximationen des Eulerverfahrens zu den Zeitpunkten \(t_i = ih \in \brackets*{0, T}\) die Abschätzung
		\begin{equation}
			\norm*{y\parentheses*{t_i} - y_i}_2 \le \frac{\parentheses*{1 + Lh}^i - 1}{2L}\norm*{y''}_{\brackets*{0, T}}h \le \frac{e^{LT} - 1}{2L}\norm*{y''}_{\brackets*{0, T}}h,
		\end{equation}
		wobei \(\norm*{y''}_{\brackets*{0, T}} = \max_{0 \le t \le T}\norm*{y''\parentheses*{t}}_2\).
		Wenn wir annehmen, dass durch die Rechnung mit Maschinenzahlen bei jedem Schritt ein zusätzlicher Rundungsfehler \(\epsilon > 0\) auftritt, so gilt für die Approximationen die Abschätzung
		\begin{equation}\label{eq:4-2}
			\norm*{y\parentheses*{t_i} - y_i}_2 \le \frac{\parentheses*{1 + Lh}^i - 1}{2L}\parentheses*{\norm*{y''}_{\brackets*{0, T}}h + 2\frac{\epsilon}{h}} \le \frac{e^{LT} - 1}{2L}\parentheses*{\norm*{y''}_{\brackets*{0, T}}h + 2\frac{\epsilon}{h}}.
		\end{equation}
		\begin{enumerate}
			\item Beweisen Sie die Abschätzung \eqref{eq:4-2} induktiv.
			Benutzen Sie dazu die Rekursionsformel
			\[
				\norm*{y_{i + 1} - y\parentheses*{t_{i + 1}}}_2 \le \parentheses*{1 + hL}\norm*{y_i - y\parentheses*{t_i}}_2 + \frac{1}{2}\norm*{y''}_{\brackets*{0, T}}h^2 + \epsilon
			\]
			für den Gesamtfehler des Euler-Verfahrens.
			\item Für welchen Zeitschritt \(h\) ist der abgeschätzte Fehler minimal?
		\end{enumerate}
	\end{quote}

	\begin{enumerate}
		\item
		\item
	\end{enumerate}
\end{document}
