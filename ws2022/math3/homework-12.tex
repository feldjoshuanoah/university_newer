\documentclass[english]{exercise}

\DeclareMathOperator{\HI}{HI}
\DeclareMathOperator{\IoU}{IoU}


\title{Hausaufgabe 11}
\author{René Dopichay (356986) \quad Joshua Feld (406718)\\Thilo Kloos (410343) \quad Shunta Takushima (430043)}
\professor{Prof. Torrilhon \& Dr. Speck}
\course{Mathematische Grundlagen III}

\begin{document}
	\maketitle


	\section{}

	\begin{quote}
		Verify Gauss' theorem for the region \(R\) bounded by the paraboloid \(P\) and the plane \(Z\)
		\[
			P = \braces*{\parentheses*{x, y, z} \in \R^3 : z = x^2 + y^2} \quad \text{and} \quad Z = \braces*{\parentheses*{x, y, z} \in \R^3 : z = 1}
		\]
		where the vector field \(F\) is given as follows
		\[
			F: \R^3 \to \R^3, \parentheses*{x, y, z} \mapsto F\parentheses*{x, y, z} := \parentheses*{xz, yz, 3z^2}^\top.
		\]
	\end{quote}

	The region \(R\) bounded by the paraboloid \(P\) and plane \(Z\) can be described in cylindrical coordinates as
	\[
		0 \le r \le 1, \quad 0 \le \theta \le 2\pi, \quad r^2 \le z \le 1.
	\]
	To verify Gauss' theorem we need to show that
	\[
		\iiint_R \parentheses*{\nabla \cdot F}\d\lambda_3 = \iint_{\partial R}F \cdot \d\sigma.
	\]
	We first compute the integral on the left-hand side.
	Because we transferred from cartesian to cylindrical coordinates, we need use \(\d x\d y\d z = r\d z\d\theta\d r\), with which we obtain
	\begin{align*}
		\iiint_R \parentheses*{\nabla \cdot F}\d\lambda_3 &= \int_0^1 \int_0^{2\pi}\int_{r^2}^1 \parentheses*{\frac{\partial F_x\parentheses*{x, y, z}}{\partial x} + \frac{\partial F_y\parentheses*{x, y, z}}{\partial y} + \frac{\partial F_z\parentheses*{x, y, z}}{\partial z}}r\d z\d\theta\d r\\
		&= \int_0^1 \int_0^{2\pi}\int_{r^2}^1 8zr\d z\d\theta\d r\\
		&= \int_0^1 \int_0^{2\pi}4r\parentheses*{1 - r^4}\d\theta\d r\\
		&= \int_0^1 8\pi r\parentheses*{1 - r^4}\d r\\
		&= \brackets*{4\pi r^2 \parentheses*{1 - \frac{1}{3}r^4}}_0^1 = \frac{8\pi}{3}.
	\end{align*}
	For the second integral we need to parametrize the disk and the paraboloid, of which the boundary \(\partial R\) is composed.
	We get
	\begin{align*}
		\Phi_Z\parentheses*{r, \theta} &= \parentheses*{r\cos\theta, r\sin\theta, 1}, \quad \nu_Z = \parentheses*{0, 0, 1}^\top,\\
		\Phi_P\parentheses*{r, \theta} &= \parentheses*{r\cos\theta, r\sin\theta, r^2}, \quad \nu_P = \parentheses*{2r^2 \cos\theta, 2r^2 \sin\theta, -r}^\top,
	\end{align*}
	with \(0 \le r \le 1, 0 \le \theta \le 2\pi\).
	Plugging this in yields
	\begin{align*}
		\iint_{\partial R}F \cdot \d\sigma &= \int_0^1 \int_0^{2\pi}\begin{pmatrix}
			0\\
			0\\
			3
		\end{pmatrix} \cdot \begin{pmatrix}
			0\\
			0\\
			1
		\end{pmatrix}r\d\theta\d r + \int_0^1 \begin{pmatrix}
			r^3 \cos\theta\\
			r^3 \sin\theta\\
			3r^4
		\end{pmatrix} \cdot \begin{pmatrix}
			2r^2 \cos\theta\\
			2r^2 \sin\theta\\
			-r
		\end{pmatrix}\d\theta\d r\\
		&= \int_0^1 \int_0^{2\pi}3r\d\theta\d r + \int_0^1 \int_0^{2\pi}\parentheses*{2r^5 \cos^2\parentheses*{\theta} + 2r^5 \sin^2\parentheses*{\theta} - 3r^5}\d\theta\d r\\
		&= 
 	\end{align*}



	\section{}

	\begin{quote}
		Prove the first and the second Green's identity by proceeding the following tasks
		\begin{enumerate}
			\item First, for \(V \subseteq \R^3, \phi \in C^1\parentheses*{V} f \in C^1\parentheses*{V; \R^3}\) prove the identity
			\[
				\div\parentheses*{\phi f} = \angles*{\nabla\phi, f} + \phi\div f.
			\]
			Then, deduce that for \(M \subset V\), we have that
			\[
				\int_M \phi\div f\d x = \int_{\partial M}\phi f \cdot \nu\d\sigma - \int_M \angles*{\nabla\phi, f}\d x.
			\]
			\item Now, by means of a) show that for \(\phi \in C^1\parentheses*{V}, \psi \in C^2\parentheses*{V}, M \subset V\), it holds that
			\[
				\int_M \phi\Delta\psi\d x = \int_{\partial M}\phi\frac{\partial\psi}{\partial\nu}\d\sigma - \int_M \angles*{\nabla\phi, \nabla\psi}\d x.
			\]
			The above is called the first Green's identity.
			\item Hence, infer from b) that for \(\phi \in C^2\parentheses*{V}, \psi \in C^2\parentheses*{V}, M \subset V\), it holds that
			\[
				\int_M \parentheses*{\phi\Delta\psi - \psi\Delta\phi}\d x = \int_{\partial M}\parentheses*{\phi\frac{\partial\psi}{\partial\nu} - \psi\frac{\partial\phi}{\partial\nu}}\d\sigma.
			\]
			The above is called the second Green's identity.
		\end{enumerate}
	\end{quote}


	\section{}

	\begin{quote}
		We will show that the standard Newton method for optimization
		\begin{equation}\label{eq:3-1}
			x^{\parentheses*{k + 1}} = x^{\parentheses*{k}} + \alpha_k d^{\parentheses*{k}}, \quad d^{\parentheses*{k}} = -\parentheses*{\nabla^2 f\parentheses*{x^{\parentheses*{k}}}}^{-1}\nabla f\parentheses*{x^{\parentheses*{k}}}
		\end{equation}
		will not converge for all starting points \(x^{\parentheses*{0}}\).
		Let \(f: \R \to \R, x \mapsto \sqrt{1 + x^2}\) be given.
		\begin{enumerate}
			\item Derive the sets of all starting points, for which the method \eqref{eq:3-1} with \(\alpha_k = 1\) converges.
			\item Set \(\alpha_k = \frac{1}{2}\) and perform three steps of the algorithm with starting point \(x^{\parentheses*{0}} = -\frac{3}{2}\).
			\item Interpret the results of a) and b).
		\end{enumerate}
	\end{quote}


	\section{}

	Consider
	\[
		f: \R^2 \to \R, \parentheses*{x, y} \mapsto f\parentheses*{x, y} := x^2 + xy + y^2.
	\]
	\begin{enumerate}
		\item Perform an approximate line search with the backtracking algorithm, which is a stepsize control algorithm with Armijo condition
		\begin{align*}
			\text{(i)} \quad & \text{Input} && x^{\parentheses*{k}}, \alpha_{\text{max}} > 0, \beta \in \parentheses*{0, 1}\\
			\text{(ii)} \quad & \text{Initialize} && \alpha_k = \alpha_{\text{max}}\\
			\text{(iii)} \quad & \text{While} && \text{(Armijo cond. for }\alpha_k\text{ not fulfilled)}\\
			& \text{(iii.1)} \quad \text{Decrease step size} && \alpha_k \gets \beta\alpha_k\\
			\text{(iv)} \quad & \text{Return} && \alpha_k
		\end{align*}
		on \(f\parentheses*{x, y}\) from \(x^{\parentheses*{0}} = \parentheses*{1, 1}^\top\) in the direction \(d^{\parentheses*{0}} = -\nabla f\parentheses*{x^{\parentheses*{0}}}\) using a first Wolfe condition parameter \(c_1 = 10^{-4}\).
		In each step, start with a maximum step size \(\alpha_{\text{max}} = 2\) and shrink the step size with the factor \(\beta = \frac{1}{2}\) until the Armijo condition is fulfilled.
		\item Is the second Wolfe condition with \(c_2 = 0.9\) for \(\alpha_0 = \frac{1}{2}\) fulfilled in the setup of a)?
		\item Imagine that the second Wolfe condition in b) would not be fulfilled.
		Which parameter in a) would you change, such that the backtracking algorithm returns an \(\alpha\) that satisfies the second Wolfe condition?
	\end{enumerate}
\end{document}