\documentclass{exercise}

\DeclareMathOperator*{\cond}{cond}
\DeclareMathOperator*{\diag}{diag}
\DeclareMathOperator*{\dist}{dist}
\DeclareMathOperator*{\esssup}{ess\,sup}
\DeclareMathOperator*{\vol}{vol}


\title{Hausaufgabe 0}
\author{René Dopichay (356986) \quad Joshua Feld (406718)\\Thilo Kloos (410343) \quad Shunta Takushima (430043)}
\professor{Prof. Torrilhon \& Dr. Speck}
\course{Mathematische Grundlagen III}

\begin{document}
	\maketitle


	\section{}

	\begin{enumerate}
		\item
		\begin{align*}
			\partial F_1\parentheses*{y; v} &= \lim_{t \to 0}\frac{1}{t}\parentheses*{F_1\parentheses*{y + tv} - F_1\parentheses*{y}}\\
			&= \lim_{t \to 0}\frac{1}{t}\parentheses*{\sum_{i = 1}^n \parentheses*{y + tv}^2\parentheses*{x_i} - \sum_{i = 1}^n y^2\parentheses*{x_i}}\\
			&= \lim_{t \to 0}\frac{1}{t}\sum_{i = 1}^n \parentheses*{\parentheses*{y\parentheses*{x_i} + tv\parentheses*{x_i}}^2 - y^2\parentheses*{x_i}}\\
			&= \lim_{t \to 0}\frac{1}{t}\sum_{i = 1}^n \parentheses*{2ty\parentheses*{x_i}v\parentheses*{x_i} + t^2 v^2\parentheses*{x_i}}\\
			&= \sum_{i = 1}^n 2y\parentheses*{x_i}v\parentheses*{x_i}
		\end{align*}
		\item
		\begin{enumerate}
			\item
			\begin{align*}
				\partial F_2\parentheses*{y; v} &= \lim_{t \to 0}\frac{1}{t}\parentheses*{F_2\parentheses*{y + tv} - F_2\parentheses*{y}}\\
				&= \lim_{t \to 0}\frac{1}{t}\parentheses*{\frac{\parentheses*{y + tv}\parentheses*{\pi}}{\parentheses*{y + tv}^2\parentheses*{0} + 1} - \frac{y\parentheses*{\pi}}{y^2\parentheses*{0} + 1}}\\
				&= \lim_{t \to 0}\frac{1}{t}\parentheses*{\frac{y\parentheses*{\pi} + tv\parentheses*{\pi}}{y^2\parentheses*{0} + 2ty\parentheses*{0}v\parentheses*{0} + t^2 v^2\parentheses*{0} + 1} - \frac{y\parentheses*{\pi}}{y^2\parentheses*{0} + 1}}\\
				&\stackrel{v = \cos}{=}\lim_{t \to 0}\frac{1}{t}\parentheses*{\frac{y\parentheses*{\pi} - t}{y^2\parentheses*{0} + 2ty\parentheses*{0} + t^2 + 1} - \frac{y\parentheses*{\pi}}{y^2\parentheses*{0} + 1}}\\
				&= \lim_{t \to 0}\frac{1}{t}\frac{\parentheses*{y\parentheses*{\pi} - t}\parentheses*{y^2\parentheses*{0} + 1} - y\parentheses*{\pi}\parentheses*{y^2\parentheses*{0} + 2ty\parentheses*{0} + t^2 + 1}}{\parentheses*{y^2\parentheses*{0} + 2ty\parentheses*{0} + t^2 + 1}\parentheses*{y^2\parentheses*{0} + 1}}\\
				&= -\lim_{t \to 0}\frac{1}{t}\frac{t\parentheses*{y^2\parentheses*{0} + 1 + 2y\parentheses*{\pi}y\parentheses*{0}} + t^2 y\parentheses*{\pi}}{\parentheses*{y^2\parentheses*{0} + 2ty\parentheses*{0} + t^2 + 1}\parentheses*{y^2\parentheses*{0} + 1}}\\
				&= -\frac{y^2\parentheses*{0} + 1 + 2y\parentheses*{\pi}y\parentheses*{0}}{\parentheses*{y^2\parentheses*{0} + 1}^2}.
			\end{align*}
			\item
			\[
				\partial F_2\parentheses*{\sin; \cos} = -\frac{0 + 1 + 2 \cdot 0 \cdot 0}{\parentheses*{0 + 1}^2} = -1.
			\]
		\end{enumerate}
		\item
		\begin{align*}
			F_3\parentheses*{y; v} &= \lim_{t \to 0}\frac{1}{t}\parentheses*{F_3\parentheses*{y + tv} - F_3\parentheses*{y}}\\
			&= \lim_{t \to 0}\frac{1}{t}\parentheses*{2\pi\int_a^b \parentheses*{y + tv}\sqrt{1 + \parentheses*{y + tv}'^2}\d x - 2\pi\int_a^b y\sqrt{1 + y'^2}\d x}\\
			&= 2\pi\lim_{t \to 0}\frac{1}{t}\int_a^b \parentheses*{\parentheses*{y + tv}\sqrt{1 + \parentheses*{y + tv}'^2} - y\sqrt{1 + y'^2}}\d x\\
			&= 2\pi\lim_{t \to 0}\frac{1}{t}\int_a^b \parentheses*{y\parentheses*{\sqrt{1 + y'^2 + 2ty'v' + t^2 v'^2} - \sqrt{1 + y'^2}} + tv\sqrt{1 + y'^2 + 2ty'v' + t^2 v'^2}}\d x\\
			&= 2\pi\lim_{t \to 0}\int_a^b \parentheses*{y\frac{2y'v' + tv'}{\sqrt{1 + y'^2 + 2ty'v' + t^2 v'^2} + \sqrt{1 + y'^2}} + v\sqrt{1 + y'^2 + 2ty'v' + t^2 v'^2}}\d x\\
			&= 2\pi\int_a^b \parentheses*{y\frac{y'v'}{\sqrt{1 + y'^2}} + v\sqrt{1 + y'^2}}\d x.
		\end{align*}
	\end{enumerate}


	\section{}

	\begin{enumerate}
		\item
		\begin{align*}
			\partial F\parentheses*{y; v} &= \lim_{t \to 0}\frac{1}{t}\parentheses*{\int_{-1}^1 \parentheses*{y\parentheses*{x} + tx^2}^2 \d x + \int_{-1}^1 \parentheses*{y\parentheses*{x} + tx^2}'^2 \d x - \int_{-1}^1 y^2\parentheses*{x}\d x - \int_{-1}^1 y'^2\parentheses*{x}\d x}\\
			&= \lim_{t \to 0}\frac{1}{t}\parentheses*{\int_{-1}^1 \parentheses*{y^2\parentheses*{x} + 2ty\parentheses*{x}x^2 + t^2 x^4}\d x - \int_{-1}^1 y^2\parentheses*{x}\d x + \int_{-1}^1 \parentheses*{y'\parentheses*{x} + 2tx}^2 \d x - \int_{-1}^1 y'^2\parentheses*{x}\d x}\\
			&= \lim_{t \to 0}\frac{1}{t}\parentheses*{\int_{-1}^1 \parentheses*{2ty\parentheses*{x}x^2 + t^2 x^4}\d x + \int_{-1}^1 \parentheses*{4ty'\parentheses*{x}x + 4t^2 x^2}\d x}\\
			&= \int_{-1}^1 2y\parentheses*{x}x^2 \d x + \int_{-1}^1 4y'\parentheses*{x}x\d x\\
			&= \int_{-1}^1 2x\parentheses*{xy\parentheses*{x} + 2y'\parentheses*{x}}\d x.
		\end{align*}
		\item
		\begin{align*}
			\partial F\parentheses*{y; v} &= \lim_{t \to 0}\frac{1}{t}\parentheses*{\int_0^1 \parentheses*{y\parentheses*{x} + t\cos\parentheses*{x}}\parentheses*{y\parentheses*{x} + t\cos\parentheses*{x}}''\d x - \int_0^1 y\parentheses*{x}y''\parentheses*{x}\d x}\\
			&= \lim_{t \to 0}\frac{1}{t}\int_0^1 \parentheses*{\parentheses*{y\parentheses*{x} + t\cos\parentheses*{x}}\parentheses*{y''\parentheses*{x} - t\cos\parentheses*{x}} - y\parentheses*{x}y''\parentheses*{x}}\d x\\
			&= \lim_{t \to 0}\frac{1}{t}\int_0^1 \parentheses*{t\parentheses*{-y\parentheses*{x}\cos\parentheses*{x} + \cos\parentheses*{x}y''\parentheses*{x}} - t^2 \cos^2\parentheses*{x}}\d x\\
			&= \int_0^1 \parentheses*{y''\parentheses*{x} - y\parentheses*{x}}\cos\parentheses*{x}\d x.
		\end{align*}
	\end{enumerate}


	\section{}

	For the exact solution we use a Taylor expansion
	\[
		y\parentheses*{t + h} = y\parentheses*{t} + hy'\parentheses*{t} + \frac{1}{2}h^2 y''\parentheses*{t} + \mathcal{O}\parentheses*{h^3},
	\]
	where \(y'\parentheses*{t} = f\parentheses*{t, y\parentheses*{t}}\) and thus
	\[
		y''\parentheses*{t} = \frac{\d}{\d t}f\parentheses*{t, y\parentheses*{t}} = f_t\parentheses*{t, y\parentheses*{t}} + f_y\parentheses*{t, y\parentheses*{t}}y'\parentheses*{t} = f_t\parentheses*{t, y\parentheses*{t}} + f_y\parentheses*{t, y\parentheses*{t}}f\parentheses*{t, y\parentheses*{t}}.
	\]
	We can now rewrite the exact solution as
	\[
		y\parentheses*{t + h} = y\parentheses*{t} + hf\parentheses*{t, y\parentheses*{t}} + \frac{1}{2}h^2 \parentheses*{f_t\parentheses*{t, y\parentheses*{t}} + f_y\parentheses*{t, y\parentheses*{t}}f\parentheses*{t, y\parentheses*{t}}} + \mathcal{O}\parentheses*{h^3}
	\]
	and the approximation using the explicit method as
	\[
		y^{j + 1} = y^j + h\Psi\parentheses*{h},
	\]
	where
	\begin{align*}
		\Psi\parentheses*{h} &= \frac{1}{2}\parentheses*{f\parentheses*{t, y\parentheses*{t}} + f\parentheses*{t + h, y\parentheses*{t} + hf\parentheses*{t, y\parentheses*{t}}}}\\
		\Psi'\parentheses*{h} &= \frac{1}{2}f_t\parentheses*{t + h, y\parentheses*{t} + hf\parentheses*{t, y\parentheses*{t}}} + \frac{1}{2}f_y\parentheses*{t + h, y\parentheses*{t} + hf\parentheses*{t, y\parentheses*{t}}}f\parentheses*{t, y\parentheses*{t}}.
	\end{align*}
	With
	\[
		\Psi\parentheses*{0} = f\parentheses*{t, y\parentheses*{t}}, \quad \Psi'\parentheses*{0} = \frac{1}{2}f_t\parentheses*{t, y\parentheses*{t}} + \frac{1}{2}f_y\parentheses*{t, y\parentheses*{t}}f\parentheses*{t, y\parentheses*{t}}
	\]
	we can now expand \(\Psi\parentheses*{h}\) around \(0\):
	\begin{align*}
		\Psi\parentheses*{h} &= \Psi\parentheses*{0} + h\Psi'\parentheses*{0} + \mathcal{O}\parentheses*{h^2}\\
		&= f\parentheses*{t, y\parentheses*{t}} + \frac{1}{2}h\parentheses*{f_t\parentheses*{t, y\parentheses*{t}} + f_y\parentheses*{t, y\parentheses*{t}}f\parentheses*{t, y\parentheses*{t}}} + \mathcal{O}\parentheses*{h^2}.
	\end{align*}
	Alltogether the approximation is
	\[
		y^{j + 1} = y^j + hf\parentheses*{t, y^j} + \frac{1}{2}h^2 \parentheses*{f_t\parentheses*{t, y^j} + f_y\parentheses*{t, y^j}f\parentheses*{t, y^j}} + \mathcal{O}\parentheses*{h^3}.
	\]
	and the error is
	\[
		\absolute*{y\parentheses*{t + h} - y^{j + 1}} = \mathcal{O}\parentheses*{h^3},
	\]
	which gives us an order of \(2\) for the given explicit method.


	\section{}

	\begin{enumerate}
		\item Wir definieren
		\[
			y_0\parentheses*{t} = y\parentheses*{t}, \quad y_1\parentheses*{t} = y'\parentheses*{t}.
		\]
		Dann können wir das Problem auf das ein System gewöhnlicher Differentialgleichungen erster Ordnung transformieren
		\begin{align*}
			y_0'\parentheses*{t} &= y'\parentheses*{t},\\
			y_1'\parentheses*{t} &= ty_0\parentheses*{t} - y_1\parentheses*{t}
		\end{align*}
		für \(t \in \brackets*{0, \frac{1}{2}}\) mit den Randbedingungen
		\[
			y_0\parentheses*{0} = 0, \quad y_1\parentheses*{0} = 1.
		\]
		\item Es gilt
		\begin{align*}
			\phi_0 &= \begin{pmatrix}
				0\\
				1
			\end{pmatrix},\\
			\phi_1 &= \begin{pmatrix}
				0\\
				1
			\end{pmatrix} + h\begin{pmatrix}
				1 - 0\\
				h \cdot 0 - 1
			\end{pmatrix} = \begin{pmatrix}
				h\\
				1 - h
			\end{pmatrix},\\
			\phi_2 &= \begin{pmatrix}
				h\\
				1 - h
			\end{pmatrix} + h\begin{pmatrix}
				1 - h\\
				h \cdot h - \parentheses*{1 - h}
			\end{pmatrix} = \begin{pmatrix}
				2h - h^2\\
				1 - 2h + h^2 + h^3
			\end{pmatrix}.
		\end{align*}
		Für die Bestimmung einer Näherung von \(y\parentheses*{\frac{1}{2}}\) in zwei Schritten, berechnen wir (mit \(h = \frac{1}{4}\)) dann
		\[
			\phi_0 = \begin{pmatrix}
				y_0\parentheses*{0}\\
				y_1\parentheses*{0}
			\end{pmatrix} = \begin{pmatrix}
				0\\
				1
			\end{pmatrix}, \quad \phi_1 = \begin{pmatrix}
				y_0\parentheses*{\frac{1}{4}}\\
				y_1\parentheses*{\frac{1}{4}}
			\end{pmatrix} = \begin{pmatrix}
				\frac{1}{4}\\
				\frac{3}{4}
			\end{pmatrix}, \quad \phi_2 = \begin{pmatrix}
				y_0\parentheses*{\frac{1}{2}}\\
				y_1\parentheses*{\frac{1}{2}}
			\end{pmatrix} = \begin{pmatrix}
				\frac{7}{16}\\
				\frac{37}{64}
			\end{pmatrix}
		\]
		und erhalten \(y\parentheses*{\frac{1}{2}} = y_0\parentheses*{\frac{1}{2}} = \frac{7}{16}\).
		\item Der lokale Abbruchfehler gibt die Differenz zwischen dem exakten Wert und dem errechneten Wert an der Stelle \(t_{j + 1}\) an, wenn der Wert an der Stelle \(t_j\) richtig ist.
		Da das im Allgmeinen nicht der Fall sein muss, gibt der lokale Fehler die Abweichung vom exakten Wert an, die nur durch den letzten Schritt verursacht wird.
		Er ist für dieses Problem gegeben durch
		\[
			\delta_{j, h} = y\parentheses*{t_{j + 1}} - y_h\parentheses*{t_{j + 1}; t_j, y\parentheses*{t_j}} = y\parentheses*{t_{j + 1}} - y\parentheses*{t_j} - h\Psi_f\parentheses*{t_j, y\parentheses*{t_j}, h}.
		\]
		Aus dieser Beziehung erhalten wir
		\[
			y\parentheses*{t_{j + 1}} = y\parentheses*{t_j} + h\Psi_f\parentheses*{t_j, y\parentheses*{t_j}, h} + \delta_{j, h}.
		\]
		Für den globalen Fehler \(e_j := y\parentheses*{t_j} - y^j\) ergibt sich die Rekursion
		\[
			e_{j + 1} = \parentheses*{1 + h\lambda}e_j + \delta_{j, h}, \quad j = 0, \ldots, n - 1.
		\]
		Es gilt \(e_0 = 0\).
		Daher erhalten wir
		\begin{align*}
			e_1 &= \parentheses*{1 + h\lambda}e_0 + \delta_{0, h} = \delta_{0, h},\\
			e_2 &= \parentheses*{1 + h\lambda}e_1 + \delta_{1, h} = \parentheses*{1 + h\lambda}\delta_{0, h} + \delta_{1, h},\\
			e_3 &= \parentheses*{1 + h\lambda}e_2 + \delta_{2, h} = \parentheses*{1 + h\lambda}^2 \delta_{0, h} + \parentheses*{1 + h\lambda}\delta_{1, h} + \delta_{2, h},\\
			&\hspace{4cm}\vdots\\
			e_n &= \parentheses*{1 + h\lambda}e_{n - 1} + \delta_{n - 1, h} = \sum_{i = 0}^{k - 1}\parentheses*{1 + h\lambda}^i \delta_{n - 1 - i, h}.
		\end{align*}
		Der maximale globale Fehler entsteht also durch eine Akkumulation der lokalen Fehler
		\[
			\max_{j = 0, \ldots, n}\absolute*{y\parentheses*{t_j} - y^j} \le \sum_{i = 0}^{j - 1}\parentheses*{1 + h\lambda}^i \absolute*{\delta_{j - 1 - i, h}}.
		\]
	\end{enumerate}
\end{document}
