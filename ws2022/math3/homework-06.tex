\documentclass{exercise}

\DeclareMathOperator*{\cond}{cond}
\DeclareMathOperator*{\diag}{diag}
\DeclareMathOperator*{\dist}{dist}
\DeclareMathOperator*{\esssup}{ess\,sup}
\DeclareMathOperator*{\vol}{vol}


\title{Hausaufgabe 6}
\author{René Dopichay (356986) \quad Joshua Feld (406718)\\Thilo Kloos (410343) \quad Shunta Takushima (430043)}
\professor{Prof. Torrilhon \& Dr. Speck}
\course{Mathematische Grundlagen III}

\begin{document}
	\maketitle


	\section{}

	\begin{quote}
		Use the change of variables \(u = x - y\) and \(v = x + y\) to evaluate the integral
		\begin{equation}
			\iint_R \parentheses*{x - y}e^{x^2 - y^2}\d A
		\end{equation}
		where \(R\) is the region bounded by the lines \(x + y = 1\) and \(x + y = 3\) and the curves \(x^2 - y^2 = -1\) and \(x^2 - y^2 = 1\).
	\end{quote}

	First we solve the given transformation for \(x\) and \(y\):
	\[
		u = x - y \iff x = u + y \iff x = \frac{1}{2}\parentheses*{u + v}, \quad v = x + y \iff y = v - x \iff y = \frac{1}{2}\parentheses*{v - u}.
	\]
	Thus the transformation function \(\Phi\) is given by
	\[
		\Phi\parentheses*{u, v} = \parentheses*{\frac{1}{2}\parentheses*{u + v}, \frac{1}{2}\parentheses*{v - u}}.
	\]
	For the integrand we obtain
	\[
		\parentheses*{x - y}e^{x^2 - y^2} = \parentheses*{x - y}e^{\parentheses*{x - y}\parentheses*{x + y}} = ue^{uv}.
	\]
	The transformation also changes the boundaries of the integrals for which we get
	\begin{align*}
		x^2 - y^2 = -1 &\iff u = -\frac{1}{v}, & x + y = 1 &\iff v = 1,\\
		x^2 - y^2 = 1 &\iff u = \frac{1}{v}, & x + y = 3 &\iff v = 3.
	\end{align*}
	Lastly \(\d x\d y\) changes to \(\det D\Phi\parentheses*{u, v}\d u\d v\), where \(D\Phi\parentheses*{u, v}\) is the Jacobian of the transformation function.
	The Jacobian determinant thus solves to
	\[
		\det D\Phi\parentheses*{u, v} = \begin{vmatrix}
			\frac{\partial\Phi_1}{\partial u} & \frac{\partial\Phi_1}{\partial v}\\
			\frac{\partial\Phi_2}{\partial u} & \frac{\partial\Phi_2}{\partial v}
		\end{vmatrix} = \begin{vmatrix}
			\frac{1}{2} & \frac{1}{2}\\
			-\frac{1}{2} & \frac{1}{2}
		\end{vmatrix} = \frac{1}{2}.
	\]
	Finally, we can evaluate the transformed integral in \(uv\)-space as
	\begin{align*}
		\iint_R \parentheses*{x - y}e^{x^2 - y^2}\d A &= \frac{1}{2}\int_1^3 \int_{-\frac{1}{v}}^{\frac{1}{v}} ue^{uv}\d u\d v\\
		&= \frac{1}{2}\int_1^3 \brackets*{\frac{uv - 1}{v^2}e^{uv}}_{-\frac{1}{v}}^{\frac{1}{v}}\d v\\
		&= \frac{1}{2}\int_1^3 \frac{2}{ev^2}\d v\\
		&= -\brackets*{\frac{1}{ev}}_1^3 = \frac{2}{3e}.
	\end{align*}


	\section{}

	\begin{quote}
		Transform the triple integral
		\begin{equation}
			\int_0^3 \int_0^4 \int_{\frac{y}{2}}^{\frac{y}{2} + y}\frac{3x + z}{3}\d x\d y\d z
		\end{equation}
		in \(xyz\)-space to \(uvw\)-space by using the transformation \(u = \frac{2x - y}{2}\), \(v = \frac{y}{2}\), and \(w = \frac{z}{3}\) and then evaluate the transformed integral in \(uvw\)-space.
	\end{quote}

	The approach is analogous to the one we had in the first exercise.
	First we solve the given transformations for \(x\), \(y\) and \(z\):
	\[
		u = \frac{2x - y}{2} \iff x = u + v, \quad v = \frac{y}{2} \iff y = 2v, \quad w = \frac{z}{3} \iff z = 3w.
	\]
	Thus the transformation function \(\Phi\) is given by
	\[
		\Phi\parentheses*{u, v, w} = \parentheses*{u + v, 2v, 3w}.
	\]
	For the integrand we obtain
	\[
		\frac{3x + z}{3} = \frac{3\parentheses*{u + v} + 3w}{3} = u + v + w.
	\]
	The transformation also changes the boundaries of the integrals for which we get
	\begin{align*}
		x = \frac{y}{2} &\iff u = 0, & y = 0 &\iff v = 0, & z = 0 &\iff w = 0,\\
		x = \frac{y}{2} + y &\iff u = 2v, & y = 4 &\iff v = 2, & z = 3 &\iff w = 1.
	\end{align*}
	Lastly \(\d x\d y\d z\) changes to \(\det D\Phi\parentheses*{u, v, w}\d u\d v\d w\), where \(D\Phi\parentheses*{u, v, w}\) is the Jacobian of the transformation function.
	The Jacobian determinant thus solves to
	\[
		\det D\Phi\parentheses*{u, v, w} = \begin{vmatrix}
			\frac{\partial \Phi_1}{\partial u} & \frac{\partial \Phi_1}{\partial v} & \frac{\partial \Phi_1}{\partial w}\\
			\frac{\partial \Phi_2}{\partial u} & \frac{\partial \Phi_2}{\partial v} & \frac{\partial \Phi_2}{\partial w}\\
			\frac{\partial \Phi_3}{\partial u} & \frac{\partial \Phi_3}{\partial v} & \frac{\partial \Phi_3}{\partial w}\\
		\end{vmatrix} = \begin{vmatrix}
			1 & 1 & 0\\
			0 & 2 & 0\\
			0 & 0 & 3
		\end{vmatrix} = 6.
	\]
	Finally, we can evalute the transformed integral in \(uvw\)-space as
	\begin{align*}
		\int_0^3 \int_0^4 \int_{\frac{y}{2}}^{\frac{y}{2} + y}\frac{3x + z}{3}\d x\d y\d z &= 6\int_0^1 \int_0^2 \int_0^{2v}\parentheses*{u + v + w}\d u\d v\d w\\
		&= 6\int_0^1 \int_0^2 \brackets*{\frac{1}{2}u^2 + vu + wu}_0^{2v}\d v\d w\\
		&= 12\int_0^1 \int_0^2 v\parentheses*{2v + w}\d v\d w\\
		&= 12\int_0^1 \brackets*{v^2\parentheses*{\frac{2}{3}v + \frac{w}{2}}}_0^2 \d w\\
		&= 24\int_0^1 \parentheses*{w + \frac{8}{3}}\d w\\
		&= 4 \cdot \brackets*{w\parentheses*{3w + 16}}_0^1 = 76
	\end{align*}


	\section{}

	\begin{quote}
		Proof the following statements:
		\begin{enumerate}
			\item Every eigenvalue of a Hermitian \(n \times n\)-matrix \(H\) is real.
			\item For every unitary matrix \(U\) it holds for every eigenvalue \(\lambda\): \(\absolute*{\lambda} = 1\).
		\end{enumerate}
	\end{quote}

	\begin{enumerate}
		\item Let \(\lambda \in \C\) be an arbitrary eigenvalue of the Hermitian matrix \(H\) and let \(v\) be an eigenvector corresponding to the eigenvalue \(\lambda\).
		Then we have
		\[
			Hv = \lambda v \iff \bar{v}^T \parentheses*{Hv} = \bar{v}^T \parentheses*{\lambda v} = \lambda\bar{v}^T v = \lambda\norm*{v}.
		\]
		We also have \(\bar{v}^T \parentheses*{Hv} = \parentheses*{Hv}^T \bar{v} = v^T H^T \bar{v}\).
		Thus we obtain
		\[
			v^T H^T \bar{v} = \lambda\norm*{v} \iff \bar{v}^T \bar{H}^T v = \bar{\lambda}\norm*{v}.
		\]
		Since \(H\) is Hermitian, we have \(\bar{H}^T = H\), which yields that
		\[
			\bar{\lambda}\norm*{v} = \bar{v}^T Hv = \bar{v}^T \lambda v = \lambda\norm*{v}.
		\]
		Recall that \(v\) is an eigenvector, hence \(v\) is not the zero vector and \(\norm*{v} \ne 0\).
		Therefore, we divide by the length \(\norm*{v}\) and get \(\lambda = \bar{\lambda}\). which means that \(\lambda\) has to be real.
		\item Let \(\lambda \in \C\) be an arbitrary eigenvalue fo the unitary matrix \(U\) and let \(v\) be an eigenvector corresponding to the eigenvalue \(\lambda\).
		Then we have
		\[
			Uv = \lambda v \quad \text{and} \quad \bar{v}^T \bar{U}^T = \bar{\lambda}\bar{v}^T.
		\]
		Let us multiply these equalities:
		\[
			\bar{v}^T \bar{U}^T Uv = \bar{\lambda}\bar{v}^T \lambda v = \absolute*{\lambda}^2 \norm*{v}^2,
		\]
		but since \(\bar{U}^T U = I\), the left-hand side is \(\norm*{v}^2\). Sice \(v\) is an eigenvector and thus \(v\) is not the zero vector, we have \(\norm*{v} \ne 0\). This implies \(\absolute*{\lambda} = 1\).
	\end{enumerate}


	\section{}

	\begin{quote}
		Compute \(A^T A\) and \(AA^T\) and their eigenvalues and unit eigenvectors \(V\) and \(U\):
		\begin{equation}
			A = \begin{pmatrix}
				1 & 1 & 0\\
				0 & 1 & 1
			\end{pmatrix}.
		\end{equation}
		Use this to state the SVD of the matrix \(A\).
	\end{quote}

	We calculate
	\[
		A^T A = \begin{pmatrix}
			1 & 1 & 0\\
			1 & 2 & 1\\
			0 & 1 & 1
		\end{pmatrix}, \quad AA^T = \begin{pmatrix}
			2 & 1\\
			1 & 2
		\end{pmatrix}.
	\]
	First we compute the singular values \(\sigma_i\) by finding the eigenvalues \(\lambda_i\) of \(AA^T\).
	The characteristic polynomial of \(AA^T\) is
	\[
		\chi_{AA^T}\parentheses*{\lambda} = \det\parentheses*{AA^T - \lambda I} = \begin{vmatrix}
			2 - \lambda & 1\\
			1 & 2 - \lambda
		\end{vmatrix} = \parentheses*{2 - \lambda}^2 - 1 = \parentheses*{\lambda - 3}\parentheses*{\lambda - 1}.
	\]
	Thus the singular values are \(\sigma_1 = \sqrt{3}\) and \(\sigma_2 = 1\).
	Now we find the right singular vectors (the columns of \(V\)) by finding an orthonormal set of eigenvectors of \(A^T A\).
	It is also possible to proceed by finding the left singular vectors (columns of \(U\)) instead.
	The eigenvalues of \(A^T A\) are \(\lambda_1 = 3\), \(\lambda_2 = 1\), and \(\lambda_3 = 0\), and since \(A^T A\) is symmetric we know that the eigenvectors will be orthogonal.
	For \(\lambda_1 = 3\), we have
	\[
		A^T A - 3I = \begin{pmatrix}
			-2 & 1 & 0\\
			1 & -1 & 1\\
			0 & 1 & -2
		\end{pmatrix}
	\]
	which row-reduces to \(\begin{pmatrix}
		1 & 0 & -1\\
		0 & 1 & -2\\
		0 & 0 & 0
	\end{pmatrix}\).
	A unit-length vector in the kernel of that matrix is \(v_1 = \begin{pmatrix}
		\frac{1}{\sqrt{6}}\\
		\sqrt{\frac{2}{3}}\\
		\frac{1}{\sqrt{6}}
	\end{pmatrix}\).
	For \(\lambda_2 = 1\) we have \(A^T A - I = \begin{pmatrix}
		0 & 1 & 0\\
		1 & 1 & 1\\
		0 & 1 & 0
	\end{pmatrix}\) which row-reduces to \(\begin{pmatrix}
		1 & 0 & 1\\
		0 & 1 & 0\\
		0 & 0 & 0
	\end{pmatrix}\).
	A unit-length vector in the kernel is \(v_2 = \begin{pmatrix}
		-\frac{1}{\sqrt{2}}\\
		0\\
		\frac{1}{\sqrt{2}}
	\end{pmatrix}\).
	For the last eigenvector we have \(A^T A\) which row-reduces to \(\begin{pmatrix}
		1 & 0 & -1\\
		0 & 1 & 1\\
		0 & 0 & 0
	\end{pmatrix}\). A unit-length vector in the kernel is \(v_3 = \begin{pmatrix}
		\frac{1}{\sqrt{3}}\\
		-\frac{1}{\sqrt{3}}\\
		\frac{1}{\sqrt{3}}
	\end{pmatrix}\).
	So at this point we know that
	\[
		A = U\Sigma V^T = U\begin{pmatrix}
			3 & 0 & 0\\
			0 & 1 & 0
		\end{pmatrix}\begin{pmatrix}
			\frac{1}{\sqrt{6}} & \sqrt{\frac{2}{3}} & \frac{1}{\sqrt{6}}\\
			-\frac{1}{\sqrt{2}} & 0 & \frac{1}{\sqrt{2}}\\
			\frac{1}{\sqrt{3}} & -\frac{1}{\sqrt{3}} & \frac{1}{\sqrt{3}}
		\end{pmatrix}.
	\]
	Finally, we can compute \(U\) by the formula \(\sigma u_i = A v_i\) or \(u_i = \frac{1}{\sigma}Av_i\).
	This gives \(U = \begin{pmatrix}
		\frac{1}{\sqrt{2}} & -\frac{1}{\sqrt{2}}\\
		\frac{1}{\sqrt{2}} & \frac{1}{\sqrt{2}}
	\end{pmatrix}\).
	So in its full glory the SVD is
	\[
		A = U\Sigma V^T = \begin{pmatrix}
			\frac{1}{\sqrt{2}} & -\frac{1}{\sqrt{2}}\\
			\frac{1}{\sqrt{2}} & \frac{1}{\sqrt{2}}
		\end{pmatrix}\begin{pmatrix}
			3 & 0 & 0\\
			0 & 1 & 0
		\end{pmatrix}\begin{pmatrix}
			\frac{1}{\sqrt{6}} & \sqrt{\frac{2}{3}} & \frac{1}{\sqrt{6}}\\
			-\frac{1}{\sqrt{2}} & 0 & \frac{1}{\sqrt{2}}\\
			\frac{1}{\sqrt{3}} & -\frac{1}{\sqrt{3}} & \frac{1}{\sqrt{3}}
		\end{pmatrix}.
	\]
\end{document}
