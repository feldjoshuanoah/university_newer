\documentclass{exercise}

\DeclareMathOperator*{\cond}{cond}
\DeclareMathOperator*{\diag}{diag}
\DeclareMathOperator*{\dist}{dist}
\DeclareMathOperator*{\esssup}{ess\,sup}
\DeclareMathOperator*{\vol}{vol}


\title{Hausaufgabe 8}
\author{René Dopichay (356986) \quad Joshua Feld (406718)\\Thilo Kloos (410343) \quad Shunta Takushima (430043)}
\professor{Prof. Torrilhon \& Dr. Speck}
\course{Mathematische Grundlagen III}

\begin{document}
	\maketitle


	\section{}

	\begin{quote}
		\begin{enumerate}
			\item Gegeben sei die Funktion \(f\) wie folgt.
			Zeigen Sie, dass \(f \not\in L^1\parentheses*{\Omega}\) aber \(f \in L^2\parentheses*{\Omega}\).
			\[
				f: \Omega := \parentheses*{0, \infty} \to \R, x \mapsto f\parentheses*{x} := \begin{cases}
					1, & \text{falls }0 \le x \le 1,\\
					x^{-\frac{2}{3}}, & \text{falls }x > 1.
				\end{cases}
			\]
			\item Gegeben sei die Funktion \(g\) wie folgt.
			Zeigen Sie dass \(g \in L^1\parentheses*{\Omega}\) aber \(g \not\in L^2\parentheses*{\Omega}\).
			\[
				g: \Omega := \parentheses*{0, \infty} \to \R, x \mapsto g\parentheses*{x} := \begin{cases}
					x^{-\frac{3}{4}}, & \text{falls }0 < x \le 1,\\
					0, & \text{falls }x > 1.
				\end{cases}
			\]
			\item Benutzen Sie die Hölder-Ungleichung in der Form
			\[
				\norm*{fg}_1 \le \norm*{f}_p \norm*{g}_q = \norm*{f}_3 \norm*{g}_{\frac{3}{2}},
			\]
			um das folgende Integral abzuschätzen:
			\[
				\int_\Omega \frac{\exp\parentheses*{-\frac{2}{3}x}}{\sqrt[3]{\parentheses*{x + 2}^4}}\d x.
			\]
		\end{enumerate}
	\end{quote}


	\section{}

	\begin{quote}
		\begin{enumerate}
			\item Berechnen Sie die Bogenlängen der Kurven, die durch ihre Trajektorien gegeben sind:
			\begin{enumerate}
				\item \(\Gamma_1 = \braces*{\parentheses*{x, y} \in \R^2 : x \in \brackets*{-1, 1}, y = a\cosh\parentheses*{\frac{x}{a}}}\) mit \(a \ne 0\),
				\item \(\Gamma_2 = \gamma_2\parentheses*{\brackets*{0, 2\pi}}\), wobei \(\gamma_2\parentheses*{t} = \begin{pmatrix}
					a\parentheses*{1 - \cos\parentheses*{t}}\cos\parentheses*{t}\\
					a\parentheses*{1 - \cos\parentheses*{t}}\sin\parentheses*{t}
				\end{pmatrix}\) mit \(a \ne 0\),
				\item \(\Gamma_3 = \gamma_3\parentheses*{\brackets*{1, e^{2\pi}}}\), wobei \(\gamma_3\parentheses*{t} = \parentheses*{\cos\parentheses*{\ln\parentheses*{t}}, \sin\parentheses*{\ln\parentheses*{t}}, h\ln\parentheses*{t}}^\top\) mit \(h > 0\).
			\end{enumerate}
			\item Ist \(\Gamma_1 \cup \Gamma_2\) ein Weg?
			\item Was ist das begleitende Dreibein von \(\Gamma_3\) bzw. \(\gamma_3\)?
			\item Finden Sie die natürliche Parametrisierung von \(\gamma_1\) mit \(a = 1\).
		\end{enumerate}
	\end{quote}

	\begin{enumerate}
		\item
		\begin{enumerate}
			\item Wir können die gegebene Kurve schreiben als \(\Gamma_1 = \gamma_1\parentheses*{\brackets*{-1, 1}}\), wobei \(\gamma_1\parentheses*{x} = \begin{pmatrix}
				x\\
				a\cosh\parentheses*{\frac{x}{a}}
			\end{pmatrix}\) und folglich \(\gamma_1'\parentheses*{x} = \begin{pmatrix}
				1\\
				\sinh\parentheses*{\frac{x}{a}}.
			\end{pmatrix}\).
			Damit können wir nun die Bogenlänge berechnen:
			\begin{align*}
				L\parentheses*{\gamma_1} &= \int_{-1}^1 \norm*{\gamma_1'\parentheses*{x}}\d x\\
				&= \int_{-1}^1 \sqrt{1 + \sinh^2\parentheses*{\frac{x}{a}}}\d x\\
				&= \int_{-1}^1 \cosh\parentheses*{\frac{x}{a}}\d x\\
				&= \brackets*{a\sinh\parentheses*{\frac{x}{a}}}_{-1}^1 = 2a\sinh\parentheses*{\frac{1}{a}}.
			\end{align*}
			\item Die Ableitung ist \(\gamma_2'\parentheses*{t} = a\begin{pmatrix}
				\sin\parentheses*{t}\parentheses*{2\cos\parentheses*{t} - 1}\\
				\sin^2\parentheses*{t} + \parentheses*{1 - \cos\parentheses*{t}}\cos\parentheses*{t}.
			\end{pmatrix}\) und es folgt
			\begin{align*}
				L\parentheses*{\gamma_2} &= \int_0^{2\pi}\norm*{\gamma_2'\parentheses*{t}}\d t\\
				&= a\int_0^{2\pi}\sqrt{\parentheses*{\sin\parentheses*{t}\parentheses*{2\cos\parentheses*{t} - 1}}^2 + \parentheses*{\sin^2\parentheses*{t} + \parentheses*{1 - \cos\parentheses*{t}}\cos\parentheses*{t}}^2}\d t\\
				&= a\int_0^{2\pi}\sqrt{\sin^2\parentheses*{t} + 1 - 2\cos\parentheses*{t} + \cos^2\parentheses*{t}}\d t\\
				&= 2a\int_0^{2\pi}\absolute*{\sin\parentheses*{\frac{t}{2}}}\d t = 8a.
			\end{align*}
			\item In diesem Fall erhalten wir \(\gamma_3'\parentheses*{t} = \frac{1}{t}\begin{pmatrix}
				-\sin\parentheses*{\ln\parentheses*{t}}\\
				\cos\parentheses*{\ln\parentheses*{t}}\\
				h
			\end{pmatrix}\) und für die Bogenlänge von \(\Gamma_3\) dann
			\begin{align*}
				L\parentheses*{\gamma_3} &= \int_1^{e^{2\pi}}\norm*{\gamma_3'\parentheses*{t}}\d t\\
				&= \int_1^{e^{2\pi}}\frac{1}{t}\sqrt{-\sin^2\parentheses*{\ln\parentheses*{t}} + \cos^2\parentheses*{\ln\parentheses*{t}} + h^2}\d t\\
				&= \sqrt{1 + h^2}\int_1^{e^{2\pi}}\frac{1}{t}\d t\\
				&= \sqrt{1 + h^2}\brackets*{\ln\parentheses*{t}}_1^{e^{2\pi}} = 2\pi\sqrt{1 + h^2}.
			\end{align*}
		\end{enumerate}
	\end{enumerate}


	\section{}

	\begin{quote}
		Consider the matrix \(A \in \R^{3 \times 3}\) given as follows
		\[
			A = \begin{pmatrix}
				3 & 2 & 0\\
				2 & 0 & 0\\
				0 & 0 & 2
			\end{pmatrix}
		\]
		with the following eigenvalues and their corresponding eigenvectors:
		\[
			\lambda_1 = -1, v_1 = \parentheses*{-1, 2, 0}^\top, \quad \lambda_2 = 2, v_2 = \parentheses*{0, 0, 1}^\top, \quad \lambda_3 = 4, v_3 = \parentheses*{2, 1, 0}^\top.
		\]
		\begin{enumerate}
			\item To which eigenvalue will the vector iteration converge, when we choose the starting vector \(v^{\parentheses*{0}} = \parentheses*{1, 1, 1}^\top\)?
			\item To which eigenvalue will the Wielandt's method converge, when we choose the starting vector \(v^{\parentheses*{0}} = \parentheses*{1, 1, 1}^\top\) and the initial guess \(\lambda = 0\)?
			\item To which eigenvalue will the inverse vector iteration with spectral shift converge, when we choose the starting vector \(v^{\parentheses*{0}} = \parentheses*{1, 1, 1}^\top\) and the initial guess \(\lambda = 1\)?
			Likewise, what do we expect for a different starting vector \(v^{\parentheses*{0}} = \parentheses*{1, 3, 0}^\top\)?
			\item How do we approximate the eigenvalue \(\lambda_2\) with the help of the QR-method?
		\end{enumerate}
	\end{quote}

	\begin{enumerate}
		\item The vector iteration converges to the eigenvalue with the largest absolute value, e.g. \(\lambda_3 = 4\).
		\item The Wielandt's method converges to the eigenvalue with the smallest absolute value, e.g. \(\lambda_1 = -1\).
		\item The inverse vector iteration with spectral shift converges to the eigenvalue with the smallest distance to the initial guess. For the starting vector \(v = \parentheses*{1, 1, 1}^\top\) and the initial guess \(\lambda = -1\) we thus expect it to converge to the eigenvalue \(\lambda_2 = 2\). For \(v = \parentheses*{1, 3, 0}^\top\) we won't obtain the covergence to \(\lambda_2 = 2\).
		\item The eigenvalues are on the diagonal of the iteration matrix of the QR-method.
	\end{enumerate}


	\section{}

	\begin{quote}
		Consider the matrix \(A\) given as follows
		\[
			A = \begin{pmatrix}
				5 & 1 & 2\\
				1 & -1 & 0\\
				0 & 0 & 6
			\end{pmatrix}.
		\]
		\begin{enumerate}
			\item Estimate the value of the eigenvalues of matrix \(A\).
			\item Perform one step of the QR-method to approximate the eigenvalues of the matrix \(A\).
		\end{enumerate}
	\end{quote}

	\begin{enumerate}
		\item Since \(A\) is not symmetric, the eigenvalues may be complex.
		For the given matrix we get the Gerschgorin disks
		\[
			D_1 = \braces*{\lambda \in \C : \absolute*{\lambda - 5} < 3}, \quad D_2 = \braces*{\lambda \in \C : \absolute*{\lambda + 1} \le 1}, \quad D_3 = \braces*{\lambda \in \C : \absolute*{\lambda - 6} < 0}.
		\]
		For the spectrum of \(A\) we thus obtain by the Gerschgorin theorem that
		\[
			\sigma\parentheses*{A} \subseteq \bigcup_{i = 1}^3 D_i.
		\]

		\item We first want to calculate a QR-decomposition of the matrix \(A_0 = A\) by using the Gram-Schmidt process.
		We denote by \(a_i\) and \(q_i\) the \(i\)-th column vector of \(A\) and the resulting matrix \(Q\). Then
		\begin{align*}
			a_1^\bot &= a_1 = \begin{pmatrix}
				5\\
				1\\
				0
			\end{pmatrix},\\
			q_1 &= \frac{a_1^\bot}{\norm*{a_1^\bot}} = \frac{1}{\sqrt{26}}\begin{pmatrix}
				5\\
				1\\
				0
			\end{pmatrix},\\
			a_2^\bot &= a_2 - \angles*{a_2, q_1}q_1 = \begin{pmatrix}
				1\\
				-1\\
				0
			\end{pmatrix} - \frac{4}{\sqrt{26}} \cdot \frac{1}{\sqrt{26}}\begin{pmatrix}
				5\\
				1\\
				0
			\end{pmatrix} = \frac{3}{13}\begin{pmatrix}
				1\\
				-5\\
				0
			\end{pmatrix},\\
			q_2 &= \frac{a_2^\bot}{\norm*{a_2^\bot}} = \frac{1}{\sqrt{26}}\begin{pmatrix}
				1\\
				-5\\
				0
			\end{pmatrix},\\
			a_3^\bot &= a_3 - \angles*{a_3, q_1}q_1 - \angles*{a_3, q_2}q_2 = \begin{pmatrix}
				2\\
				0\\
				6
			\end{pmatrix} - \frac{10}{\sqrt{26}} \cdot \frac{1}{\sqrt{26}}\begin{pmatrix}
				5\\
				1\\
				0
			\end{pmatrix} - \frac{2}{\sqrt{26}} \cdot \frac{1}{\sqrt{26}}\begin{pmatrix}
				1\\
				-5\\
				0
			\end{pmatrix} = \begin{pmatrix}
				0\\
				0\\
				6
			\end{pmatrix},\\
			q_3 &= \frac{a_3^\bot}{\norm*{a_3^\bot}} = \begin{pmatrix}
				0\\
				0\\
				1
			\end{pmatrix}.
		\end{align*}
		The resulting QR-factorization is
		\begin{align*}
			A &= \begin{pmatrix}
				\vert & \vert & \vert\\
				q_1 & q_2 & q_3\\
				\vert & \vert & \vert
			\end{pmatrix}\begin{pmatrix}
				\angles*{a_1, q_1} & \angles*{a_2, q_1} & \angles*{a_3, q_1}\\
				0 & \angles*{a_2, q_2} & \angles*{a_3, q_2}\\
				0 & 0 & \angles*{a_3, q_3}
			\end{pmatrix}\\
			&= \underbrace{\begin{pmatrix}
				\frac{5}{\sqrt{26}} & \frac{1}{\sqrt{26}} & 0\\
				\frac{1}{\sqrt{26}} & -\frac{5}{\sqrt{26}} & 0\\
				0 & 0 & 1
			\end{pmatrix}}_{=: Q}\underbrace{\begin{pmatrix}
				\sqrt{26} & \frac{4}{\sqrt{26}} & \frac{10}{\sqrt{26}}\\
				0 & \frac{6}{\sqrt{26}} & \frac{2}{\sqrt{26}}\\
				0 & 0 & 6
			\end{pmatrix}}_{=: R}.
		\end{align*}
		The new iteration matrix can then be calculated as follows:
		\[
			A_1 = RQ = \begin{pmatrix}
				\frac{67}{13} & \frac{3}{13} & 5\sqrt{\frac{2}{13}}\\
				\frac{3}{13} & -\frac{15}{13} & \sqrt{\frac{2}{13}}\\
				0 & 0 & 6
			\end{pmatrix}.
		\]
		The approximated eigenvalues are on the diagonal of \(A_1\):
		\[
			\lambda_1 \approx \frac{67}{13}, \quad \lambda_2 \approx -\frac{15}{13}, \quad \lambda_3 \approx 6.
		\]
	\end{enumerate}
\end{document}
