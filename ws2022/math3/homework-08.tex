\documentclass{exercise}

\DeclareMathOperator{\HI}{HI}
\DeclareMathOperator{\IoU}{IoU}


\title{Hausaufgabe 8}
\author{René Dopichay (356986) \quad Joshua Feld (406718)\\Thilo Kloos (410343) \quad Shunta Takushima (430043)}
\professor{Prof. Torrilhon \& Dr. Speck}
\course{Mathematische Grundlagen III}

\begin{document}
	\maketitle


	\section{}

	\begin{quote}
		\begin{enumerate}
			\item Gegeben sei die Funktion \(f\) wie folgt.
			Zeigen Sie, dass \(f \not\in L^1\parentheses*{\Omega}\) aber \(f \in L^2\parentheses*{\Omega}\).
			\[
				f: \Omega := \parentheses*{0, \infty} \to \R, x \mapsto f\parentheses*{x} := \begin{cases}
					1, & \text{falls }0 \le x \le 1,\\
					x^{-\frac{2}{3}}, & \text{falls }x > 1.
				\end{cases}
			\]
			\item Gegeben sei die Funktion \(g\) wie folgt.
			Zeigen Sie dass \(g \in L^1\parentheses*{\Omega}\) aber \(g \not\in L^2\parentheses*{\Omega}\).
			\[
				g: \Omega := \parentheses*{0, \infty} \to \R, x \mapsto g\parentheses*{x} := \begin{cases}
					x^{-\frac{3}{4}}, & \text{falls }0 < x \le 1,\\
					0, & \text{falls }x > 1.
				\end{cases}
			\]
			\item Benutzen Sie die Hölder-Ungleichung in der Form
			\[
				\norm*{fg}_1 \le \norm*{f}_p \norm*{g}_q = \norm*{f}_3 \norm*{g}_{\frac{3}{2}},
			\]
			um das folgende Integral abzuschätzen:
			\[
				\int_\Omega \frac{\exp\parentheses*{-\frac{2}{3}x}}{\sqrt[3]{\parentheses*{x + 2}^4}}\d x.
			\]
		\end{enumerate}
	\end{quote}


	\section{}

	\begin{quote}
		\begin{enumerate}
			\item Berechnen Sie die Bogenlängen der Kurven, die durch ihre Trajektorien gegeben sind:
			\begin{enumerate}
				\item \(\Gamma_1 = \braces*{\parentheses*{x, y} \in \R^2 : x \in \brackets*{-1, 1}, y = a\cosh\parentheses*{\frac{x}{a}}}\) mit \(a \ne 0\),
				\item \(\Gamma_2 = \gamma_2\parentheses*{\brackets*{0, 2\pi}}\), wobei \(\gamma_2\parentheses*{t} = \begin{pmatrix}
					a\parentheses*{1 - \cos\parentheses*{t}}\cos\parentheses*{t}\\
					a\parentheses*{1 - \cos\parentheses*{t}}\sin\parentheses*{t}
				\end{pmatrix}\) mit \(a \ne 0\),
				\item \(\Gamma_3 = \gamma_3\parentheses*{\brackets*{1, e^{2\pi}}}\), wobei \(\gamma_3\parentheses*{t} = \parentheses*{\cos\parentheses*{\ln\parentheses*{t}}, \sin\parentheses*{\ln\parentheses*{t}}, h\ln\parentheses*{t}}^\top\) mit \(h > 0\).
			\end{enumerate}
			\item Ist \(\Gamma_1 \cup \Gamma_2\) ein Weg?
			\item Was ist das begleitende Dreibein von \(\Gamma_3\) bzw. \(\gamma_3\)?
			\item Finden Sie die natürliche Parametrisierung von \(\gamma_1\) mit \(a = 1\).
		\end{enumerate}
	\end{quote}


	\section{}

	\begin{quote}
		Consider the matrix \(A \in \R^{3 \times 3}\) given as follows
		\[
			A = \begin{pmatrix}
				3 & 2 & 0\\
				2 & 0 & 0\\
				0 & 0 & 2
			\end{pmatrix}
		\]
		with the following eigenvalues and their corresponding eigenvectors:
		\[
			\lambda_1 = -1, v_1 = \parentheses*{-1, 2, 0}^\top, \quad \lambda_2 = 2, v_2 = \parentheses*{0, 0, 1}^\top, \quad \lambda_3 = 4, v_3 = \parentheses*{2, 1, 0}^\top.
		\]
		\begin{enumerate}
			\item To which eigenvalue will the vector iteration converge, when we choose the starting vector \(v^{\parentheses*{0}} = \parentheses*{1, 1, 1}^\top\)?
			\item To which eigenvalue will the Wielandt's method converge, when we choose the starting vector \(v^{\parentheses*{0}} = \parentheses*{1, 1, 1}^\top\) and the initial guess \(\lambda = 0\)?
			\item To which eigenvalue will the inverse vector iteration with spectral shift converge, when we choose the starting vector \(v^{\parentheses*{0}} = \parentheses*{1, 1, 1}^\top\) and the initial guess \(\lambda = 1\)?
			Likewise, what do we expect for a different starting vector \(v^{\parentheses*{0}} = \parentheses*{1, 3, 0}^\top\)?
			\item How do we approximate the eigenvalue \(\lambda_2\) with the help of the QR-method?
		\end{enumerate}
	\end{quote}


	\section{}

	\begin{quote}
		Consider the matrix \(A\) given as follows
		\[
			A = \begin{pmatrix}
				5 & 1 & 2\\
				1 & -1 & 0\\
				0 & 0 & 6
			\end{pmatrix}.
		\]
		\item Estimate the value of the eigenvalues of matrix \(A\).
		\item Perform one step of the QR-method to approximate the eigenvalues of the matrix \(A\).
	\end{quote}
\end{document}
