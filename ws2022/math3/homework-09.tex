\documentclass{exercise}

\DeclareMathOperator*{\cond}{cond}
\DeclareMathOperator*{\diag}{diag}
\DeclareMathOperator*{\dist}{dist}
\DeclareMathOperator*{\esssup}{ess\,sup}
\DeclareMathOperator*{\vol}{vol}


\title{Hausaufgabe 8}
\author{René Dopichay (356986) \quad Joshua Feld (406718)\\Thilo Kloos (410343) \quad Shunta Takushima (430043)}
\professor{Prof. Torrilhon \& Dr. Speck}
\course{Mathematische Grundlagen III}

\begin{document}
	\maketitle


	\section{}

	\begin{quote}
		
	\end{quote}


	\section{}

	\begin{quote}
		Calculate the total mass \(M\) of the helix \(\Gamma = \gamma\parentheses*{\brackets*{0, 2\pi}}\) with density \(\rho\) given as follows
		\[
			\gamma: \brackets*{0, 2\pi} \to \R^3, t \mapsto \gamma\parentheses*{t} = \parentheses*{\cos t, \sin t, ht} \quad \text{and} \quad \rho: \Gamma \to \R, \parentheses*{x, y, z} \mapsto \rho\parentheses*{x, y, z} = z.
		\]
	\end{quote}


	\section{}

	\begin{quote}
		Gegeben sei das Vektorfeld \(f\) wie folgt
		\[
			f: \R^2 \to \R^2, \parentheses*{x, y} \mapsto f\parentheses*{x, y} = \begin{pmatrix}
				xe^y\\
				\sin\parentheses*{x} + y
			\end{pmatrix},
		\]
		sowie die beiden Wege \(\Gamma_1\) und \(\Gamma_2\) von \(\parentheses*{0, 0}\) nach \(\parentheses*{1, 1}\):
		\begin{center}
			\begin{tikzpicture}
				\draw[<->] (0,2.5) node[right] {\(y\)} -- (0,0) -- (2.5,0) node[above] {\(x\)};
				\draw[->] (0,0) -- (2,0);
				\draw[->] (2,0) -- (2,2);
				\draw[dashed] (0,2) -- (2,2);
				\draw (2,.125) -- (2,-.125) node[below] {\(1\)};
				\draw (.125,2) -- (-.125,2) node[left] {\(1\)};
			\end{tikzpicture}
		\end{center}
	\end{quote}

	
	\section{}

	\begin{quote}
		Benutzen Sie das QR-Verfahren mit Shift zur Berechnung der Eigenwerte der Matrix
		\[
			A = \begin{pmatrix}
				3 & \varepsilon\\
				\varepsilon & 1
			\end{pmatrix}_{2 \times 2}.
		\]
		\begin{enumerate}
			\item Berechnen Sie die QR-Zerlegung von \(A - \sigma_1 I := QR\) für beliebiges \(\sigma_1\).
			\item Führen Sie nun einen Schritt von QR-Verfahren mit Shift \(\sigma_1\) durch, d.h. stellen Sie die QR-Zerlegung der Matrix \(A - \sigma_1 I := QR\) auf, und berechnen Sie die Transformierte mit Rück-Shift \(A_1 := RQ + \sigma_1 I\) für
			\begin{enumerate}
				\item \(\sigma_1 = 0\), d.h. ohne Shift,
				\item \(\sigma_1 = 1\), d.h. mit Shift.
				\item Wie ändert sich die Konvergenz der Nicht- bzw. Nebendiagonalelemente im Vergleich \(A_1\) für \(\sigma_1 = 0\) zu \(A_1\) für \(\sigma_1 = 1\) unter der Voraussetzung \(\varepsilon \ll 1\)?
			\end{enumerate}
		\end{enumerate}
	\end{quote}
\end{document}