\documentclass{exercise}

\DeclareMathOperator{\HI}{HI}
\DeclareMathOperator{\IoU}{IoU}


\title{Selbstrechenübung 3}
\author{Joshua Feld (406718)}
\professor{Prof. Torrilhon \& Dr. Speck}
\course{Mathematische Grundlagen III}

\begin{document}
	\maketitle


	\section{}

	\begin{quote}
		Gesucht ist eine Funktion \(y \in D\) mit
		\[
			D := \braces*{w \in C^2\parentheses*{\brackets*{0, 1}} : w\parentheses*{0} = w\parentheses*{1} = 1},
		\]
		die das Funktional
		\[
			F: D \to \R, y \mapsto F\parentheses*{y} = \int_0^1 \parentheses*{{y'}^2\parentheses*{t} + e^{ay\parentheses*{t}}}\d t,
		\]
		welches von einer Konstanten \(a \in \R\) abhängt, minimiert.
		\begin{enumerate}
			\item Bestimmen Sie die erste Variation \(\partial F\parentheses*{y; v}\) von \(F\) in Richtung \(v\), wobei
			\[
				v \in D_0 := \braces*{w \in C^2\parentheses*{\brackets*{0, 1}} : w\parentheses*{0} = w\parentheses*{1} = 0}.
			\]
			\item Welcher Differentialgleichung müssen die Extremalen des Funktionals \(F\) genügen?
			\item Ist \(D\) konvex?
			Ist das Funktional \(F\) konvex?
			Was können Sie über globale Extremalen des Funktionals \(F\) aussagen?
			Gibt es eine Lösung für \(a = 0\)?
		\end{enumerate}
	\end{quote}


	\section{}

	\begin{quote}
		Define \(D = \braces*{w \in C^2\parentheses*{\brackets*{-1, 1}} : w\parentheses*{-1} = -1}\) and
		\[
			F: D \to \R, y \mapsto F\parentheses*{y} = \int_{-1}^1 \exp\parentheses*{{y'}^2\parentheses*{x}}\d x, \quad y \in D.
		\]
		\begin{enumerate}
			\item Calculate the first variation \(\partial F\parentheses*{u; v}\) of the functional \(F\) for \(u \in D\) and
			\[
				v \in H = \braces*{w \in C^2\parentheses*{\brackets*{-1, 1}} : w\parentheses*{-1} = 0, w\parentheses*{1} = 0}.
			\]
			\item Formulate the Euler-Lagrange equation for extremals of \(F\) and work out all solutions \(u \in D\) which fulfill
			\begin{enumerate}
				\item the boundary condition \(w\parentheses*{1} = 1\),
				\item the natural boundary condition at \(x = 1\).
			\end{enumerate}
		\end{enumerate}
	\end{quote}


	\section{}

	\begin{quote}
		Following the notations used in the lecture, let a linear \(m\)-step method be given by \(\parentheses*{\alpha_l, \beta_l}\) where \(l = 0, \ldots, m\) satisfy the following equations
		\begin{align}
			\sum_{l = 0}^m a_l &= 0,\\
			\sum_{l = 0}^m \parentheses*{l^q \alpha_l - ql^{q - 1}\beta_l} &= 0, \quad q = 1, \ldots, p.
		\end{align}
		Then the linear \(m\)-step method has the consistency order of \(p\).
		Prove this statement.
	\end{quote}


	\section{}

	\begin{quote}
		Bestimme für das lineare 2-Schrittverfahren
		\[
			y^{j + 2} = y^j + \frac{h}{3}\parentheses*{f\parentheses*{t_j, y^j} + 4f\parentheses*{t_{j + 1}, y^{j + 1}} + f\parentheses*{t_{j + 2}, y^{j + 2}}}
		\]
		die Konsistenzordnung.
	\end{quote}
\end{document}
