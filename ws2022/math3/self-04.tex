\documentclass{exercise}

\DeclareMathOperator{\HI}{HI}
\DeclareMathOperator{\IoU}{IoU}


\title{Selbstrechenübung 4}
\author{Joshua Feld (406718)}
\professor{Prof. Torrilhon \& Dr. Speck}
\course{Mathematische Grundlagen III}

\begin{document}
	\maketitle


	\section{}

	\begin{quote}

	\end{quote}


	\section{}

	\begin{quote}
		Die Cantormenge ist gegeben durch \(C = \brackets*{0, 1} \setminus \parentheses*{\bigcup_n A_n}\), mit
		\[
			A_1 = \parentheses*{\frac{1}{3}, \frac{2}{3}}, \quad A_2 = \parentheses*{\frac{1}{9}, \frac{2}{9}} \cup \parentheses*{\frac{7}{9}, \frac{8}{9}}, \quad \ldots,
		\]
		d.h. man schneidet in jedem Schritt jeweils das mittlere Drittel der verbleibenden Intervalle heraus.
		Zeigen Sie, dass die Cantormenge eine Lebesgue-Nullmenge ist.
	\end{quote}


	\section{}

	\begin{quote}
		Wir betrachten die 2-Schrittmethode
		\[
			y^{j + 2} = -4y^{j + 1} + 5y^j + h\parentheses*{2f\parentheses*{t_j, y^j} + 4f\parentheses*{t_{j + 1}, y^{j + 1}}}.
		\]
		Zeigen Sie, dass das Verfahren nicht nullstabil ist.
	\end{quote}

	The given method can be written as
	\[
		y^{j + 2} + 4y^{j + 1} - 5y^j = h\parentheses*{2f\parentheses*{t_j, y^j} + 4f\parentheses*{t_{j + 1}, y^{j + 1}}}.
	\]
	Clearly, the characteristic polynomial for the method is
	\[
		\rho\parentheses*{z} = z^2 + 4z - 5 = \parentheses*{z + 5}\parentheses*{z - 1}.
	\]
	The zeros of the characteristic polynomial \(\rho\parentheses*{z}\) are \(z = -5\) with multiplicity \(1\) and \(z = 1\) with multiplicity \(1\).
	Obviously, the root \(z = -5\) does not reside in the closed complex unit disc.
	Therefore, the root condition is not satisfied for the given method.


	\section{}

	\begin{quote}
		
	\end{quote}
\end{document}
