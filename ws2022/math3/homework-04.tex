\documentclass{exercise}

\DeclareMathOperator*{\cond}{cond}
\DeclareMathOperator*{\diag}{diag}
\DeclareMathOperator*{\dist}{dist}
\DeclareMathOperator*{\esssup}{ess\,sup}
\DeclareMathOperator*{\vol}{vol}


\title{Hausaufgabe 4}
\author{René Dopichay (356986) \quad Joshua Feld (406718)\\Thilo Kloos (410343) \quad Shunta Takushima (430043)}
\professor{Prof. Torrilhon \& Dr. Speck}
\course{Mathematische Grundlagen III}

\begin{document}
	\maketitle


	\section{}

    \begin{quote}
        Zeigen Sie, dass das äußere Lebesgue-Maß \(\lambda^*\) translationsinvariant ist, d.h. es gilt
        \[
            \lambda^*\parentheses*{x + M} = \lambda^*\parentheses*{M}
        \]
        für alle \(x \in \R^n\) und alle \(M \subset \R^n\).
    \end{quote}

    Um die Translationsinvarianz des Lebesgue-Maßes \(\lambda^*\) zu zeigen, müssen wir zunächst zeigen, dass das Volumen \(\vol\) translationsinvariant ist. Dies ist leicht zu zeigen.
    Sei dazu \(Q = \brackets*{a, b}^n \in q_n\) und \(x \in \R^n\).
    Dann gilt \(x + Q = \brackets*{x + a, x + b} \in q_n\) und somit
    \[
        \vol\parentheses*{x + Q} = \prod_{i = 1}^n \parentheses*{x_i + b_i - \parentheses*{x_i + a_i}} = \prod_{i = 1}^n \parentheses*{b_i - a_i} = \vol\parentheses*{Q}.
    \]
    Somit gilt nun für das Lebesgue-Maß
    \begin{align*}
        \lambda^*\parentheses*{x + M} &= \inf\braces*{\sum_{k = 1}^\infty \vol\parentheses*{Q_k} : x + M \subset \bigcup_{k \in \N}Q_k}\\
        &= \inf\braces*{\sum_{k = 1}^\infty \vol\parentheses*{Q_k} : M \subset \bigcup_{k \in \N}\parentheses*{-x + Q_k}}\\
        &= \inf\braces*{\sum_{k = 1}^\infty \vol\parentheses*{-x + Q_k} : M \subset \bigcup_{k \in \N}\parentheses*{-x + Q_k}}\\
        &= \inf\braces*{\sum_{k = 1}^\infty \vol\parentheses*{Q_k} : M \subset \bigcup_{k \in \N}Q_k}\\
        &= \lambda^*\parentheses*{M}.
    \end{align*}
    
    
    \section{}
    
    \begin{quote}
        Gegeben sei \(\Omega = \parentheses*{0, 1}\) sowie die Funktionen
        \begin{align*}
            f: \Omega \to \R, &f\parentheses*{x} = \frac{1}{\sqrt{x}},\\
            g: \Omega \to \R, &g\parentheses*{x} = \exp\parentheses*{-x}.
        \end{align*}
        \begin{enumerate}
            \item Ist \(g\) Lebesgue-messbar?
            Ist \(g\) Lebesgue-integrierbar?
            \item Geben Sie eine Funktion \(h\) an mit \(f \ne h\) und \(h \sim f\) bzgl. der Äquivalenzrelation
            \[
                u \sim v \iff u = v\text{ fast überall auf }\Omega.
            \]
        \end{enumerate}
    \end{quote}

    \begin{enumerate}
        \item Die Umkehrfunktion von \(g\) ist gegeben durch
        \[
            g^{-1}: \R \to \Omega, g^{-1}\parentheses*{x} = -\log\parentheses*{x}.
        \]
        \(g^{-1}\parentheses*{\parentheses*{\ell, u}}\) ist Lebesgue-messbar für alle \(\ell, u \in \Omega\) mit \(\ell \le u\), weshalb auch \(g\) Lebesgue-messbar ist.
        Die Funktion \(g\) ist Riemann-integrierbar auf \(\Omega\), denn
        \[
            \int_\Omega g\d\mu = \int_0^1 \exp\parentheses*{-x}\d x = \left.-\exp\parentheses*{-x}\right|_0^1 < \infty,
        \]
        und somit auch Lebesgue-integrierbar.
        \item Sei
        \[
            h: \Omega \to \R, h\parentheses*{x} = \begin{cases}
                f, & \text{falls }x \ne a,\\
                c, & \text{sonst},
            \end{cases} \quad a \in \parentheses*{0, 1}, c \in \R.
        \]
        Dann ist \(f \ne h\) aber \(f = h\) fast überall auf \(\Omega\) und somit \(h \sim f\).
    \end{enumerate}
    
    
    \section{}
    
    \begin{quote}
        Zeigen Sie, dass für die Adams-Bashforth- und Adams-Moulton-Verfahren die Wurzelbedingung erfüllt ist.
    \end{quote}

    Die Wurzelbedingung ist wie folgt definiert:
    Die Nullstellen \(\lambda\) des charakteristischen Polynoms \(\rho\) erfüllen \(\absolute*{\lambda} \le 1\), und die Nullstellen \(\lambda\) mit \(\absolute*{\lambda} = 1\) sind einfache Nullstellen.
    Die beiden Verfahren sind gegeben durch
    \begin{align*}
        y^{j + k} - y^{j + k - 1} &= h\sum_{\ell = 0}^{k - 1}b_{k, \ell}f\parentheses*{t_{j + \ell}, y^{j + \ell}} \quad \text{(Adams-Bashforth)},\\
        y^{j + k} - y^{j + k - 1} &= h\sum_{\ell = 0}^k b_{k, \ell}f\parentheses*{t_{j + \ell}, y^{j + \ell}} \quad \text{(Adams-Moulton)}.
    \end{align*}
    Offensichtlich haben beide Verfahren das charakteristische Polynom
    \[
        \rho\parentheses*{\lambda} = \lambda^{k - 1}\parentheses*{\lambda - 1}.
    \]
    Setzen wir nun \(\rho\parentheses*{\lambda} = 0\), so erhalten wir die Nullstelle \(\lambda = 0\) mit algebraischer Vielfachheit \(k - 1\) und \(\lambda = 1\) mit algebraischer Vielfachheit \(1\). Damit ist die Wurzelbedingung für beide Verfahren erfüllt.

    
    \section{}
    
    \begin{quote}
        Betrachten Sie das Adams-Bashforth Schema
        \[
            y^{j + 2} - y^{j + 1} = h\parentheses*{\frac{3}{2}f\parentheses*{t_{j + 1}, y^{j + 1}} - \frac{1}{2}f\parentheses*{t_j, y^j}}
        \]
        und das Anfangswertproblem
        \[
            \frac{\d^2 y}{\d t^2}\parentheses*{t} = t + y\parentheses*{t},
        \]
        mit den Anfangswerten \(y\parentheses*{0} = 1\) und \(y'\parentheses*{0} = -2\).
        \begin{enumerate}
            \item Was ist die Ordnung des Schemas?
            Ist das Verfahren implizit oder explizit?
            \item Weisen Sie nach, dass \(y\parentheses*{t} =e^{-t} - t\) die exakte Lösung des Problems ist.
            \item Zum Lösen des Anfangswertproblems mit dem oben angegebenen Adams-Bashforth Schema werden zwei Anfangswerte benötigt: \(y\parentheses*{0}\) und \(y\parentheses*{h}\).
            Welche Verfahren sind geeignet, um den Anfangswert \(y\parentheses*{h}\) numerisch zu berechnen?
            \item Überführen Sie das Anfangswertproblem in ein System von gewöhnlichen Differentiagleichungen erster Ordnung.
            \item Sei \(h = 0,1\).
            Approximieren Sie \(y\parentheses*{h}\) mit Hilfe des modifizierten Euler-Verfahrens, das durch das folgende Tableau gegeben ist:
            \[
                \renewcommand\arraystretch{1.2}
				\begin{array}{c|cc}
					0 & 0 & 0\\
					\frac{1}{2} & \frac{1}{2} & 0\\
					\hline
					& 0 & 1
				\end{array}
            \]
            \item Berechnen Sie \(y\parentheses*{2h}\) mit Hilfe von \(y\parentheses*{0}\) und \(y\parentheses*{h}\) mit dem Adams-Bashforth Schema.
        \end{enumerate}
    \end{quote}

    \begin{enumerate}
        \item Das Verfahren ist explizit und hat die Ordnung \(2\).
        \item Die Anfangsbedingungen sind erfüllt, denn \(y\parentheses*{0} = e^{-0} - 0 = 1\) und \(y'\parentheses*{0} = -e^{-0} - 1 = -2\).
        Wir müssen also nur noch die Differentialgleichung prüfen:
        \[
            \frac{\d^2 y}{\d t^2}\parentheses*{t} = \frac{\d^2}{\d t^2}\parentheses*{e^{-t} - t} = e^{-t} = t + \parentheses*{e^{-t} - t} = t + y\parentheses*{t}.
        \]
        \item Zur Berechnung des Anfangswertes \(y\parentheses*{h}\) sind alle Verfahren mit Ordnung \(\ge 2\) geeignet.
        \item Als System gewöhnlicher Differentialgleichungen erster Ordnung können wir das gegebene Anfangswertproblem darstellen durch
        \[
            \frac{\d}{\d t}\begin{pmatrix}
                y\parentheses*{t}\\
                y'\parentheses*{t}
            \end{pmatrix} = \begin{pmatrix}
                y'\parentheses*{t}\\
                t + y\parentheses*{t}
            \end{pmatrix},
        \]
        mit den Anfangsbedingungen
        \[
            \begin{pmatrix}
                y\parentheses*{0}\\
                y'\parentheses*{0}
            \end{pmatrix} = \begin{pmatrix}
                1\\
                -2
            \end{pmatrix}.
        \]
        \item
        \begin{align*}
            \begin{pmatrix}
                y\parentheses*{h}\\
                y'\parentheses*{h}
            \end{pmatrix} &= \begin{pmatrix}
                1\\
                -2
            \end{pmatrix} + hf\parentheses*{0 + \frac{h}{2}, \begin{pmatrix}
                1\\
                -2
            \end{pmatrix} + \frac{h}{2}f\parentheses*{0, \begin{pmatrix}
                1\\
                -2
            \end{pmatrix}}}\\
            &= \begin{pmatrix}
                1\\
                -2
            \end{pmatrix} + hf\parentheses*{\frac{h}{2}, \begin{pmatrix}
                1\\
                -2
            \end{pmatrix} + \frac{h}{2}\begin{pmatrix}
                -2\\
                1
            \end{pmatrix}}\\
            &= \begin{pmatrix}
                1\\
                -2
            \end{pmatrix} + hf\parentheses*{\frac{h}{2}, \begin{pmatrix}
                1 - h\\
                -2 + \frac{h}{2}
            \end{pmatrix}}\\
            &= \begin{pmatrix}
                1\\
                -2
            \end{pmatrix} + h\begin{pmatrix}
                -2 + \frac{h}{2}\\
                1 - \frac{h}{2}
            \end{pmatrix}\\
            &= \begin{pmatrix}
                \frac{1}{2}h^2 - 2h + 1\\
                -\frac{1}{2}h^2 + h - 2
            \end{pmatrix}.
        \end{align*}
        Setzen wir hier nun noch \(h = 0,1\) ein und betrachten nur die erste Zeile, so erhalten wir
        \[
            y\parentheses*{0,1} = \frac{1}{2} \cdot 0,1^2 - 2 \cdot 0,1 + 1 = 0,805.
        \]
        \item
        \begin{align*}
            \begin{pmatrix}
                y\parentheses*{2h}\\
                y'\parentheses*{2h}
            \end{pmatrix} &= \begin{pmatrix}
                y\parentheses*{h}\\
                y'\parentheses*{h}
            \end{pmatrix} + h\parentheses*{\frac{3}{2}f\parentheses*{h, \begin{pmatrix}
                y\parentheses*{h}\\
                y'\parentheses*{h}
            \end{pmatrix}} - \frac{1}{2}f\parentheses*{0, \begin{pmatrix}
                y\parentheses*{0}\\
                y'\parentheses*{0}
            \end{pmatrix}}}\\
            &= \begin{pmatrix}
                \frac{1}{2}h^2 - 2h + 1\\
                -\frac{1}{2}h^2 + h - 2
            \end{pmatrix} + h\parentheses*{\frac{3}{2}f\parentheses*{h, \begin{pmatrix}
                \frac{1}{2}h^2 - 2h + 1\\
                -\frac{1}{2}h^2 + h - 2
            \end{pmatrix}} - \frac{1}{2}f\parentheses*{0, \begin{pmatrix}
                1\\
                -2
            \end{pmatrix}}}\\
            &= \begin{pmatrix}
                \frac{1}{2}h^2 - 2h + 1\\
                -\frac{1}{2}h^2 + h - 2
            \end{pmatrix} + h\parentheses*{\frac{3}{2}\begin{pmatrix}
                -h^2 + h - 2\\
                \frac{1}{2}h^2 - h + 1
            \end{pmatrix} - \frac{1}{2}\begin{pmatrix}
                -2\\
                1
            \end{pmatrix}}\\
            &= \begin{pmatrix}
                -\frac{3}{4}h^3 + 2h^2 - 4h + 1\\
                \frac{3}{4}h^3 - 2h^2 + 2h - 2
            \end{pmatrix}.
        \end{align*}
        Setzen wir hier wieder \(h = 0,1\) ein und betrachten erneut nur die erste Zeile, so erhalten wir
        \[
            y\parentheses*{0,2} = -\frac{3}{4} \cdot 0,1^3 + 2 \cdot 0,1^2 - 4 \cdot 0,1 + 1 = 0,61925.
        \]
    \end{enumerate}
\end{document}
