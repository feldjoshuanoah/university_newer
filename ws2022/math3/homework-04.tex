\documentclass{exercise}

\DeclareMathOperator{\HI}{HI}
\DeclareMathOperator{\IoU}{IoU}


\title{Hausaufgabe 4}
\author{René Dopichay (356986) \quad Joshua Feld (406718)\\Thilo Kloos (410343) \quad Shunta Takushima (430043)}
\professor{Prof. Torrilhon \& Dr. Speck}
\course{Mathematische Grundlagen III}

\begin{document}
	\maketitle


	\section{}

    \begin{quote}
        Zeigen Sie, dass das äußere Lebesgue-Maß \(\lambda^*\) translationsinvariant ist, d.h. es gilt
        \[
            \lambda^*\parentheses*{x + M} = \lambda^*\parentheses*{M}
        \]
        für alle \(x \in \R^n\) und alle \(M \subset \R^n\).
    \end{quote}
    
    
    \section{}
    
    \begin{quote}
        Gegeben sei \(\Omega = \parentheses*{0, 1}\) sowie die Funktionen
        \begin{align*}
            f: \Omega \to \R, &f\parentheses*{x} = \frac{1}{\sqrt{x}},\\
            g: \Omega \to \R, &g\parentheses*{x} = \exp\parentheses*{-x}.
        \end{align*}
        \begin{enumerate}
            \item Ist \(g\) Lebesgue-messbar?
            Ist \(g\) Lebesgue-integrierbar?
            \item Geben Sie eine Funktion \(h\) an mit \(f \ne h\) und \(h \sim f\) bzgl. der Äquivalenzrelation
            \[
                u \sim v \iff u = v\text{ fast überall auf }\Omega.
            \]
        \end{enumerate}
    \end{quote}
    
    
    \section{}
    
    \begin{quote}
        Zeigen Sie, dass für die Adams-Bashforth- und Adams-Moulton-Verfahren die Wurzelbedingung erfüllt ist.
    \end{quote}
    
    
    \section{}
    
    \begin{quote}
        Betrachten Sie das Adams-Bashforth Schema
        \[
            y^{j + 2} - y^{j + 1} = h\parentheses*{\frac{3}{2}f\parentheses*{t_{j + 1}, y^{j + 1}} - \frac{1}{2}f\parentheses*{t_j, y^j}}
        \]
        und das Anfangswertproblem
        \[
            \frac{\d^2 y}{\d t^2}\parentheses*{t} = t + y\parentheses*{t},
        \]
        mit den Anfangswerten \(y\parentheses*{0} = 1\) und \(y'\parentheses*{0} = -2\).
        \begin{enumerate}
            \item Was ist die Ordnung des Schemas?
            Ist das Verfahren implizit oder explizit?
            \item Weisen Sie nach, dass \(y\parentheses*{t} =e^{-t} - t\) die exakte Lösung des Problems ist.
            \item Zum Lösen des Anfangswertproblems mit dem oben angegebenen Adams-Bashforth Schema werden zwei Anfangswerte benötigt: \(y\parentheses*{0}\) und \(y\parentheses*{h}\).
            Welche Verfahren sind geeignet, um den Anfangswert \(y\parentheses*{h}\) numerisch zu berechnen?
            \item Überführen Sie das Anfangswertproblem in ein System von gewöhnlichen Differentiagleichungen erster Ordnung.
            \item Sei \(h = 0,1\).
            Approximieren Sie \(y\parentheses*{h}\) mit Hilfe des modifizierten Euler-Verfahrens, das durch das folgende Tableau gegeben ist:
            \[
                \renewcommand\arraystretch{1.2}
				\begin{array}{c|cc}
					0 & 0 & 0\\
					\frac{1}{2} & \frac{1}{2} & 0\\
					\hline
					& 0 & 1
				\end{array}
            \]
            \item Berechnen Sie \(y\parentheses*{2h}\) mit Hilfe von \(y\parentheses*{0}\) und \(y\parentheses*{h}\) mit dem Adams-Bashforth Schema.
        \end{enumerate}
    \end{quote}
\end{document}
