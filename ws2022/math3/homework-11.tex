\documentclass{exercise}

\DeclareMathOperator*{\cond}{cond}
\DeclareMathOperator*{\diag}{diag}
\DeclareMathOperator*{\dist}{dist}
\DeclareMathOperator*{\esssup}{ess\,sup}
\DeclareMathOperator*{\vol}{vol}


\title{Hausaufgabe 11}
\author{René Dopichay (356986) \quad Joshua Feld (406718)\\Thilo Kloos (410343) \quad Shunta Takushima (430043)}
\professor{Prof. Torrilhon \& Dr. Speck}
\course{Mathematische Grundlagen III}

\begin{document}
	\maketitle


	\section{}

	\begin{quote}
		Gegeben Sei die Fläche \(F \subset \R^3\)
		\[
			F = \braces*{\parentheses*{e^\alpha, \alpha\cos\parentheses*{\beta}, \alpha\sin\parentheses*{\beta}} : \alpha \in \brackets*{0, 1}, \beta \in \left[0, 2\pi\right[}
		\]
		und das Vektorfeld \(f\) mit \(f\parentheses*{x, y, z} = \parentheses*{x\sqrt{y^2 + z^2}, y, z}\).
		\begin{enumerate}
			\item Berechnen Sie die Oberflächennormale \(\nu\) zu \(F\).
			Die Oberflächennormale \(\nu\) sei mit einer positiven ersten Komponente definiert.
			\item Bestimmen Sie \(\int_F f \cdot \nu\d\sigma\).
		\end{enumerate}
	\end{quote}

	\begin{enumerate}
		\item Wir berechnen zunächst
		\[
			\tilde{\nu} = \frac{\partial F}{\partial\alpha} \times \frac{\partial F}{\partial\beta} = \begin{pmatrix}
				e^{\alpha}\\
				\cos\beta\\
				\sin\beta
			\end{pmatrix} \times \begin{pmatrix}
				0\\
				-\alpha\sin\beta\\
				\alpha\cos\beta
			\end{pmatrix} = \begin{pmatrix}
				\alpha\\
				-\alpha\cos\parentheses*{\beta}e^\alpha\\
				-\alpha\sin\parentheses*{\beta}e^\alpha
			\end{pmatrix},
		\]
		wobei wir im letzten Schritt genutzt haben, dass \(\cos^2\parentheses*{\beta} + \sin^2\parentheses*{\beta} = 1\) gilt.
		Dies gilt es nun noch zu normieren, was uns die tatsächliche Oberflächennormale
		\begin{align*}
			\nu &= \frac{\tilde{\nu}}{\norm*{\tilde{\nu}}}\\
			&= \frac{1}{\sqrt{\alpha^2 + \alpha^2 \cos^2\parentheses*{\beta}e^{2\alpha} + \alpha^2 \sin^2\parentheses*{\beta}e^{2\alpha}}}\begin{pmatrix}
				\alpha\\
				-\alpha\cos\parentheses*{\beta}e^\alpha\\
				-\alpha\sin\parentheses*{\beta}e^\alpha
			\end{pmatrix}\\
			&= \frac{1}{\alpha\sqrt{1 + e^{2\alpha}}}\begin{pmatrix}
				\alpha\\
				-\alpha\cos\parentheses*{\beta}e^\alpha\\
				-\alpha\sin\parentheses*{\beta}e^\alpha
			\end{pmatrix}\\
			&= \frac{1}{\sqrt{1 + e^{2\alpha}}}\begin{pmatrix}
				1\\
				-\cos\parentheses*{\beta}e^\alpha\\
				-\sin\parentheses*{\beta}e^\alpha
			\end{pmatrix}.
		\end{align*}
		\item
		\begin{align*}
			\int_F f \cdot \nu\d\sigma &= \int_0^1 \int_0^{2\pi}f\parentheses*{F\parentheses*{\alpha, \beta}} \cdot \tilde{\nu}\d\beta\d\alpha\\
			&= \int_0^1 \int_0^{2\pi} \begin{pmatrix}
				e^\alpha \sqrt{\alpha^2 \cos^2\parentheses*{\beta} + \alpha^2 \sin^2\parentheses*{\beta}}\\
				\alpha\cos\parentheses*{\beta}\\
				\alpha\sin\parentheses*{\beta}
			\end{pmatrix} \cdot \begin{pmatrix}
				\alpha\\
				-\alpha\cos\parentheses*{\beta}e^\alpha\\
				-\alpha\sin\parentheses*{\beta}e^\alpha
			\end{pmatrix}\d\beta\d\alpha\\
			&= \int_0^1 \int_0^{2\pi} \alpha^2 e^\alpha - \underbrace{\parentheses*{\alpha^2 \cos^2\parentheses*{\beta}e^\alpha + \alpha^2 \sin^2\parentheses*{\beta}e^\alpha}}_{= a^2 e^a} \d\beta\d\alpha = 0
		\end{align*}
	\end{enumerate}


	\section{}

	\begin{quote}
		Given is the vector field \(f\) with
		\[
			f: \R^2 \to \R^2, \parentheses*{x, y} \mapsto f\parentheses*{x, y} = \parentheses*{x + y, y^2},
		\]
		and let \(B \subset \R^2\) be a triangle with vertices at \(\parentheses*{0, 0}\), \(\parentheses*{1, 0}\), and \(\parentheses*{1, 2}\).
		Verify the Gauss theorem by calculating the two integrals \(\int_{\partial B}f \cdot \nu\d s\) and \(\int_B \div f\d A\) and comparing them.
	\end{quote}

	The area \(B\) is bounded by the curves
	\begin{align*}
		\gamma_1\parentheses*{t} &= \begin{pmatrix}
			t\\
			0
		\end{pmatrix}, \quad t \in \brackets*{0, 1}, \quad \norm*{\gamma_1'} = 1, \quad \nu_1 = \frac{1}{\norm{\gamma_1'}}\begin{pmatrix}
			\frac{\d}{\d t}\gamma_{1, 2}\parentheses*{t}\\
			-\frac{\d}{\d t}\gamma_{1, 1}\parentheses*{t}
		\end{pmatrix} = \begin{pmatrix}
			0\\
			-1
		\end{pmatrix},\\
		\gamma_2\parentheses*{t} &= \begin{pmatrix}
			1\\
			t
		\end{pmatrix}, \quad t \in \brackets*{0, 2}, \quad \norm*{\gamma_2'} = 1, \quad \nu_2 = \frac{1}{\norm{\gamma_2'}}\begin{pmatrix}
			\frac{\d}{\d t}\gamma_{2, 2}\parentheses*{t}\\
			-\frac{\d}{\d t}\gamma_{2, 1}\parentheses*{t}
		\end{pmatrix} = \begin{pmatrix}
			1\\
			0
		\end{pmatrix},\\
		\gamma_3\parentheses*{t} &= \begin{pmatrix}
			1 - t\\
			2 - 2t
		\end{pmatrix}, \quad t \in \brackets*{0, 1}, \quad \norm*{\gamma_3'} = \sqrt{5}, \quad \nu_3 = \frac{1}{\norm*{\gamma_3'}}\begin{pmatrix}
			\frac{\d}{\d t}\gamma_{3, 2}\parentheses*{t}\\
			-\frac{\d}{\d t}\gamma_{3, 1}\parentheses*{t}
		\end{pmatrix} = \frac{1}{\sqrt{5}}\begin{pmatrix}
			-2\\
			1
		\end{pmatrix}.
	\end{align*}
	We can now calculate
	\begin{align*}
		\int_{\partial B} f \cdot \nu\d s &= \int_{\gamma_1}f \cdot \nu_1 \d s + \int_{\gamma_2}f \cdot \nu_2 \d s + \int_{\gamma_3}f \cdot \nu_3 \d s\\
		&= \int_0^1 \begin{pmatrix}
			t\\
			0
		\end{pmatrix}^\top \begin{pmatrix}
			0\\
			-1
		\end{pmatrix}\d t + \int_0^2 \begin{pmatrix}
			1 + t\\
			t^2
		\end{pmatrix}^\top \begin{pmatrix}
			1\\
			0
		\end{pmatrix}\d t + \int_0^1 \begin{pmatrix}
			3\parentheses*{1 - t}\\
			4\parentheses*{t^2 - 2t + 1}
		\end{pmatrix}^\top \begin{pmatrix}
			-2\\
			1
		\end{pmatrix}\d t\\
		&= \int_0^2 \parentheses*{1 + t}\d t + 2\int_0^1 \parentheses*{2t^2 - t - 1}\d t = \frac{7}{3}
	\end{align*}
	and
	\begin{align*}
		\int_B \div f\d A &= \int_0^1 \int_0^{2\pi}\div\begin{pmatrix}
			x + y\\
			y^2
		\end{pmatrix}\d y\d x\\
		&= \int_0^1 \int_0^{2x}\parentheses*{1 + 2y}\d y\d x\\
		&= \int_0^1 4x^2 + 2x\d x = \frac{7}{3}.
	\end{align*}
	Since \(\int_{\partial B}f \cdot \nu\d s = \int_B \div f\d A\) the Gauss theorem is verified.


	\section{}

	\begin{quote}
		Compute the gradient \(\nabla f\parentheses*{x}\) and Hessian \(\nabla^2 f\parentheses*{x}\) of the Rosenbrock function
		\[
			f\parentheses*{x} = 100\parentheses*{x_2 - x_1^2}^2 + \parentheses*{1 - x_1}^2.
		\]
		Prove that \(x^* = \parentheses*{1, 1}^\top\) is the only local minimizer of this function, by showing that the gradient of \(f\) vanishes only at \(x^*\) while the Hessian matrix at that point is positive definite.
	\end{quote}

	The gradient is
	\[
		\nabla f\parentheses*{x} = \begin{pmatrix}
			\frac{\partial f}{\partial x_1}\\
			\frac{\partial f}{\partial x_2}
		\end{pmatrix} = \begin{pmatrix}
			-400x_1\parentheses*{x_2 - x_1^2} - 2\parentheses*{1 - x_1}\\
			200\parentheses*{x_2 - x_1^2}
		\end{pmatrix}.
	\]
	For the Hessian matrix we obtain
	\[
		\nabla^2 f\parentheses*{x} = \begin{pmatrix}
			\frac{\partial^2 f}{\partial x_1^2} & \frac{\partial^2 f}{\partial x_1 \partial x_2}\\
			\frac{\partial^2 f}{\partial x_2 \partial x_1} & \frac{\partial^2 f}{\partial x_2^2}
		\end{pmatrix} = \begin{pmatrix}
			-400\parentheses*{x_2 - 3x_1^2} + 2 & -400x_1\\
			-400x_1 & 200
		\end{pmatrix}.
	\]
	We first want to obtain the roots of the gradient \(\nabla f\parentheses*{x}\), which gives us a nonlinear system of equations
	\begin{align*}
		-400x_1\parentheses*{x_2 - x_1^2} - 2\parentheses*{1 - x_1} &= 0,\\
		200\parentheses*{x_2 - x_1^2} &= 0.
	\end{align*}
	From the second equation we get \(x_2 = x_1^2\).
	Inserting this into the first equation leaves us with
	\[
		-400x_1\parentheses*{x_1^2 - x_1^2} - 2\parentheses*{1 - x_1} = 2\parentheses*{x_1 - 1} = 0 \iff x_1 = 1.
	\]
	Thus again using \(x_2 = x_1^2\) we know that \(x^* = \parentheses*{1, 1}^\top\) is the only valid candidate for a local minimizer of this function.
	Using Sylvester's criterion we can show that the Hessian matrix at that point is positive definite because all leading principal minors are positive
	\[
		H_1 = 802 > 0, \quad H_2 = \begin{vmatrix}
			802 & -400\\
			-400 & 200
		\end{vmatrix} = 802 \cdot 200 - \parentheses*{-400} \cdot \parentheses*{-400} = 400
	\]
	Because the Hessian matrix at \(x^*\) is positive definite, we have a local minimizer of \(f\) which is in fact the only one because it is the only root of the gradient.


	\section{}

	\begin{quote}
		Sei die Funktion \(f\)
		\[
			f: \R^2 \to \R, \parentheses*{x, y} \mapsto f\parentheses*{\bm{x}} = f\parentheses*{x, y} := \begin{pmatrix}
				x\\
				y
			\end{pmatrix}^\top \begin{pmatrix}
				2 & 1\\
				1 & 4
			\end{pmatrix}\begin{pmatrix}
				x\\
				y
			\end{pmatrix} + 3.
		\]
		\begin{enumerate}
			\item Bestimmen Sie die Matrix \(A \in \R^{2 \times 2}\) so, dass
			\[
				\nabla f\parentheses*{\bm{x}} = 2A\bm{x}, \quad \nabla^2 f\parentheses*{\bm{x}} = 2A.
			\]
			\item Hat die Funktion \(f\) ein globales Minimum \(\bm{x}^*\)?
		\end{enumerate}
	\end{quote}

	Wir formen zunächst die Funktionsgleichung in eine Form um, mit der sich einfacher rechnen lässt.
	Wir erhalten
	\[
		f\parentheses*{x, y} = \begin{pmatrix}
			x\\
			y
		\end{pmatrix}^\top \begin{pmatrix}
			2 & 1\\
			1 & 4
		\end{pmatrix}\begin{pmatrix}
			x\\
			y
		\end{pmatrix} + 3 = \begin{pmatrix}
			2x + y\\
			x + 4y
		\end{pmatrix}^\top \begin{pmatrix}
			x\\
			y
		\end{pmatrix} + 3 = 2x^2 + 2xy + 4y^2 + 3
	\]
	\begin{enumerate}
		\item Für den Gradient und die Hessematrix gilt
		\begin{align*}
			\nabla f\parentheses*{\bm{x}} &= \begin{pmatrix}
				\frac{\partial f\parentheses*{\bm{x}}}{\partial x}\\
				\frac{\partial f\parentheses*{\bm{x}}}{\partial y}
			\end{pmatrix} = \begin{pmatrix}
				4x + 2y\\
				2x + 8y
			\end{pmatrix} = 2\begin{pmatrix}
				2x + y\\
				x + 4y
			\end{pmatrix} = 2\begin{pmatrix}
				2 & 1\\
				1 & 4
			\end{pmatrix}\begin{pmatrix}
				x\\
				y
			\end{pmatrix}\\
			\nabla^2 f\parentheses*{\bm{x}} &= \begin{pmatrix}
				\frac{\partial^2 f\parentheses*{\bm{x}}}{\partial^2 x} & \frac{\partial^2 f\parentheses*{\bm{x}}}{\partial x\partial y}\\
				\frac{\partial^2 f\parentheses*{\bm{x}}}{\partial y\partial x} & \frac{\partial^2 f\parentheses*{\bm{x}}}{\partial^2 y}
			\end{pmatrix} = \begin{pmatrix}
				4 & 2\\
				2 & 8
			\end{pmatrix} = 2\begin{pmatrix}
				2 & 1\\
				1 & 4
			\end{pmatrix},
		\end{align*}
		d.h. wir erhalten die Matrix \(A = \begin{pmatrix}
			2 & 1\\
			1 & 4
		\end{pmatrix}\) für die die geforderten Bedingungen gelten.
		\item Wir suchen die Nullstellen des Gradienten, was uns das folgende Gleichungssystem beschert
		\[
			\parentheses*{\begin{array}{cc|c}
				4 & 2 & 0\\
				2 & 8 & 0
			\end{array}} \xrightarrow{I - 2 \cdot II} \parentheses*{\begin{array}{cc|c}
				0 & -14 & 0\\
				2 & 8 & 0
			\end{array}},
		\]
		woraus folgt, dass \(\bm{x}^* = \parentheses*{0, 0}^\top\) unser einziger Kandidat für ein globales Minimum ist.
		Mit dem Kriterium von Sylvester können wir zeigen, dass die Hessematrix positiv definit ist, denn alle Hauptminoren sind positiv:
		\begin{align*}
			H_1 &= 4 > 0,\\
			H_2 &= \begin{vmatrix}
				4 & 2\\
				2 & 8
			\end{vmatrix} = 4 \cdot 8 - 2 \cdot 2 = 28 > 0.
		\end{align*}
		Weil die Matrix positiv definit ist, haben wir also ein lokales Minimum.
		Da die Matrix für alle \(x, y\) die Gestalt
		\[
			\begin{pmatrix}
				4 & 2\\
				2 & 8
			\end{pmatrix}
		\]
		hat (und damit immer positiv definit ist), folgt automatisch, dass unser lokales Minimum auch das gesuchte globale Minimum ist.
	\end{enumerate}
\end{document}
