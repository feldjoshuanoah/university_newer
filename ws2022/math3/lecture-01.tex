\section{Lebesgue-Integration}


\subsection{Vorbemerkung zur Maßtheorie}

Es gibt verschiedene Situationen, in denen wir Mengen messen wollen.
Eine ist das Bestimmen von \emph{Flächen} und \emph{Volumen} im \(\R^n\).
Das geschieht mit dem Lebesgue-Maß.
Diese Maß ist natürlich verschiebungs-invariant, d.h. eine Fläche von einem Quadratmeter hat überall dasselbe Maß, egal wohin wir sie verschieben.
Sobald wir uns Sicherheit im Messen von Mengen erarbeitet haben, können wir das Lebesgue-Integral einführen, welches gewisse Schwierigkeiten des Riemann-Integrals bereinigt.
Die Lebesgue'schen Grenzwertsätze erlauben es, Integrale von Funktionenfolgen zu bestimmen.
Und wir können die sogenannten \(L^p\)-Normen einführen, die es u.a. ermöglichen, Funktionen in integralen Normen abzuschätzen.
Das ist für die Analysis und die numerische Analysis grundlegend.

Eine andere Situation ist das Bestimmen von \emph{Wahrscheinlichkeiten}.
Werfen wir einen Dart-Pfeil auf ein Ziel, so ist die Wahrscheinlichkeit, dass eine Fläche von einem Quadratmeter getroffen wird, vermutlich desto größer, je näher die Fläche am Ziel liegt.
Ein Wahrscheinlichkeitsmaß ist also in der Regel nicht verschiebungs-invariant.
Ein Hauptziel von Vorlesungen zur Wahrscheinlichkeitstheorie (stochastischen Analysis) ist der zentrale Grenzwertsatz, der besagt, dass der Mittelwert einer identisch verteilten, unabhängigen Folge von Zufallsvariablen normalverteilt ist.

Eine weitere Situation tritt beim \emph{Lösen partieller Differentialgleichungen} auf.
Um Lösungen zu beschreiben, braucht man in der Regel Maße, die auf Punkten oder Kurven konzentriert sind.
Das führt uns in der Vorlesung Mathematische Grundlagen IV zur Theorie der Distributionen, Faltungen und Fundamentallösungen.

Da Mengen eine überraschend komplizierte Gestalt haben können, beschränkt sich das Lebesgue-Maß auf die Klasse sogenannter \emph{Lebesgue-messbarer Mengen}.


\subsection{Lebesgue-Maß}

Für den Rest dieses Abschnitts schränken wir und auf das Lebesgue-Maß ein, und bezeichnen es der Kürze halber einfach als \emph{Maß}.
Wir betrachten Funktionen \(f: \parentheses*{a, b} \to \left[0, \infty\right)\).
Mit einer Zerlegung \(Z = \parentheses*{x_0, \ldots, x_m}\) von \(\parentheses*{a, b}\),
\[
	a = x_0 < \cdots < x_m = b,
\]
sind die Obersumme \(O_Z\parentheses*{f}\) und Untersumme \(U_Z\parentheses*{f}\) definiert:
\begin{align}
	O_Z^R\parentheses*{f} &:= \sum_{i = 1}^n \parentheses*{\sup_{x_{i - 1} < x < x_i}f\parentheses*{x}}\parentheses*{x_i - x_{i - 1}},\label{eq:1-1}\\
	U_Z^R\parentheses*{f} &:= \sum_{i = 1}^n \parentheses*{\inf_{x_{i - 1} < x < x_i}f\parentheses*{x}}\parentheses*{x_i - x_{i - 1}}.\label{eq:1-2}
\end{align}
Damit sind also die besten Annäherungen des Flächeninhaltes unter \(f\)
\begin{equation}\label{eq:1-3}
	O^R\parentheses*{f} := \inf_{Z \in \mathcal{Z}}O_Z^R\parentheses*{f}, \quad \text{beziehungsweise} \quad U^R\parentheses*{f} := \sup_{Z \in \mathcal{Z}}U_Z^R\parentheses*{f},
\end{equation}
wobei \(\mathcal{Z}\) alle möglichen Zerlegungen von \(\parentheses*{a, b}\) enthält.
Wir nennen \(f\) \emph{Riemann-integrierbar}, wenn
\begin{equation}
	U^R\parentheses*{f} = O^R\parentheses*{f} < \infty.
\end{equation}
In diesem Fall schreiben wir
\[
	\int_a^b f := \int_a^b f\parentheses*{x}\d x := U^R\parentheses*{f}.
\]
Wir betrachten nun eine Folge \(\braces*{f_k}_{k = 1}^\infty\) von Riemann-integrierbaren Funktionen, die an jeder Stelle \(x \in \parentheses*{a, b}\) konvergiert.
Das heißt, es existiert \(f\), sodass
\[
	\lim_{k \to \infty}f_k\parentheses*{x} = f\parentheses*{x}
\]
für alle \(x \in \parentheses*{a, b}\).
Dann stellen sich die Fragen:
Ist \(f\) Riemann-integrierbar?
Und falls ja, gilt
\begin{equation}
	\lim_{k \to \infty}\int_a^b f_k = \int_a^b f\text{?}
\end{equation}
Leider ist dies oft nicht der Fall.
Das klassische Gegenbeispiel ist die Dirichlet-Funktion \(f = \chi_Q\), wobei \(\chi_A\) die charakteristische Funktion zu einer Menge \(A\),
\begin{equation}\label{eq:1-6}
	\chi_A\parentheses*{x} = \begin{cases}
		1, & \text{falls }x \in A,\\
		0, & \text{falls }x \not\in A
	\end{cases}
\end{equation}
und \(Q = \Q \cap \parentheses*{0, 1}\) die Menge der rationalen Zahlen zwischen Null und Eins ist.
Da \(Q\) abzählbar ist, also
\[
	Q = \braces*{q_1, q_2, \ldots},
\]
können wir die Funktionenfolge \(\braces*{f_k}\) mit
\[
	f_k\parentheses*{x} = \begin{cases}
		1, & \text{falls }x \in \braces*{q_1, \ldots, q_k},\\
		0, & \text{sonst}
	\end{cases}
\]
definieren.
Dann gilt
\begin{equation}\label{eq:1-7}
	\lim_{k \to \infty}f_k\parentheses*{x} = \chi_Q\parentheses*{x}
\end{equation}
für alle \(x \in \parentheses*{0, 1}\), sowie
\begin{equation}\label{eq:1-8}
	O^R\parentheses*{f_k} = U^R\parentheses*{f_k} = 0,
\end{equation}
aber
\[
	0 = U^R\parentheses*{\chi_Q} \ne O^R\parentheses*{\chi_Q} = 1\text{!}
\]
Intuitiv würden wir wegen \eqref{eq:1-7} und \eqref{eq:1-8} folgern,
\[
	\int_0^1 \chi_Q = 0.
\]
Dies folgt hauptsächlich daraus, dass die Menge \(Q\) so klein ist.
(Während \(Q\) abzählbar ist, gibt es überabzählbar viele reelle Zahlen!)
Wir möchten also die Größe einer Menge genau messen können.
Dafür führen wir das \emph{Lebesgue-Maß} ein.

\emph{Motivation der Lebesgue'schen Maßtheorie im \(\R^n\):}
\begin{itemize}
	\item Wir wollen die Größe einer Menge messen.
	\item Wir wollen im \(\R^n\) integrieren.
\end{itemize}

Ein Maß \(\lambda\) ist eine Verallgemeinerung der Länge eines Intervalles in \(\R\), des Flächeninhaltes eines Rechteckes in \(\R^2\), und im Allgemeinen des Rauminhaltes eines Quaders \(I = \parentheses*{a_1, b_1} \times \cdots \times \parentheses*{a_n, b_n} \subset \R^n\).
Es ordnet jeder Teilmenge in einem Mengensystem \(\mathcal{S}\) in \(\R^n\) eine nicht negative reelle Zahl zu und sollte intuitiv die vier folgenden Eigenschaften haben:
\begin{enumerate}
	\item \emph{(Konsistenz mit dem Volumenmaß)} Für jeden Quader \(I = \parentheses*{a_1, b_1} \times \cdots \times \parentheses*{a_n, b_n}\) gilt
	\begin{equation}\label{eq:1-9}
		\lambda\parentheses*{I} = \vol\parentheses*{I} := \parentheses*{b_1 - a_1}\cdots\parentheses*{b_n - a_n}.
	\end{equation}
	\item \emph{(Translations-Invarianz)} Seien \(y \in \R^n\), \(A \subset \R^n\) und \(A + y := \braces*{x + y : x \in A}\).
	Dann gilt
	\begin{equation}
		\lambda\parentheses*{A + y} = \lambda\parentheses*{A}.
	\end{equation}
	\item \emph{(Monotonie)} Wenn \(B \subset A\), dann gilt
	\begin{equation}\label{eq:1-11}
		\lambda\parentheses*{B} \le \lambda\parentheses*{A}.
	\end{equation}
	\item \emph{(\(\sigma\)- oder abzählbare Additivität)} Seien \(\braces*{A_1, A_2, \ldots}\) paarweise disjunkte Mengen.\footnote{Die Mengen \(\braces*{A_1, A_2, \ldots}\) sind \emph{paarweise disjunkt}, wenn \(A_i \cap A_j = \emptyset\) für alle \(i \ne j\).}
	Dann ist
	\begin{equation}\label{eq:1-12}
		\lambda\parentheses*{\bigcup_{k = 1}^\infty A_k} = \sum_{k = 1}^\infty \lambda\parentheses*{A_k}.
	\end{equation}
\end{enumerate}

Das \emph{äußere Lebesgue-Maß} \(\lambda^*: \mathcal{P}\parentheses*{\R^n} \to \left[0, \infty\right)\), welches durch\footnote{Für eine Menge \(A\) bezeichnet \(\mathcal{P}\parentheses*{A} \equiv 2^A\) die Potenzmenge, d.h. die Menge aller Untermengen.}
\[
	\lambda^*\parentheses*{A} := \inf\braces*{\sum_{k = 1}^\infty \vol\parentheses*{I_k} : A \subset \bigcup_{k = 1}^\infty I_k\text{, wobei jedes }I_k\text{ ein Quader ist}}
\]
definiert ist, scheint ein sinnvoller Kandidat für ein Maß zu sein.
Es besitzt die ersten drei Eigenschaften, aber es erfüllt nur
\begin{equation}\label{eq:1-13}
	\lambda^*\parentheses*{\bigcup_{k = 1}^\infty A_k} \le \sum_{k = 1}^\infty \lambda^*\parentheses*{A_k}.
\end{equation}
Denn es gibt disjunkte Mengen \(A\) und \(B\), so dass
\begin{equation}\label{eq:1-14}
	\lambda^*\parentheses*{A \cup B} < \lambda^*\parentheses*{A} + \lambda^*\parentheses*{B},
\end{equation}
oder äquivalent gibt es (nicht unbedingt disjunkte) Mengen \(A\) und \(B\), so dass\footnote{\(B \setminus A := \braces*{x \in B : x \not\in A}\).}
\begin{equation}\label{eq:1-15}
	\lambda^*\parentheses*{A} < \lambda^*\parentheses*{A \cap B} + \lambda^*\parentheses*{A \setminus B}.
\end{equation}
Solche Beispiele liegen sehr tief.\footnote{Z.B. das Banach Tarski Paradox von 1924, bei welchem eine Kugel in zwei Teile zerlegt wird, welche jeweils das Maß der Originalkugel besitzen.}
Tatsächlich gibt es kein einziges Maß, das auf der ganzen Potenzmenge \(\mathcal{S} = \mathcal{P}\parentheses*{\R^n} \equiv 2^{\R^n}\) definiert ist und diese vier Eigenschaften hat.
Deshalb schließen wir Mengen solcher Art im Folgenden aus.
Die übrigen Mengen nennen wir \emph{Lebesgue-messbar}.

\begin{definition}[Lebesgue-messbare Menge]
	Eine Menge \(M\) heißt \emph{Lebesgue-messbar}, falls
	\[
		\lambda^*\parentheses*{A} = \lambda^*\parentheses*{A \cap M} + \lambda\parentheses*{A \setminus M}
	\]
	für alle Mengen \(A \subseteq \R^n\).
	Sei \(\mathcal{L}\) das System der Lebesgue-messbaren Mengen, definieren wir das \emph{Lebesgue-Maß} \(\lambda: \mathcal{L} \to \left[0, \infty\right)\) durch
	\[
		\lambda\parentheses*{M} = \lambda^*\parentheses*{M}.
	\]
\end{definition}

Dieses Mengensystem \(\mathcal{L}\) ist ausreichend zu betrachten, weil wir Mengen wie \(A\) und \(B\) in \eqref{eq:1-14} und \eqref{eq:1-15} im Weiteren nicht begegnen werden.

\begin{proposition}
	Die Lebesgue-messbaren Mengen \(\mathcal{L}\) bilden eine \emph{\(\sigma\)-Algebra} über \(\R^n\), das heißt:
	\begin{enumerate}
		\item \(\emptyset \in \mathcal{L}\),
		\item ist \(M \in \mathcal{L}\), so ist auch \(\R^n \setminus M \in \mathcal{L}\), und
		\item ist \(M_k \in \mathcal{L}\) für alle \(k \in \N\), so ist auch
		\[
			\bigcup_{k = 1}^\infty M_k \in \mathcal{L}.
		\]
	\end{enumerate}
	Außerdem erfüllt das Lebesgue-Maß die vier Eigenschaften \eqref{eq:1-9} -- \eqref{eq:1-12}.
\end{proposition}

Um nun beantworten zu können, wie klein die Menge \(Q\) ist, müssen wir zunächst kontrollieren, ob \(Q\) messbar ist.
Weil wir wegen \eqref{eq:1-7} und \eqref{eq:1-8} erwarten, dass das Maß von \(Q\) Null ist, betrachten wir sogenannte Nullmengen:
Eine Menge \(A \subset \R^n\) heißt \emph{(Lebesgue-)Nullmenge}, falls \(\lambda^*\parentheses*{A} = 0\).

\begin{lemma}
	Jede Nullmenge ist messbar.
\end{lemma}

\begin{proof}
	Sei \(M \subset \R^n\) mit \(\lambda^*\parentheses*{M} = 0\) und \(A \subset \R^n\) eine beliebige Menge.
	Es folgt
	\[
		\lambda^*\parentheses*{A} \le \lambda^*\parentheses*{A \cap M} + \lambda^*\parentheses*{A \setminus M}
	\]
	aus \eqref{eq:1-13}.
	Wegen Monotonie \eqref{eq:1-11} folgt \(\lambda^*\parentheses*{A \cap M} = 0\) aus \(\lambda^*\parentheses*{M} = 0\) und \(A \cap M \subseteq M\), sowie \(\lambda^*\parentheses*{A \setminus M} \le \lambda^*\parentheses*{A}\) aus \(A \setminus M \subseteq A\).
	Also gilt
	\[
		\lambda^*\parentheses*{A} \ge \lambda^*\parentheses*{A \setminus M} = \lambda^*\parentheses*{A \cap M} + \lambda^*\parentheses*{A \setminus M}.
	\]
\end{proof}

\begin{example}
	Die Menge \(Q\) sowie alle abzählbaren Mengen sind Mullmengen:
	Sei \(M = \braces*{x_1, x_2, \ldots}\) eine beliebige abzählbare Menge in \(\R\) und betrachte die Überdeckung
	\[
		M \subset \bigcup_{k = 1}^\infty \brackets*{x_k, x_k + \varepsilon \cdot 2^{-k}}
	\]
	für alle \(\varepsilon > 0\).
	Dann
	\[
		\lambda^*\parentheses*{M} \le \vol\parentheses*{\bigcup_{k = 1}^\infty \brackets*{x_k, x_k + \varepsilon \cdot 2^{-k}}} \le \varepsilon\sum_{k = 1}^\infty 2^{-k} = \varepsilon,
	\]
	woraus wir folgern, dass \(M\) eine Nullmenge sein muss.
\end{example}

\begin{remark}\label{rem:1-5}
	Abzählbare Vereinigungen von Nullmengen sind wieder Nullmengen.
	Demnach ist zum Beispiel die Menge \(\Q\) auch eine Nullmenge.
\end{remark}


\subsection{Lebesgue-Integration}

Sei \(f: M \to W\), wobei der Wertbereich \(W \subseteq \R\) und \(M \subset \R^n\) eine messbare Menge ist.
Zuerst betrachten wir den Fall \(\lambda\parentheses*{M} < \infty\) und \(W = \brackets*{a, b} \subset \left[0, \infty\right)\), also eine begrenzte nicht negative Funktion auf einer Definitionsmenge von endlichem Maß.
Um ihr Integral zu definieren, betrachten wir im ersten Schritt Funktionen, die wir inuitiv genau integrieren können.

\emph{Schritt 1:} Für den Fall der Riemann-Integration haben wir die \emph{Treppenfunktionen} betrachtet.
Dies ist eine Funktion, die sich als eine lineare Kombination von charakteristischen Funktion (s. \eqref{eq:1-6}) schreiben lässt:
\begin{equation}\label{eq:1-16}
	\psi\parentheses*{x} = \sum_{i = 1}^m y_i \chi_{Q_i}\parentheses*{x},
\end{equation}
wobei jedes \(Q_i \subset M\) ein Quader ist.
Das Integral von \(\psi\) ist also trivialerweise
\begin{equation}
	\int_M \psi = \sum_{i = 1}^m y_i \vol\parentheses*{Q_i}.
\end{equation}
Nun, da wir aber das Maß von mehr als nur Quadern kennen, können wir die Treppenfunktion \eqref{eq:1-16} verallgemeinern, indem wir ihre Quader durch messbare Mengen ersetzen.

\emph{Schritt 2:} Eine soche Funktion heißt eine \emph{einfache Funktion}.
Eine Funktion \(f: M \to W\) nennt man eine einfache Funktion, wenn
\begin{enumerate}
	\item \(W = \braces*{y_1, \ldots, y_m}\),
	\item die Urbilder \(M_i := f^{-1}\parentheses*{\braces*{y_i}} := \braces*{x \in M : f\parentheses*{x} = y_i}\) messbar sind, i.e. \(M_i \in \mathcal{L}\).
\end{enumerate}
Daher lässt sich eine einfache Funktion folgendermaßen schreiben:
\begin{equation}
	\phi\parentheses*{x} = \sum_{i = 1}^m y_i \chi_{M_i}\parentheses*{x}.
\end{equation}
Das Integral von \(\phi\) ist also mit
\begin{equation}\label{eq:1-19}
	\int_M \phi := \sum_{i = 1}^m y_i \lambda\parentheses*{M_i}
\end{equation}
sinnvoll definiert.

\begin{remark}
	Gleichung \eqref{eq:1-19} ist tatsächlich eine neue Definition, da zum Beispiel \(\chi_Q\) eine einfache Funktion ist, die wir aber bisher als nicht integrierbar angesehen haben.
	Das heißt, \(y_1 = 0\), \(y_2 = 1\) und \(M_1 = \parentheses*{0, 1} \setminus \Q\), \(M_2 = Q\) und
	\[
		\int_0^1 \chi_Q = 0 \cdot \underbrace{\lambda\parentheses*{\parentheses*{0, 1} \setminus \Q}}_{= \lambda\parentheses*{\parentheses*{0, 1}} = 1} + 1 \cdot \underbrace{\lambda\parentheses*{Q}}_{= 0} = 0.
	\]
\end{remark}

\emph{Schritt 3:} Als nächstes definieren wir das Integral einer nicht einfachen Funktion.
So wie wir in \eqref{eq:1-1} und \eqref{eq:1-2} Annäherungen mit Treppenfunktionen von oben und unten betrachtet haben, definieren wir nun soche Annäherungen mit einfachen Funktionen neu.
Diese einfachen Funktionen können mittels einer Zerlegung des Wertbereiches (statt der Definitionsmenge wie in der Herleitung des Riemann-Integrals) gebildet werden.
Dafür brauchen wir die \emph{Urbilder} von \(A \subseteq W\):
\[
	f^{-1}\parentheses*{A} := \braces*{x \in M : f\parentheses*{x} \in A}.
\]
Seien \(\braces*{y_0, \ldots, y_m}\) eine Zerlegung des Wertbereiches \(W = \braces*{a, b}\) und \(M_i\) die Mengen
\[
	M_i = f^{-1}\parentheses*{\left[y_{i - 1}, y_i\right)}, \quad i \in \braces*{1, \ldots, m}.
\]
Dann sind
\[
	\phi_U := \sum_{i = 1}^m y_{i - 1}\chi_{M_i} \quad \text{und} \quad \phi_O := \sum_{i = 1}^m y_i \chi_{M_i}
\]
Funktionen, die sich \(f\) von unten beziehungsweise oben annähern.
Aber diese sind nicht unbedingt einfache Funktionen, weil die Urbilder nicht unbedingt messbar sind!
Wir beschränken uns also auf solche Funktionen.

\begin{definition}[Lebesgue-messbare Funktion]
	Die Funktion \(f: M \to W\) heißt \emph{Lebesgue-messbar}, falls \(f^{-1}\parentheses*{\left[y_1, y_2\right)}\) messbar für alle Intervalle \(\left[y_1, y_2\right) \subseteq W\) ist.
\end{definition}

\begin{remark}
	Eine äquivalente Definition ist, dass \(f\) Lebesgue-messbar heißt, falls \(f^{-1}\parentheses*{\parentheses*{-\infty, y}}\) messbar für alle \(y \in W\) ist.
	Die Äquivalenz folgt aus der Tatsache, dass das Mengensystem der messbaren Mengen eine \(\sigma\)-Algebra bildet.
\end{remark}

Wir werden keine andere Art von Messbarkeit betrachten, also sagen wir im Folgenden lediglich \emph{messbar} statt Lebesgue-messbar.
Die besten Annäherungen sind so wie in \eqref{eq:1-3} mittels Infimum und Supremum definiert:
\begin{align}
	O\parentheses*{f} &:= \inf\braces*{\int_M \phi : \phi\text{ ist eine einfache Funktion mit }\phi \ge f},\\
	U\parentheses*{f} &:= \sup\braces*{\int_M \phi : \phi\text{ ist eine einfache Funktion mit }\phi \le f}.
\end{align}
Es stellt sich heraus, dass messbare Funktionen genau die Funktionen sind, für die dieses neue Integral wohldefiniert ist.

\begin{proposition}
	Sei \(f: M \to W\) Lebesgue-messbar mit \(\lambda\parentheses*{M} < \infty\) und \(W \subset \left[0, \infty\right)\) beschränkt, dann gilt \(O\parentheses*{f} = U\parentheses*{f}\), und wir schreiben wie zuvor
	\[
		\int_M f := \int_M f\parentheses*{x}\d x := U\parentheses*{f}.
	\]
\end{proposition}

Außerdem haben wir das Problem nicht mehr, dass eine Funktionenfolge von integrierbaren Funktionen gegen eine nicht integrierbare Funktion konvergieren kann.
Um dieses Ergebnis im Allgemeinen zu erklären, brauchen wir die folgende Definition.

\begin{definition}[Fast überall bzw. ``a.e.'']
	Eine Relation gilt \emph{fast überall} oder \emph{an fast jeder Stelle} (abgekürzt \emph{a.e.} aus dem Englischen ``almost everywhere'') auf \(M\), falls es eine Nullmenge \(M_0\) gibt, so dass die Relation auf \(M \setminus M_0\) gilt.
\end{definition}

\begin{example}
	Sei \(f: \brackets*{0, 1} \to \brackets*{0, 1}\) eine messbare Funktion und \(g := f + \chi_Q\).
	Dann gilt
	\[
		f = g \quad \text{a.e.}
	\]
	Daraus folgt zum Beispiel
	\[
		\int_0^1 f = \int_0^1 g.
	\]
\end{example}

\begin{proposition}\label{prop:1-12}
	Sei \(\braces*{f_1, f_2, \ldots}\) eine Folge von Lebesgue-messbaren Funktionen auf \(M\), die an fast jeder Stelle gegen die Funktion \(f\) konvergiert.
	Dann ist \(f\) Lebesgue-messbar.
\end{proposition}

\begin{proof}
	Zu zeigen ist, dass die Urbilder \(f^{-1}\parentheses*{\parentheses*{-\infty, y}}\) messbar für jedes \(y \in \R\) sind.
	Wir müssen nur die Stellen betrachten, an denen die Funktionenfolge konvergiert, weil die anderen Stellen nur eine Nullmenge bilden und die immer messbar sind.
	Zuerst nehmen wir ein beliebiges \(x \in f^{-1}\parentheses*{\parentheses*{-\infty, y}}\).
	Wegen der Konvergenz gibt es zwei natürliche Zahlen \(m, j\), so dass
	\[
		f_k\parentheses*{x} < y - \frac{1}{m}
	\]
	für alle \(k \ge j\).
	Also gilt
	\begin{equation}\label{eq:1-22}
		x \in \bigcup_{k = j}^\infty f_k^{-1}\parentheses*{\parentheses*{-\infty, y - \frac{1}{m}}} =: M_{m, j}\parentheses*{y}.
	\end{equation}
	Da jede Funktion \(f_k\) messbar ist und das Mengensystem der messbaren Mengen eine \(\sigma\)-Algebra bildet, ist \(M_{m, j}\parentheses*{y}\) messbar.
	Ebenso lässt sich für jedes \(x\) ein Paar \(m\) und \(j\) finden, so dass \eqref{eq:1-22} gilt.
	Daher gilt
	\[
		f^{-1}\parentheses*{\parentheses*{-\infty, y}} \subseteq \bigcup_{m = 1}^\infty \bigcup_{j = 1}^\infty M_{m, j}\parentheses*{y} =: M\parentheses*{y}.
	\]
	Nochmals folgern wir, dass \(M\parentheses*{y}\) messbar ist, da das Mengensystem der messbaren Mengen eine \(\sigma\)-Algebra bildet.
	Das heißt, dass wir aus \(x \in f^{-1}\parentheses*{\parentheses*{-\infty, y}}\) die Aussage \(x \in M\parentheses*{y}\) folgern und daher \(M\parentheses*{y} \subseteq f^{-1}\parentheses*{\parentheses*{-\infty, y}}\).
	Jetzt können wir \(f^{-1}\parentheses*{\parentheses*{-\infty, y}} = M\parentheses*{y}\) folgern, indem wir zeigen, dass \(f^{-1}\parentheses*{\parentheses*{-\infty, y}} \supseteq M\parentheses*{y}\) gilt.
	Sei \(x\) ein beliebiges Element von \(M\parentheses*{y}\).
	Es gibt also ein Paar \(m\) und \(j\), so dass \(x \in M_{m, j}\parentheses*{y}\).
	Aber das heißt, dass
	\[
		f_k\parentheses*{x} < y - \frac{1}{m}
	\]
	für alle \(k \ge j\) und demnach
	\[
		f\parentheses*{x} = \lim_{k \to \infty}f_k\parentheses*{x} \le y - \frac{1}{m} < y,
	\]
	also \(x \in f^{-1}\parentheses*{\parentheses*{-\infty, y}}\).
\end{proof}

Dieses Integral kann nun auch ganz einfach für Funktionen verallgemeinert werden, deren Definitionsmenge unendliches Maß hat und deren Wertebereich nicht nur \(\left[0, \infty\right)\) ist.

\begin{definition}\label{definition:13}
	Sei \(f: M \to W\) eine Lebesgue-messbare Funktion.
	\begin{enumerate}
		\item Wenn \(\lambda\parentheses*{M} = \infty\), ist das Integral mit
		\begin{align*}
			\int_M f := \sup\Bigg\{\int_M \phi : &\phi\text{ ist eine einfache Funktion mit }\phi \le f\\
			&\text{und es gibt ein }M_0\text{ mit }\lambda\parentheses*{M_0} < \infty\text{, so dass }\left.\phi\right|_{M \setminus M_0} \equiv 0
		\end{align*}
		definiert.
		\item Wenn \(W \not\subseteq \left[0, \infty\right)\) (\(f\) kann also auch negative Werte annehmen), ist das Integral mit
		\[
			\int_M f := \int_M f_+ - \int_M f_-
		\]
		definiert, wobei
		\[
			f_+\parentheses*{x} := \max\braces*{f\parentheses*{x}, 0} \quad \text{und} \quad f_-\parentheses*{x} := -\min\braces*{f\parentheses*{x}, 0}.
		\]
		\item Die Funktion \(f\) heißt \emph{Lebesgue-integrierbar}, falls
		\[
			\int_M \absolute*{f} < \infty.
		\]
	\end{enumerate}
\end{definition}
Das Lebesgue-Integral ist tatsächlich eine Verallgemeinerung des Riemann-Integrals.

\begin{proposition}
	Sei \(f\) Riemann-integrierbar.
	Dann ist \(f\) messbar, und die Riemann- und Lebesgue-Integrale stimmen überein.
\end{proposition}

Das Lebesgue-Integral besitzt außerdem die erwartete Eigenschaft von Linearität.

\begin{proposition}[Linearität des Lebesgue-Integrales]\label{prop:1-15}
	Seien \(f\) und \(g\) Lebesgue-messbare Funktionen auf \(M\) und \(\alpha\) und \(\beta\) reelle Zahlen.
	Dann gilt
	\begin{equation}
		\int_M \parentheses*{\alpha f + \beta g} = \alpha\int_M f + \beta\int_M g.
	\end{equation}
\end{proposition}

Es könnte aber immer geschehen, dass der Limes der Integrale einer Funktionenfolge nicht das Integral des Limes der Funktionenfolge ist.
Der durchschlagende Erfolg der Lebesgue'schen Theorie basiert auf den folgenden drei Theoremen über die Integrale von Funktionenfolgen:

\begin{proposition}[Lebesgues Satz von der monotonen Konvergenz]\label{prop:1-16}
	Sei \(E \subset \R^n\) messbar, und \(\parentheses*{f_k}_{k \in \N}\) eine Folge messbarer Funktionen mit
	\begin{equation}
		0 \le f_1\parentheses*{x} \le f_2\parentheses*{x} \le \cdots
	\end{equation}
	und Grenzwert
	\begin{equation}
		f\parentheses*{x} := \lim_{k \to \infty}f_k\parentheses*{x}
	\end{equation}
	für alle \(x \in E\).
	Dann ist \(f\) auf \(E\) messbar, und es gilt
	\begin{equation}\label{eq:1-26}
		\lim_{k \to \infty}\int_E f_k = \int_E f.
	\end{equation}
\end{proposition}

\begin{lemma}[Fatous Lemma]\label{lem:1-17}
	Sei \(E \subset \R^n\) messbar, und \(\parentheses*{f_k}_{k \in \N}\) eine Folge nicht negativer messbarer Funktionen.
	Setze
	\begin{equation}
		f\parentheses*{x} := \liminf_{k \to \infty}f_k\parentheses*{x}
	\end{equation}
	für \(x \in E\).
	Dann ist \(f\) auf \(E\) messbar, und es gilt
	\begin{equation}
		\int_E \le \liminf_{k \to \infty}\int_E f_k.
	\end{equation}
\end{lemma}

\begin{proposition}[Lebesgues Satz von der majorisierten Konvergenz]
	Sei \(E \subset \R^n\) messbar und \(\parentheses*{f_k}_{k \in \N}\) eine Folge von Lebesgue-messbaren Funktionen auf \(E\) und \(g\) eine Lebesgue-integrierbare Funktion auf \(E\) mit \(\absolute*{f_k} \le g\) fast überall auf \(E\) für alle \(k \ge 1\).
	Wenn \(f_k\) an fast jeder Stelle gegen eine Funktion \(f\) konvergiert, dann ist \(f\) Lebesgue-integrierbar und
	\begin{equation}
		\lim_{k \to \infty}\int_E f_k = \int_E f.
	\end{equation}
\end{proposition}

\begin{example}
	Seien \(E = \brackets*{0, 1}\) und
	\[
		f_k\parentheses*{x} = k\chi_{\brackets*{0, \frac{1}{k}}}.
	\]
	Dann konvergiert die Funktionenfolge \(\braces*{f_k}\) zwar fast überall gegen \(f\parentheses*{x} \equiv 0\), aber
	\[
		\lim_{k \to \infty}\int_0^1 f_k = 1.
	\]
	Hier wäre die Funktion \(g\parentheses*{x} := \frac{1}{x}\) eine Majorante, die aber nicht integrierbar ist.
\end{example}

Wir beweisen zum Abschluss den Satz der monotonen Konvergenz.
Beim Beweis spielen viele der eingeführten Techniken zusammen.

\begin{proof}
	\emph{(von Satz \ref{prop:1-16})} Nach Satz \ref{prop:1-12} ist \(f\) messbar.
	Nach Definition \ref{definition:13} gilt
	\[
		\int_E f = \sup\braces*{\int_E \phi : \phi \le f, \phi\text{ einfach}}.
	\]
	Wähle eine feste nicht-negative einfache Funktion \(\phi\), so dass \(0 \le \phi \le f\).
	Da \(\phi\) einfach ist, schreibt sich \(\phi\) als \(\phi = \sum_{i = 1}^m y_i \chi_{M_i}\).
	Wähle zudem eine beliebige Konstante \(0 < c < 1\), und setze
	\[
		E_k := \braces*{x \in E : f_k\parentheses*{x} \ge c\phi\parentheses*{x}}.
	\]
	Dann gilt \(E_1 \le \cdots \le E_k\) und \(\bigcup_{k = 1}^\infty E_k = E\) (bis auf eine Nullmenge).
	Außerdem gilt
	\begin{equation}
		\int_E f_k \ge \int_{E_k}f_k \ge c\int_{E_k}\phi.
	\end{equation}
	Daraus folgt
	\[
		\lim_{k \to \infty}\int_E f_k \ge c\lim_{k \to \infty}\int_{E_k}\phi = c\int_E \phi
	\]
	und mit \(c \to 1\)
	\[
		\lim_{k \to \infty}\int_E f_k \ge \int_E \phi,
	\]
	also
	\[
		\int_E f = \sup_\phi \int_E \phi \le \lim_{k \to \infty}\int_E f_k.
	\]
	Selbstverständlich gilt auch
	\[
		\lim_{k \to \infty}\int_E f_k \le \int_E f,
	\]
	also die Gleichheit \eqref{eq:1-26}.
\end{proof}


\subsection{Die $L^p$-Räume}

Wir nennen eine messbare Funktion \(f\) \emph{\(p\)-integrierbar auf \(M\)}, falls
\[
	\int_M \absolute*{f}^p < \infty.
\]
Dieses \(p\) kann gewissermaßen beschreiben, wie die Funktion \(f\) aussieht.

\begin{example}
	Sei \(f\parentheses*{x} := \absolute*{x}^{-\alpha}\) für \(\alpha \in \parentheses*{0, 1}\).
	\begin{enumerate}
		\item Wenn \(M = \parentheses*{-1, 1}\), ist \(f\) \(p\)-integrierbar für \(p < \frac{1}{\alpha}\).
		Also beschreibt \(p\), welche Singularitäten die Funktion haben kann.
		\item Wenn \(M = \parentheses*{1, \infty}\) ist \(f\) \(p\)-integrierbar für \(p > \frac{1}{\alpha}\).
		Also beschreibt \(p\), wie schnell die Funktion für \(x \to \infty\) gegen Null konvergiert.
	\end{enumerate}
\end{example}

Die wichtigsten Fälle sind \(p \in \braces*{1, 2, \infty}\), aber in anderen Vorlesungen werden wir auch Gleichungen betrachten, deren Lösung \(p\)-integrierbar für weitere \(p\) ist.

Da nach Satz \ref{prop:1-15} Lebesgue-Integration linear ist, hat die Menge der integrierbaren Funktionen die Struktur eines Vektorraums.
Dies gilt auch für \(p > 1\), da
\[
	\absolute*{f + g}^p \le \parentheses*{\absolute*{f} + \absolute*{g}}^p \le \parentheses*{2\max\braces*{\absolute*{f}, \absolute*{g}}}^p = 2^p \max\braces*{\absolute*{f}^p, \absolute*{g}^p} \le 2^p \parentheses*{\absolute*{f}^p + \absolute*{g}^p},
\]
und demnach gilt
\[
	\int_M \absolute*{f + g}^p \le 2^p \parentheses*{\int_M \absolute*{f}^p + \int_M \absolute*{g}^p}.
\]
Also wenn \(f\) und \(g\) \(p\)-integrierbar sind, ist ebenso \(f + g\) \(p\)-integrierbar.

Eine mögliche Norm für den Vektorraum der \(p\)-integrierbaren Funktionen ergibt sich natürlicherweise:
\[
	\norm*{f}_p := \parentheses*{\int_M \absolute*{f}^p}^{\frac{1}{p}}.
\]
Diese Norm wird die \emph{\(p\)-Norm} ernannt.
Aber erfüllt \(\norm*{\cdot}_p\) überhaupt die Voraussetzungen einer Norm?
Zum Einen muss eine Norm für jedes \(\alpha \in \R\)
\[
	\norm*{\alpha f}_p = \absolute*{\alpha}\norm*{f}_p
\]
erfüllen.
Dies gilt trivialerweise.
Zum Zweiten muss eine Norm
\[
	\norm*{f}_p = 0 \iff f = 0
\]
erfüllen.
Dies folgt jedoch nicht unmittelbar, da \(\norm*{f}_p = 0\) für \emph{alle} \(f\) gilt, die \emph{fast überall} Null sind.
Wir können aber sagen, dass eine solche Funktion ``so gut wie Null sei'', indem wir die Äquivalenzrelation ``\(\simeq\)'' mit
\[
	f \simeq 0 \iff f = 0 \quad \text{a.e.}
\]
einführen.
Im Allgemeinen ist die Äquivalenzrelation mit
\[
	f \simeq g \iff f = g \quad \text{a.e.}
\]
definiert.
Damit zerfällt die Menge der \(p\)-integrierbaren Funktionen in Äquivalenzklassen
\[
	\brackets*{f}_\simeq := \braces*{g : g\text{ ist }p\text{-integrierbar und }f \simeq g}.
\]
Nun gilt
\[
	\norm*{f}_p = 0 \iff f \in \brackets*{0}_\simeq.
\]

\begin{definition}[\(L^p\)-Raum]
	Die Menge der Äquivalenzklassen bezüglich \(\simeq\) der \(p\)-integrierbaren Funktionen auf \(M\),
	\[
		L^p\parentheses*{M} := \braces*{\brackets*{f}_\simeq : \int_M \absolute*{f}^p < \infty},
	\]
	heißt \emph{\(L^p\)-Raum}.
	Die \(p\)-Norm einer Äquivalenzklasse ist mit
	\begin{equation}\label{eq:1-31}
		\norm*{\brackets*{f}_\simeq}_p := \norm*{f}_p
	\end{equation}
	definiert.
	Selbst wenn wir tatsächlich \(\brackets*{f}_\simeq\) meinen, schreiben wir ab jetzt lediglich \(f\).
\end{definition}

Leicht nachzuprüfen sind:
\begin{enumerate}
	\item \(L^p\parentheses*{M}\) ist ein Vektorraum,
	\item mit \eqref{eq:1-31} ist die \(p\)-Norm einer Äquivalenzklasse eindeutig definiert.
\end{enumerate}
Wenn die Definitionsmenge \(M\) im Zusammenhang klar ist, schreiben wir lediglich \(L^p\).

Letztlich muss eine Norm die Dreiecksungleichung erfüllen.
Um dies zu zeigen, brauchen wir die folgenden drei Lemmas.

\begin{lemma}[Young-Ungleichung]
	Für vier positive reelle Zahlen \(a, b, p, q > 0\) mit
	\[
		\frac{1}{p} + \frac{1}{q} = 1
	\]
	gilt
	\begin{equation}\label{eq:1-32}
		ab \le \frac{a^p}{p} + \frac{b^q}{q}.
	\end{equation}
\end{lemma}

\begin{proof}
	Dies folgt aus der Tatsache, dass \(\ln x\) monoton wachsend und konkav ist, denn
	\[
		\ln\parentheses*{ab} = \ln a + \ln b = \frac{1}{p}\ln a^p + \frac{1}{q}\ln b^q \le \ln\parentheses*{\frac{a^p}{p} + \frac{b^q}{q}}.
	\]
\end{proof}

\begin{lemma}[Hölder-Ungleichung]
	Seien \(1 < p < \infty\) und \(q\) mit
	\begin{equation}\label{eq:1-33}
		\frac{1}{p} + \frac{1}{q} = 1.
	\end{equation}
	Falls \(f \in L^p\) und \(g \in L^q\), gelten \(fg \in L^1\) und
	\begin{equation}\label{eq:1-34}
		\norm*{fg}_1 \le \norm*{f}_p \norm*{g}_q.
	\end{equation}
\end{lemma}

\begin{proof}
	Es reicht aus, den Fall \(\norm*{f}_p > 0, \norm*{g}_q > 0\) zu betrachten.
	Aus der Young-Ungleichung folgt
	\[
		\int_M \frac{\absolute*{f\parentheses*{x}}}{\norm*{f}_p}\frac{\absolute*{g\parentheses*{x}}}{\norm*{g}_q}\d x \stackrel{\eqref{eq:1-32}}{\le} \int_M \parentheses*{\frac{1}{p}\frac{\absolute*{f\parentheses*{x}}^p}{\norm*{f}_p^p} + \frac{1}{q}\frac{\absolute*{g\parentheses*{x}}^q}{\norm*{g}_q^q}} = \frac{1}{p} + \frac{1}{q} \stackrel{\eqref{eq:1-33}}{=} 1.
	\]
\end{proof}

\begin{lemma}[Minkowski-Ungleichung]
	Seien \(1 \le p < \infty\) und \(f, g \in L^p\).
	Dann gilt
	\[
		\norm*{f + g}_p \le \norm*{f}_p + \norm*{g}_p.
	\]
\end{lemma}

\begin{proof}
	Der Fall \(p = 1\) ist wegen der Dreiecksungleichung des Absolutbetrages klar.
	Sei \(1 < p < \infty\) und \(q\) wie zuvor mit \(\frac{1}{p} + \frac{1}{q} = 1\).
	Die Summe \(f + g \in L^p\) erfüllt mit Hilfe der Dreiecksungleichung des Absolutbetrages
	\begin{equation}\label{eq:1-35}
		\norm*{f + g}_p^p = \int_M \absolute*{f + g}\absolute*{f + g}^{p - 1} \le \norm*{f\absolute*{f + g}^{p - 1}}_1 + \norm*{g\absolute*{f + g}^{p - 1}}_1.
	\end{equation}
	Nun möchten wir die Hölder-Ungleichung verwenden, aber wir müssen erst kontrollieren, ob \(\parentheses*{f + g}^{p - 1} \in L^q\) gilt.
	Tatsächlich, wenn \(h \in L^p\) liegt, gilt
	\begin{equation}\label{eq:1-36}
		\norm*{h^{p - 1}}_q = \parentheses*{\int_M \absolute*{h}^{pq - q}}^{\frac{1}{q}} = \parentheses*{\int_M \absolute*{h}^p}^{1 - \frac{1}{p}} = \norm*{h}_p^{p - 1} < \infty.
	\end{equation}
	Also können wir mit der Hölder-Ungleichung aus \eqref{eq:1-35} folgern:
	\begin{align*}
		\norm*{f + g}_p^p &\stackrel{\eqref{eq:1-34}}{\le} \norm*{f}_p \norm*{\parentheses*{f + g}^{p - 1}}_q + \norm*{g}_p \norm*{\parentheses*{f + g}^{p - 1}}_q\\
		&\stackrel{\eqref{eq:1-36}}{=} \norm*{f}_p \norm*{f + g}_p^{p - 1} + \norm*{g}_p \norm*{f + g}_p^{p - 1}\\
		&= \parentheses*{\norm*{f}_p + \norm*{g}_p}\norm*{f + g}_p^{p - 1}.
	\end{align*}
\end{proof}

\begin{lemma}
	Für \(1 \le p \le \infty\) ist \(L^p\) ein normierter Vektorraum.
\end{lemma}

\begin{proof}
	Für \(1 \le p < \infty\) folgt das aus der Minkowski-Ungleichung.
	Für \(p = \infty\) führen wir die Norm
	\begin{equation}
		\norm*{f}_\infty := \esssup_{x \in M}\absolute*{f\parentheses*{x}} := \inf\braces*{\sup_{x \in M \setminus M_0}\absolute*{f\parentheses*{x}} : M_0 \subseteq M\text{ und }\lambda\parentheses*{M_0} = 0}
	\end{equation}
	ein.\footnote{Hier bezeichnet ``\(\esssup\)'' das \emph{wesentliche Supremum} (englisch ``essential supremum'').}
	Dabei müssen wir darauf achten, dass wir uns mit Äquivalenzklassen von Funktionen beschäftigen.
	In einer einzelnen Äquivalenzklasse können die äquivalenten Funktionen verschiedene Werte auf einer Nullmenge annehmen.
	Man prüft leicht nach, dass dies tatsächlich die Voraussetzungen einer Norm erfüllt.
\end{proof}

Nun können wir drei verschiedene Arten von Konvergenz definieren:

\begin{definition}[Konvergenz von Funktionen]
	Sei \(\braces*{f_k}\) eine Folge von messbaren Funktionen.
	\begin{enumerate}
		\item Die Funktionenfolge konvergiert \emph{gleichmäßig auf \(M\)} gegen \(f\), falls
		\[
			\lim_{k \to \infty}\sup_{x \in M}\absolute*{f_k\parentheses*{x} - f\parentheses*{x}} = 0.
		\]
		Wir schreiben auch ``\(f_k \to f\) gleichmäßig''.
		\item Die Funktionenfolge konvergiert \emph{fast überall} (oder ``\emph{almost everywhere}'') \emph{punktweise} gegen \(f\), falls
		\[
			\lim_{k \to \infty}\absolute*{f_k\parentheses*{x} - f\parentheses*{x}} = 0 \quad \text{a.e.}
		\]
		Wir schreiben auch ``\(f_k \to f\) a.e.''.
		\item Die Funktionenfolge ist eine \emph{Cauchy-Folge}, falls für jedes \(\varepsilon \in \parentheses*{0, \infty}\) es ein \(N \in \N\) gibt, so dass
		\[
			\norm*{f_k - f_\ell}_p < \varepsilon
		\]
		für alle \(k, \ell \ge N\).
		Wir schreiben auch ``\(f_k \to f\) a.e.''.
		\item Die Funktionenfolge konvergiert in \(L^p\parentheses*{M}\) gegen \(f\), falls
		\[
			\lim_{k \to \infty}\norm*{f_k - f}_p = 0.
		\]
		Wir schreiben auch ``\(f_k \to f\) in \(L^p\parentheses*{M}\)''.
	\end{enumerate}
\end{definition}

Für die \(L^p\)-Räume sind natürlich die letzten Arten von Konvergenz am wichtigsten.
Tatsächlich ist die letzte gewünschte Eigenschaft eines normierten Vektorraums, dass jede Cauchy-Folge eine konvergente Folge in \(L^p\) ist.
Ist dies der Fall, heißt der Raum \emph{vollständig} oder ein \emph{Banach-Raum}.
Wir können dort zum Beispiel den Banach'schen Fixpunktsatz als Konvergenzkriterium verwenden.

\begin{proposition}[Riesz-Fischer]
	Für \(1 \le p \le \infty\) sind die \(L^p\)-Räume vollständig.
\end{proposition}

\begin{proof}
	Der Beweis für \(1 \le p < \infty\) ist eine zweifache Anwendung von Fatous Lemma \ref{lem:1-17}:
	Sei \(\parentheses*{f_n}_{n \in \N}\) eine Cauchy-Folge in \(L^p\).
	O.B.d.A. gelte
	\begin{equation}
		\norm*{f_n - f_{n + 1}} < 2^{-n}.
	\end{equation}
	Setze
	\begin{equation}
		g_k := \sum_{n = 1}^k \absolute*{f_{n + 1} - f_n}, \quad g := \lim_{k \to \infty}g_k.
	\end{equation}
	Die \(g_k\) sind gleichmäßig beschränkt in \(L^p\), denn
	\begin{equation}
		\norm*{g_k} \le \sum_{n = 1}^k 2^{-n} = \frac{1 - 2^{k + 1}}{1 - \frac{1}{2}} - 1 \le 1.
	\end{equation}
	Aus Fatous Lemma folgt
	\[
		\int\absolute*{g}^p \le \liminf_{k \to \infty}\int\absolute*{g_k}^p \le 1,
	\]
	also \(\norm*{g}_p \le 1\).
	Damit ist \(g\) fast überall beschränkt, \(\absolute*{g\parentheses*{x}} < \infty\) a.e.
	Die Reihe
	\begin{equation}
		S\parentheses*{x} := f_1\parentheses*{x} + \sum_{n = 1}^\infty \parentheses*{f_{n + 1}\parentheses*{x} - f_n\parentheses*{x}}
	\end{equation}
	konvergiert also a.e. absolut.
	In diesem Fall setze \(f\parentheses*{x} := S\parentheses*{x}\).
	Natürlich konvergiert die Folge \(\parentheses*{f_k}\) a.e. gegen \(f\).
	Wir zeigen nun, dass \(f\) in \(L^p\) liegt.
	Sei dafür \(\varepsilon > 0\) fest.
	Wähle \(N\) so groß, dass für \(n, m > N\)
	\[
		\norm*{f_n - f_m}_p \le \varepsilon.
	\]
	Mit Fatous Lemma folgt
	\[
		\int\absolute*{f - f_m}^p \le \liminf_{n \to \infty}\int\absolute*{f_n - f_m}^p \le \varepsilon^p.
	\]
	Also liegt \(f - f_m\) in \(L^p\), und wegen \(\norm*{f}_p \le \norm*{f - f_m}_p + \norm*{f}_p\) liegt auch \(f\) in \(L^p\).

	Der Beweis für \(p = \infty\) ist einfacher und braucht keine weiteren Hilfsmittel:
	Sei \(\parentheses*{f_n}_{n \in \N}\) eine \(L^\infty\)-Cauchy-Folge.
	Setze
	\begin{align*}
		A_k &:= \braces*{x : \absolute*{f_k\parentheses*{x}} > \norm*{f_k}_\infty},\\
		B_{m, n} &:= \braces*{x : \absolute*{f_m\parentheses*{x} - f_n\parentheses*{x}} > \norm*{f_m - f_n}_\infty}.
	\end{align*}
	Das sind Nullmengen und mit Bemerkung \ref{rem:1-5} ist auch
	\[
		E := \parentheses*{\bigcup_{k = 1}^\infty A_k} \cup \parentheses*{\bigcup_{m, n = 1}^\infty B_{m, n}}
	\]
	eine Nullmenge.
	Auf dem Komplement von \(E\) konvergieren die \(f_n\) gleichmäßig gegen \(f\).
	Die Funktion \(f\) ist a.e. beschränkt, und somit in \(L^\infty\).
\end{proof}
